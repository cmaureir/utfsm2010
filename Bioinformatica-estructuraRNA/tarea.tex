\documentclass[letter, 10pt]{article}
\usepackage[utf8]{inputenc}
\usepackage[spanish]{babel}
\usepackage{amsfonts}
\usepackage{amsmath}
\usepackage[top=1cm,bottom=1cm,left=2cm,right=2cm,footskip=1.5cm,headheight=1.5cm,headsep=.5cm,textheight=2cm]{geometry}
\usepackage{listings}
\usepackage{color}
\usepackage{graphicx}
\usepackage{hyperref}

\begin{document}
\bibliographystyle{plain}


\title{Bioinformática\\``Tarea: Predicción de estructura del RNA''}
\author{
Rodrigo Fernández \and Cristián Maureira \and Gabriel Zamora\\
\hspace{-8.5cm}		\texttt{\{rfernand,cmaureir,gzamora\}@inf.utfsm.cl}
}

\date{\today}
\maketitle

\section{Problema}
¿Que significa que los métodos de predicción de estructuras de RNA busquen
palíndromos, como la minimización de energía libre afecta estos resultados?
%Enviar a personaje.ts@gmail.com


\section{Respuesta}

\begin{itemize} 

\item ¿Que significa que los métodos de predicción de estructuras de RNA busquen
palíndromos? \\
% palíndromos en predicción de estructuras de RNA
%%%%%%%%%%%%%%%% PONER REFERENCIA http://genoma.unsam.edu.ar/trac/docencia/raw-attachment/wiki/Bioinformatica/Bibliografia/nbt1104-1457.pdf

Un palíndromo es una palabra o frase que se lee hacia adelante y hacia atrás,
por ejemplo ``aibohphobia''.

El significado de la búsqueda de palíndromos es que buscan el apareamiento
de bases de una estructura secundaria de RNA a través de la
construcción de palíndromos biológicos \cite{palindrome}.

Para ser más exactos, es necesario la construcción de 
palíndromos para la predicción de estructuras de RNA a través de apareamiento 
de los nucleótidos, por ejemplo \texttt{GGACU} emparejado con \texttt{ACUCC}.

Sin embargo en la realidad el hecho de buscar palíndromos en el RNA no es tan
simple, ``reengineer''  no es un palíndromo real, pero es análogo a que se
encontrara cuatro pares de base de RNA (``reen''/``neer'') con dos ciclos
basura (``gi''). ``Sniffinesses'' tampoco es un palíndromo, pero la totalidad
de sus letras pueden ser pareadas en tres grupos 
(``s''/``s'', ``nif''/``fin'' y ``es''/``se'').

Esto conlleva a que los algoritmos de predicción de RNA recurran
a la búsqueda de palíndromos no perfectos, ya que el procedimiento
no es tan simple como parece y es necesario poder buscar de todas
las formas posibles.

% sobre la minimización de energía libre

\item ¿Como la minimización de energía libre afecta estos resultados? \\

La desventaja de los métodos que buscan encontrar la mayor cantidad de bases
pareadas es que no necesariamente se encuentra la estructura más estable.~\cite{diapos}

Como el RNA se plegará de tal forma de maximizar la estabilidad de su
estructura, y la estabilidad de la misma depende directamente de la energía
libre que posea el RNA, el buscar la minimización de energía libre es un
método que aumente enormemente la precisión de la predicción realizada.

Sabemos que cuando se produce un apareamiento de bases,
la energía de molécula se reduce debido a las interacciones de atracción entre
las dos cadenas, pero a su vez, la estructuras que forman ciclos internos se
pueden definirse por la energía libre que contienen, por lo que para obtener la ``estabilidad'' nombrada anteriormente,
los programas se diseñan para encontrar una estructura que minimice la suma de
todos esos factores.

El problema que posee la minimización de la energía libre, es que debe considerar
una gran cantidad de variables, 
%(temperatura, todas las combinaciones de los enlaces y subestructuras que se puedan formar, etc.)
haciendo que el proceso sea de alto costo computacional.

\end{itemize}

\bibliography{tarea}

\vfill\hfill
RF/CM/GZ/\LaTeX
\end{document} 
