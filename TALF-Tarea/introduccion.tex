Una expresión regular,
a menudo llamada también patrón,
es una expresión que describe un conjunto de cadenas sin enumerar sus elementos.

Además,
normalmente representan otro grupo de caracteres mayor,
de tal forma que podemos comparar el patrón con otro conjunto de caracteres para ver las coincidencias.

La mayoría de las formalizaciones proporcionan los siguientes constructores:
una expresión regular es una forma de representar a los lenguajes regulares
(finitos o infinitos) y se construye utilizando caracteres del alfabeto
sobre el cual se define el lenguaje.

Específicamente,
las expresiones regulares se construyen utilizando los operadores
unión, concatenación y clausura de Kleene.

Las expresiones regulares en UNIX (RE's) están recogidas en el estándar POSIX 1003.2.

