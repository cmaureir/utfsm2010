\subsection{GREP}

\textbf{Definici\'on}\\

\emph{grep} es una utilidad de la línea de comandos escrita originalmente para ser usada con el sistema operativo Unix.
Usualmente,
\emph{grep} toma una expresión regular de la línea de comandos,
lee la entrada estándar o una lista de archivos,
e imprime las líneas que contengan coincidencias para la expresión regular.

Su nombre deriva de un comando en el editor de texto \emph{ed} que tiene la siguiente forma:
\begin{verbatim}
g/re/p
global/regular expression/print
\end{verbatim}
\textbf{nota:} \emph{se recomienda al lector que posea un sistema operativo,
basado en UNIX, obtener una documentación más robusta acerca de ``grep'',
mediante el ``man'', que su sistema operativo posee (man grep)}

\vspace{1cm}
\textbf{Ejemplos}\\

Acá tenemos algunos ejemplos que intentarán familiarizar al lector,
con el \emph{útil}, o mas bien, indispensable comando de \emph{UNIX}
como lo es \emph{grep}.

\begin{enumerate}
	\item \emph{grep 'hello' archivo}\\
	Desplegar todas las líneas del archivo que poseen la palabra 'hello'
	\item \emph{grep -v 'hello' archivo}
	Desplegar todas las líneas del archivo que NO poseen la palabra 'hello'
	\item \emph{grep -i 'hello.*world' menu.h main.c}\\
	Éste comando listará todas las lineas en los archivos \emph{menu.h}  \emph{main.c} que contienen
	el string \emph{hello} seguido por el string \emph{world};
	ésto se debe porque \emph{$.*$} encuentra 0 o mas caracteres entre líneas.
	(La opción ``-i'' provoca que grep ignore mayúsculas y minúsculas, ya las vea todas por igual)

	\item \emph{grep -l 'main' *.c}\\
	 Imprime o lista, sólo el nombre de los archivos
	 (que son de la forma \emph{nombre.c} )en que se produce el ``matching'',
	es decir que poseen la palabra ``main'', en su interior.
	 
	\item \emph{grep -r 'hello' /home/user}\\
	Busca dentro de directorios, es decir, recursivamente,
	la palabra ``hello'' dentro de los distintos archivos de cada directorio.

	\item \emph{grep -w 'hello' *}\\
	Buscar una palabra ``completa'', no una parte dentro de una cadena más grande

	\item \emph{grep 'hello$\backslash >$' *}\\
	Buscar las cadenas de caracteres que terminan con 'hello'.

	\item \emph{grep -C 2 'hello' *}\\
	Buscamos la palabra 'hello', pero al momento de desplegar dicha información por salida
	estándar, mostramos 2 lineas de contexto, alrededor de la línea que afirma el \emph{match}

	\item \emph{grep -L 'hello'}\\
	Imprime o lista, los nombres de todos los archivos que no contienen una línea con la palabra 'hello'

	\item \emph{grep -w -e '$\backslash(.\backslash)\backslash(.\backslash).\backslash2\backslash1$' file}\\
	Buscamos un palíndromo de 4 caracteres (ej. radar, civic)

\end{enumerate}

Una característica fundamental de \emph{grep} y \emph{UNIX},
es la capacidad de poder utilizar \emph{pipes} mediante comandos,
lo cual nos permite ``mezclar'' distintas operaciones.

Un ejemplo simple podría ser,
desplegar todos archivos PDF en un directorio determinado:
\begin{verbatim}
[cmaureir@nexus~]$ ls | grep -i "\.pdf$" 
\end{verbatim}

Extrapolando ésta funcionalidad,
nos daremos cuenta que la cantidad de aplicaciones \emph{útiles}
para \emph{grep} en \emph{UNIX} son casi infinitas.

\subsection{Otros sabores}
Además las Expresiones Regulares,
puede ser utilizadas con otros lenguajes u aplicaciones en UNIX,
como lo son:
\begin{itemize}
	\item AWK
	\item SED
	\item Bash (que utiliza grep, sed, awk, etc...)
	\item etc
\end{itemize}

\textbf{Ejemplos breves:}\\
\begin{itemize}
	\item \emph{AWK:}\\
	Busqueda:
	\begin{verbatim}
	awk '/expresion/{ print }' archivo
	\end{verbatim}
	\item \emph{SED:}\\
	Busqueda:
	\begin{verbatim}
	sed [opciones] "patrón/acción" archivo
	\end{verbatim}
	Sustituciones:
	\begin{verbatim}
	sed [opciones] "s/patrón1/patrón2/X" archivo
	\end{verbatim}
	(donde X puede ser g, reemplazar en todo; n, reemplazar n veces)
\end{itemize}

