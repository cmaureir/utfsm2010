\section{Background}
\subsection{Genetic Algorithm}
\frame
{
\frametitle{Background}
\framesubtitle{Genetic Algorithm}
\begin{enumerate}
	\item \blue{Generate} an initial population.
	\item \blue{Select} pair of individuals based on the fitness function.
	\item Produce \blue{next generation} from the selected pairs by applying pre-selected genetic operators.
	\item If the termination \blue{condition} (\# gens, converges sol) is satisfied stop, else go to step 2.
\end{enumerate}
}

\frame
{
\frametitle{Background}
\framesubtitle{Genetic Algorithm}
\begin{itemize}
	\item A solution is \red{represented} as a vector of real-parameter decision variable.
	\item Three main type of PGA:
	\begin{itemize}
		\item \blue{Master-slave}:
			single population, evaluation of fitness is distributed (several processors).
		\item \blue{Coarse-grained}:
			Population divided (sub-populations), occasional exchanges of individuals.
		\item \blue{Fine-grained}:
			Individuals are commonly mapped onto a 2D lattice, with one individual per node.
			selection and crossover restricted (neighborhood).
	\end{itemize}
\end{itemize}
}
