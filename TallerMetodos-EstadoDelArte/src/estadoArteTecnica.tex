%Descripción de otras experiencias similares a la que se estudiará.
%Esta sección deberá contener la descripción de experiencias con el mismo problema/técnica
%(las mejores) o para problemas similares que puedan ayudarle como refrencia para su propuesta.

Desde que De Castro et al. publicó su trabajo \emph{`` Learning and optimization using clonal selection principle''}~\cite{decastro},
distintos investigadores se han dedicado a poder utilizar dicha teoría para implementar algoritmos que buscan
optimizar una tarea determinada, de hecho, la mayoría de los trabajos se centran en por ejemplo la \emph{optimización
de funciones}, \emph{optimización multiobjetivo}, sin dejar de lado los problemas típicos de optimización como lo son
\emph{path planning}, \emph{graph coloring}, \emph{vehicle routing}, \emph{entrenamiento de sistemas}, \emph{clasificadores},
\emph{reconocimiento de patrones}, etc, pero la presente sección se centra netamente en problemas abordados que puedan
tener una cierta relación a nivel de modelamiento, objetivos o tratamiento con el problema principal del presente trabajo,
el \emph{Car Sequencing Problem}.\\


% Clonal Selection based Mobile Robot Path Planning

Respecto al área de la robótica, Xuanzi Hu~\cite{robotPlanning} a podido dar un enfoque a las problemáticas relacionadas
con el \emph{Mobile Robot Path Planning}, realizando una comparativa con los algoritmos genéticos, en los cuales queda demostrado
la eficiencia y superioridad de la selección clonal, ya que al ser dos metodologías generacionales, se basan en los mismo principios,
y son fácilmente comparables al momento de realizar un \emph{benchmark} de ambas técnicas con parámetros lo más similar posible.
Una característica especial de la presente implementación son sus operadores inmunes utilizados, en los que están el operador de mutación
que consiste en elegir aleatoriamente un nodo del camino y reemplazarlo por otro nodo que no está en el camino original; el operador
de inserción, que se utiliza para poder reparar los segmentos de un camino infactible, insertando un nodo entre el problema y por último
el operador de supresión, que se aplica a los caminos factibles e infactibles. 

Un punto importante en éste trabajo es que a nivel de programación algorítmica, la selección clonal presenta una superioridad también,
al poder ser un algoritmo mucho más sencillo de  implementar que un algoritmo genético.\\
%%


% Scheduling Aircraft Landing Based on Clonal Selection Algorithm and Receding Horizon Control
Por otro lado Xiaolan Jia et al.~\cite{aircraft} utiliza la selección clonal hibridamente en conjunto con un modelo
de control predictivo llamado \emph{Receding Horizon Control (RHC)} para abordar el conocido problema de \emph{Scheduling Aircraft Landing},
en el cual la selección clonal forma parte del algoritmo RHC, ocupándose de realizar el scheduling propiamente tal.

Detalladamente las dos aproximaciones que plantea el presente trabajo de X. Jia, señala que una selección clonal con restricciones
basado en ``grados de infactibilidad'' (IFD) programa los aviones en el \emph{receding horizon} actual, en cambio
el RHC repite el proceso de optimización usando ``propagación de segmentos excelentes de genes'' (EGSS) hasta que todos los aviones
han aterrizado.

IFD maneja las restricciones de una buena manera y guía el proceso de optimización de manera efectiva,
por otro lado EGSS mejora la velocidad de convergencia del algoritmo, demostrando empíricamente que la aproximación con RHC
soluciona el problema de una manera mas efectiva y rápida.\\
%%


% Clonal Selection Algorithm for Vehicle Routing

Otro problema de optimización conocido como \emph{Vehicle Routing Problem (VRP)} ha sido atacado utilizando la selección clonal
también por varios autores, entre ellos destaca Jacek Dabrowski~\cite{vrp} pues soluciona una pequeña variación del
problema en conjunto llamado \emph{Capacitated Vehicle Routing Problem (CVRP)} en el cual una flota fija de vehículos de repartición
con una capacidad uniforme debe atender la demanda de un cliente determinado para un solo producto de un sólo almacén, cumpliendo
obviamente el mínimo costo de transito. Por lo que la única variación entre CVRP y VRP es que el primero añade la restricción
de que todos los vehículos deben tener una capacidad uniforme de un sólo producto.

Dabrowski señala que la selección clonal es una buena herramienta para la búsqueda de caminos múltiples,
ya que cuando entregamos una instancia del programa se puede un conjunto de soluciones como un conjunto de anticuerpo,
rankeando las soluciones mediante ``operadores de comparación'' y pero que existe un control sobre el ``operador de mutación''
entre los cuales están; EXC, que elige esquinas del camino y las trata de unir; SUM, que intenta concatenar dos rutas sin violar
restricciones; NEW, que construye nuevas ruta utilizando la heurística \emph{greedy} a partir de un vértice aleatorio.

Finalmente el único problema es que el autor señala que sería correcto realizar un benchmark con otras técnicas,
para poder comprobar que tan efectiva es la técnica.\\
%%


% Stretching Technique-based Clonal Selection Algorithm for Flexible Job-shop Scheduling
Como ya se vio anteriormente, en muchas investigaciones se utiliza la mezcla entre la selección clona y otra técnica,
y es el mismo casi que propone Lu Hong~\cite{jobshop} al utilizar una técnica de \emph{Stretching} al resolver
una variación del \emph{Job-shop scheduling problem (JSP)} llamado \emph{Flexible Job-shop Scheduling Problem (FJSP)},
la cual solo posee una mayor disponibilidad de máquinas para realizar las operaciones, es decir, sigue la dinámica
de encontrar una ubicación de cada operación y definir las secuencias de éstas en cada máquina, tratando de utilizar el mínimo
tiempo posible.

La técnica de \emph{stretching} fue propuesta por M. N. Vrahatis (referencia),
%M. Vrahatis, G. Androulakis and M. Manoussakis, “A new unconstrained optimization method for Imprecise function andgradient values,”
%Journal of Mathematical Analysis and Applications, 1996, pp. 586–607.
y su idea principal es realizar dos etapas de transformación sobre la forma de la función objetivo,
basada en la información de los mínimos locales, es decir, si hay un mínimo se busca realizando un algoritmo
de optimización convencional, y cuando se encuentra, la función objetivo se ``estira'' de acuerdo a unas expresiones determinadas.

Utilizando la función de \emph{stretching} no se cambian los objetivos buscando, pero provee una forma de escape
de los óptimos locales mejorando la convergencia global.

Generalmente luego de los test, STCSA muestra ser mejor que una selección clonal normal,
siendo una excelente aproximación para resolver problemas a larga escala cuando otros algoritmos fallan dando buenas soluciones.\\
%%


% Immune Clonal Selection Algorithm for Hybrid Flow-shop Scheduling Problem
Feng Liu et al.~\cite{flowshop} para poder reducir la complejidad computacional del
\emph{Hybrid Flow-shop Scheduling Problem} utiliza selección clonal.

Éste problema es una aplicación importante en las empresas manufactureras, ya que es un problema NP-completo,
y actualmente las tecnologías de optimización y algunas heurísticas como \emph{branch and bound}, \emph{genetic algorithm}
han sido utilizadas pero sin tanto éxito, ya que tienen distintos inconvenientes al momento de trabajar con problemas
de alto tamaño.

Liu realiza ésta elección, pues la selección clonal propone mecanismos especiales como la habilidad de mantener la diversidad
de los anticuerpos, mecanismos de auto-adaptación y funciones de memoria. Además para mejorar la exploración y explotación
se agrupan estrategias y operadores de multi-mutación (mutación clonal, cruzamiento clonal y selección clonal).\\
%%


% Computer Experiments with a Parallel Clonal Selection Algorithm for the Graph Coloring Problem
Finalmente, existen trabajos dignos de destacar, como la aproximación que entrega Jacek Dabrowski~\cite{graph}
al realizar experimentos con selección clonal pero utilizando conceptos de paralelismo para un problema básico como lo es
el \emph{Graph Coloring Problem}, comparandolo contra un algoritmo \emph{Tabu search} en paralelo.

Los anticuerpos son inicializados utilizando una asignación aleatorias de colores o utilizando una heurística \emph{greedy},
por otro lado, la selección clonal se enfoca a minimizar la función objetivo, siendo ésta el numero de conflictos de colores.

El mecanismo de hiper-mutación cambia la asignación de colores a los vértices del gráfico.

Para mejorar el desempeño de la versión paralela que se ha creado se utilizada el modelo de ``isla'' (asincrónico),
donde cada procesador trabaja con su propio conjunto de anticuerpos.

Lo llamativo surge al existir mecanismos de ``migración'' que permite un intercambio de conocimiento entre los 
distintos procesos, los cuales son elegidos mediante una ``selección de torneo''.

Finalmente la selección clonal se sobrepone a \emph{Tabu search} en todas las instancias,
llegando a la conclusión que aunque no posea un operador de cruzamiento, la selección clonal
es una forma fácil de implementar un algoritmo de optimización.
%%

