%Debe contener: técnicas usadas en su resolución (en general las que han sido mas
%efectivas a la fecha), mejores resultados, estrategias mas usadas en su resolución,
%representacion, movimientos, penalización, etc..


El \emph{Car Sequencing Problem}~\cite{parello} es presentado por primera vez en la \emph{Journal of Automated Reasoning},
del año 1986 en \emph{EEUU}, por \emph{B.D Parello}, \emph{W.C. Kabat} y \emph{L. Wos}, siendo una variación
del problema ya conocido en el área de la Inteligencia Artificial, como lo es el \emph{Job-Shop scheduling problem},
pero usando otro tipo de razonamiento, el \emph{Razonamiento automatiza}, de ésta manera logran enfocar su trabajo
en el problema que ocurre en las líneas de producción de automóviles.

Desde la publicación del presente problema hasta los últimos años, han aparecido en distintas conferencias a lo largo
del mundo, diferentes formas de atacar el problema, que nos llevan desde  ``programación lineal'', ``Evolutionary Approaches''
y ``Local search'' con sus variantes.

Las técnicas que han sido utilizadas hasta el momento, no han sido desarrolladas sólo por matemáticos, sino que por una amplia
gama de investigadores, que van desde científicos del área de la producción, física, biología y gestión.

No obstante, gracias a los avances que ha tenido el campo de la Inteligencia Artificial, han aparecido nuevas metodologías,
que se relacionan con otras áreas como lo son la computación evolutiva, genética,  redes neuronales, neurofisiología;
los cuales han realizado contribuciones importantes a los problemas que tienen que ver con la ``planificación'' (scheduling).
Lo anteriormente señalado afirma el hecho que los problemas relacionados con la ``planificación'' abarcan muchas áreas en el mundo,
y que los trabajos en dicha área no tienen que ver sólo con científicos computacionales.

A continuación se describen algunos de los métodos más importantes que han sido propuestos desde el año
en que el \emph{Car Sequencing Problem} fue dado a conocer en el mundo de la investigación científica.


\subsection{Acercamientos heurísticos}
Existen distintos acercamientos que se han propuesto, que tienen como objetivo principal
encontrar una solución óptima lo más rápido posible.
Los acercamientos descritos a continuación son \emph{búsqueda local (local search)},
\emph{greedy approach}, \emph{Ant colony Optimization} y por último, \emph{Algoritmos genéticos}

\subsubsection{Búsqueda Local (Local Search)}
La \emph{Búsqueda Local} puede ser utilizada en distintos problemas que se formulan
de una forma determinada para poder encontrar una solución que maximice un criterio alrededor de un número
de soluciones candidatas

Lo que hace la \emph{Búsqueda Local}, es moverse en su vecindario solución por solución en el espacio previamente establecido
de las soluciones candidatas, también conocido como espacio de búsqueda, hasta que una solución considerada \emph{óptima}
es encontrada ó se termina el plazo de tiempo para encontrar la solución, por otra parte, 
el vecindario va a estar definido netamente con respecto al modelamiento del problema.

Actualmente existen variadas aproximaciones propuestas sobre \emph{Búsqueda Local},
para resolver el presente problema.
La diferencia aquí, es que la mayoría de las propuestas son acercamientos más
genéricos, es decir, se han utilizado varios problemas para aplicar la propuesta
planteada y entre ellos se encuentra el problema en cuestión, como por ejemplo~\cite{DTWZ94}
que nos plantea una mejora en el sentido de una \emph{Estrategia de Búsqueda}, considerando
el conflicto mínimo con respecto a la heurística \emph{hill-climbing}~\footnote{
Técnica de optimización matemática, puede ser usada para resolver problemas con muchas soluciones.
Consiste en escoger una solución aleatoria (potencialmente no óptima) y comienza a
realizar pequeños cambios a dicha solución en cada iteración, en cada momento se va mejorando un poco más.
}
y ha propuesto un escape con respecto a los mínimos locales incrementando el peso de las
restricciones que no se cumplen.

Por otra parte existen otras propuestas que se han enfocado netamente en resolver éste problema,
como es el caso de ~\cite{PG02}, que considera los algoritmos de \emph{Búsqueda local}
que se basan en una representación de una permutación y la evaluación de las restricciones que no se
cumplen utilizando funciones de penalización.

De la misma forma tenemos el caso de lo que plantea Gottlieb et al. en \cite{GPS03}, quien se refiere
a una parte en especial de nuestro problema, la construcción de la secuencia inicial, por lo tanto
sabemos que en la mayoría de los casos, la secuencia inicial de vehículos, de donde parte la
\emph{Búsqueda Local}, es una permutación aleatoria de un conjunto de vehículos para comenzar
a ensamblarlos, sin embargo la propuesta nos señala el construir ésta secuencia inicial en una forma
\emph{Voraz (Greedy)}~\footnote{
se refiere a a realizar una elección del siguiente elemento considerando una función heurística de optimización,
con la esperanza de llegar a una solución general óptima.
}
, lo que queda claramente demostrado en dicha publicación,
debido a las grandes mejoras en el proceso de solución.

\subsubsection{Acercamiento Voraz (Greedy approaches)}
Teniendo en consideración el \emph{Car Sequencing Problem}, podemos construir una secuencia
en una forma \emph{Voraz (Greedy)}, eligiendo el siguiente vehículo de la secuencia
considerando una función heurística de optimización.
El primero acercamiento utilizando \emph{Greedy} fue propuesto por
Hindi and Ploszaski en 1994~\cite{HP94}. Volviendo al procedimiento anteriormente descrito, la idea
central que nos plantea Gottlieb et al. en \cite{GPS03}. Claramente, ahora la elección que podemos
realizar en cada paso de nuestro algoritmo, debe ser la más óptima, en otras palabras,
elegir el vehículo que posea el menor número de restricciones no satisfechas.

Matemáticamente hablando, podríamos decir lo siguiente:
Dado una secuencia parcial $\pi$, el siguiente vehículo es seleccionado entre los vehículos $c_j$
que minimice la siguiente función,
$$newViolations(\pi, c_j) = \sum_{o_{i}\in O}r(c_j,o_i)\cdot violation(lastCars(\pi\cdot <c_j>,q(o_i)),o_i)$$
donde $o_j$ es una opción del vehículo $c_j$ y donde $lastCars(\pi',k)$ es la secuencia compuesta de los
últimos $k$ vehículos de $\pi'$, si $|\pi'| \geq k$, en otro caso $lastCars(\pi' , k) = \pi')$.

El problema acá, es que por lo general, el conjunto de vehículos seleccionados que minimizan la función
$newViolations$, contiene más de un auto, por lo tanto acá viene otra decisión importante. ¿Cuál de esos
autos elegimos?, por lo que Gottlieb et al.~\cite{GPS03} nos señala que es necesario utilizar otra técnica heurística para 
romper dicha incertidumbre, de la misma forma propone 6 heurísticas distintas. La heurística que presente
un mejor desempeño es llamada \emph{``dynamic sum of utilization rates''}, la cual consta en que en cada paso
añadimos el vehículo que maximiza la suma de las tasas de uso de opciones necesarias, estas tasas de uso se actualizan
dinámicamente cada vez que añadimos un nuevo auto al final de la secuencia.

\subsubsection{Ant Colony Optimization (ACO)}
El algoritmo \emph{Ant Colony Optimization (ACO)} \cite{DS05}, es un algoritmo que
busca una forma de modelar cierto problema de tal forma que busquemos siempre el camino
con mínimo costo en un grafo determinado, y por supuesto, utilizar \emph{hormigas artificiales}
para buscar los caminos que más nos pueden favorecer.

El primer algoritmo de \emph{ACO} fue propuesto por Christine Solnon~\cite{Sol00},
y se trata de resolver un problema de satisfacción de restricciones de permutación~\footnote{
El objetivo principal es poder encontrar la permutación de $n$ valores conocidos,
para asignarlos a $n$ variables, bajo ciertas restricciones.
},
utilizando el concepto de \emph{hormigas artificiales}. Principalmente está diseñado
para resolver un problema general de combinatoria.

En enfoque que entrega el trabajo previamente señalado~\cite{Sol00} con respecto
al \emph{Car Sequencing Problem} es el siguiente:
Recordando nuestro problema principal, sabemos que consiste
en realizar la óptima planificación de los vehículos en una línea de ensamblaje,
donde tenemos $n$ vehículos para producir, que están agrupados en $k$ tipos
de autos~\footnote{
Cada conjunto de autos de un mismo tipo tienen las mismas opciones
}, y sabemos también que cada opción $i$ está asociada con una restricción de capacidad
representada por una tasa $p_i/q_i$, que especifica que para cualquier secuencia
de $q_i$ autos consecutivos en le línea de ensamblaje, tenemos al menos $p_i$ vehículos
que requieren dicha opción.
Finalmente el objetivo será encontrar una permutación de $n$ vehículos que satisfacen
todas las restricciones de capacidad, tomando en consideración que nuestra feromona
se pondrá sobre las parejas de los vehículos consecutivos con el fin de aprender
las sub-secuencias de vehículos más prometedoras (óptimas).

No conforme con el desempeño mostrado en el trabajo anterior, Gottlieb et al.~\cite{GPS03}
proponen un nuevo algoritmo que introduce tres nuevas mejoras:
\begin{itemize}
	\item Se utiliza una estrategia elitista, por lo que la feromona es usada para romper el lazo entre los mejores vehículos solamente.
	\item Se integran mejoras del trabajo ~\cite{Stu}, para favorecer la exploración.
	\item Se usan funciones heurísticas de tipo \emph{voraz (greedy)} para guiar las hormigas.
\end{itemize}
Con las mejoras anteriormente planteadas se logra obtener un resultado mucho más óptimo computacionalmente.

\subsubsection{Algoritmos Genéticos (Genetic Algorithms)}
Los \emph{Algoritmos Genéticos} son una técnica computacional que tienen como objetivo principal,
el encontrar una solución exacta o aproximada en problemas de optimización y búsqueda.

En el trabajo de Warwick y Tsang~\cite{WT95} se propone un algoritmo genético especialmente
para resolver el \emph{Car Sequencing Problem} que se diferencia de los algoritmos genéticos
tradicionales pues ingresa ciertas características especiales como el \emph{elitismo}~\footnote{
proceso de seleccionar mejores individuos, seleccionándolos con el fin de obtener una mejor solución
}, plantillas adaptativas de \emph{crossover}~\footnote{
operador genético utilizado para variar la programación de uno o varios cromosomas, de una generación  a otra.
}, \emph{hill-climbing}, entre otras.

El acercamiento que nos ofrece éste mecanismo tiene su origen en una especie de inspiración
de la evolución natural y trata de introducción computacionalmente los conceptos de \emph{cross over},
\emph{mutación}, y tomar las secuencias de nuestros problemas, como una \emph{población}.

Por lo tanto, el procedimiento que realiza es que por cada \emph{generación}, las secuencias
que son seleccionadas se combinan gracias al operador \emph{cross over}, y se obtiene lo
que se denomina \emph{descendencia} que puede que no satisfaga la restricción global,
pero se va reparando en cada iteración y luego se le aplica \emph{hill-climbing} mediante
una función de intercambio.

Lamentablemente el procedimiento anteriormente descrito es bueno para resolver sólo,
versiones no tan complejas del \emph{Car Sequencing Problem}.
Debido a lo anterior Cheng et al. \cite{CLP+99} propone un algoritmo para resolver un
problema práctico, es decir, más realista, con un grado mayor de complejidad.
Cheng et al. \cite{CLP+99} proponen un operador llamado \emph{cross-switching},
que genera una descendencia intercambiando los vehículos que aparecen de forma aleatoria
en la primera secuencia del problema.

\subsubsection{Movimientos}

A continuación se explican brevemente algunos movimientos conocidos que definen el vecindario de 
una solución, y que de la misma forma, es válido para cualquier acercamiento anteriormente
mencionado.
\begin{enumerate}
	\item \emph{shift}: Puede realizarse con repecto a la izquierda o derecha, y avanzando una cantidad determinada de espacios,
		es decir, tomamos nuestra secuencia numérica, y la movemos para alguno de los dos lados. El hecho de que sea a la izquierda
		o derecha y la cantidad de espacios que se mueve, se puede elegir de manera aleatoria.

		\emph{e.g:} 1 0 1 0 0 1, aplicamos shift-right 2 veces y nos queda 0 1 1 0 1 0.
	\item \emph{swap}: Este movimiento es muy utilizado, pues ha demostrado tener alta eficiencia al momento de buscar
		nuevas soluciones, sin romper restricciones. Consiste en tomar dos elementos de nuestra secuencia, que pueden ser escogido
		de forma aleatoria, e intercambiarlos
		de lugar.

		\emph{e.g:} 1 0 1 0 0 1, aplicamos swap(1,5) y nos queda 1 1 1 0 0 0.
	\item \emph{bit-flip}: Éste movimiento consiste en cambiar el valor de un \emph{bit} de nuestra secuencia numérica.
			Sólo es válida para representaciones binarias (0 o 1). Para escoger que elemento cambiamos, lo podemos hacer de una forma
			aleatoria.
		
		\emph{e.g:} 1 0 1 0 0 1, aplicamios bit-flip(3) y nos queda 1 0 1 1 0 1.
	\item \emph{Cruzamiento en un punto (Single Point Crossover)}: La presente técnica consiste en tener dos individuos,
		elegir un punto al azar dentro de la secuencia y cruzarlos, intercambiando su material. También existen otras variaciones
		de éste movimiento, donde se elige más de un punto para intercambiar el material de los padres.

		\emph{e.g:} 1 0 1 1 y 1 1 0 1, elegimos el punto 2 y nos quedan dos hijos de la forma, 1 0 0 1 y 1 1 1 1.
				
	\item \emph{Técnica incompleta}: Es posible utilizar una misma técnica incompleta como \emph{movimiento} para poder generar
		el vecindario determinado, lo cual en la mayoría de las veces, produce secuencias mucho mejor, pero toma mucho más tiempo
		que los movimientos descritos anteriormente.
\end{enumerate}

\subsection{Acercamientos completos}
\subsubsection{Integer Lineal Programming}
En el trabajo de Grevel et al.~\cite{4} se propone un acercamiento utilizando
\emph{Programación Lineal Entera}, ILP por sus siglas en inglés,  para una variante
del problema famoso de \emph{Car sequencing problem} pero del concurso ROADEF,
donde ahí se establece condiciones extras relacionadas a la pintura de los vehículos,
que éste modelamiento no posee.

Grevel et al.~\cite{4} también nos da a entender que se puede resolver el problema
de una manera común, con ésto nos referimos a usando un simple \emph{benchmark}
con aproximadamente unos $200$ vehículos y $5$ componentes, probando así
lo óptimo que es el método, con un tiempo de respuesta razonable.
La idea principal que posee éste mecanismo o modelamiento, es poder aplicarla a un
grupo de vehículos con la misma configuración, con respecto a las opciones,
evitando así simetrías entre ellos.

Existe otra aproximación relacionada con ILP, presentada por Prandtstetter et al.~\cite{12}
que es análoga a la formulación realizada por Gravel et al., clasificando los vehículos
que tienen la misma configuración en grupos, gracias a ésto el tamaño de alguna instancia
del problema que sea propuesto, está restringido a lo más $300$ vehículos con aproximadamente
unos $8$ componentes.

\subsubsection{Ramificación y Poda (Branch-and-bound method)}

El presente método es una variación del algoritmo \emph{Backtracking}~\footnote{
la idea del algoritmo se asemeja a un recorrido en profundidad dentro de un grafo dirigido.
}
, pero con bastantes mejoras.
Por lo general, la idea es interpretar el problema como un árbol de soluciones,
en el cual las ramas van a ser caminos para una posible solución, pero posterior a la solución
actual (rama actual).
La idea central del presente método es que se encarga de encontrar las ramas que no poseen
una solución óptima en comparación al resto y las \emph{poda} para no utilizar el tiempo
de búsqueda en alguna rama que no sirva.

En el trabajo de Drexl et al.~\cite{DKM06}, se propone utilizar la misma técnica anteriormente
descrita.

Primeramente es un poco complicado por imaginarse el problema atacado del punto de vista
del \emph{Branch-and-bound}, debido a que se toma como un problema de grafos, pero lo que
Drexl et al.~\cite{DKM06} establece es que el método está basado en un esquema de ramificación
\emph{(branching)} original, que utiliza el concepto de \emph{Car sequencing state}, 
refiriéndose así a que cada \emph{Car sequencing state} está asociado a cada nodo del árbol
que podemos construir para resolver el presente problema.

Para comenzar el \emph{branching}, es necesario construir una secuencia $\delta$
asignando cada periodo $t\in {1,...,V}$ a una copia de una variante $v$, realizando ésto
se puede manipular la secuencia $\delta$ de la forma:
$$\delta : \{1,...,T\}\rightarrow\{1,...,V\}\cup\{0\}$$~\footnote{
Se agrega el $0$ para evitar las secuencias parciales, asumiendo que ningún periodo no asignado
está siendo proyectada a una variante vacía. Ésto simplifica las definiciones, sin restringirlas.
}, con
\begin{itemize}
	\item $T:$ volumen total de producción (periodos, ciclos), i.e $T=\sum\limits_{v=1}^{V}D(v)$, índice $t$
	\item $V:$ número de variantes, índice $v$
\end{itemize}

La secuencia $\delta$ induce una matriz $O\times T$ que llamaremos $M$,
la cual está representando los periodos de la planificación de T,
y cada file de $M$ incide con una opción $j$
, con
\begin{itemize}
	\item $O:$ número de opciones, índice $j$.
\end{itemize}

La interpretación de la matriz de la matriz $M$ se define a continuación:\\

$m_{j,t} = \left\{ \begin{array}{l}
\ \ 1,\ \text{si la opción}\ j\ \text{es planificada en el periodo}\ t,\ \text{respetando la restricción}\ H_{j}:H_{j}\\
\ \ 0,\ \text{si la opción}\ j\ \text{puede ser planificada en el periodo}\ t\\
-1,\ \text{si la opción}\ j\ \text{no es planificada en el periodo}\ t\text{, si es planificada puede violar}\\
\ \ \ \ \ \text{la restricción}\ H_{j}:H_{j}\\
\end{array} \right.$

\subsubsection{Constraint Programming}

El método \emph{Constraint Programming} es una herramienta genérica para poder resolver problemas
del tipo \emph{Constraint  Satisfaction Problem (CSP)}~\footnote{No confundir con Car Sequencing Problem},
por lo tanto cualquier problema de \emph{scheduling} puede ser resuelto mediante ésta forma,
ya que conjunto de restricciones que debemos satisfacer son una relación entre varias variables desconocidas,
y otras variables conocidas~\cite{Tsa93}.

%
Por lo general para obtener una solución óptima, se suelen combinar técnicas heurísticas,
y otras para deducir restricciones derivadas, como la clausura.
Como es el caso de Alfonso y Barber~\cite{esp1}, el cual utiliza un método primero de \emph{constraint-posting},
que significa ir adicionando sucesivamente las restricciones y en cada paso realizar el proceso de clausura,
en otras palabras incorporan el proceso de clausura que determina el conjunto de las restricciones que pueden
impedir  en las que ya se han asertado correctamente.

Dejando de lado el trabajo anteriormente mencionado, el método de \emph{Constraing Programming} permite la definición
y aplicación de nuevas técnicas de heurística, permitiendo así limitar el número de disyunciones
del conjunto de las restricciones y además permiten dirigir mejor el proceso de búsqueda.

\subsection{Acercamientos híbridos}

Cuando hablamos de acercamiento híbrido, nos referimos a poder combinar las ventajas de varios
métodos de optimización en un sólo algoritmo.
Anteriormente vimos que en muchos casos, existe inclusive una \emph{recomendación} para buscar
una heurística determinada complementando a un método y encontrar una solución óptima.


Tomando las palabras del estudio realizado por E.G Talbi~\cite{talbi} los algoritmos híbridos
se han aplicado satisfactoriamente a variados problemas de optimización combinatorial, como el problema
del \emph{vendedor-viajero} o simplemente problemas del mundo real.


Siguiendo las palabras de Blum, Roli y Alba en ~\cite{blum} podemos aseverar que ésta clase de acercamientos,
combinando metaheurísticas basadas en  poblaciones con alguna solución metaheurística simple, podemos llegar
a la conclusión de que es el mejor camino para poder mejorar la calidad de las soluciones que encontramos
en cualquier problema de optimización.

\subsubsection{Hybrid Integrative Evolutionary Approaches}
En el trabajo de Zinflou et al. \cite{HYB1} se presentan tres acercamientos para poder
resolver el presente problema. Los acercamientos están basados en esencia en algoritmos evolutivos
que incorporan operadores de heurística como los son los \emph{crossover operators} usando
un modelo simple, es decir basado en  Programación Lineal Entera.

La diferencia que existe entre dos de las aplicaciones de \emph{crossover} combinadas nos muestra que es posible
obtener valores competitivos, dignos de ofrecer como alternativa a los métodos anteriormente nombrado, debido
a la calidad de la solución.

Finalmente podemos darnos cuenta que la experiencia numérica de la formulación de éste método, muestra una diferencia
notoria en el tiempo de ejecución, algoritmicamente hablando.
Las diferencias indican que puede ser posible de obtener mejoras en la implementación de métodos híbridos para 
poder obtener una comparación más competitiva. Dichas mejoras significan poder involucrar al desarrollo de nuestro
algoritmos un enfoque de \emph{Grid computing} u otras técnicas de \emph{high performance computing technologies}.
