%En que nivel se ha investigado la técnica aplicada al problema
%     (existe o no investigación, cuantas, cuales modelos, componentes
%     de la técnica, como es su desempeño, etc.).
%     En caso de no existir investigación o si a usted le parece que es
%     un aporte, experiencias con problemas similares o que podría ser útil
%     para su trabajo.

%existe o no?
%experiencias similares
\frame
{
\frametitle{Estado del Arte}
\framesubtitle{Existencia del acercamiento}
\begin{itemize}
	\item Actualmente en las grandes fuentes de publicaciones, como lo son:
	\begin{itemize}
		\item ACM.
		\item Springerlink.
		\item IEEE.
		\item Google Scholar.
	\end{itemize}
	\item \textbf{NO} existe un acercamiento utilizando \emph{Clonal Selection}
		para la resolución del \emph{Car Sequencing Problem}.
	\item Estado del arte del \emph{ROADEF}.
\end{itemize}
}

\frame
{
\frametitle{Estado del Arte}
\framesubtitle{Experiencias Similares}
\begin{block}{Mobile Robot Path Planning}
\begin{itemize}
	\item Operadores Inmunes:
	 \begin{itemize}
	 	\item \emph{Operador de Mutación:}
			consiste en elegir aleatoreamente un nodo del camino y \textbf{reemplazarlo} por otro nodo que no esté en el camino original. (intercambio de opciones válidas)
		\item \emph{Operador de Inserción:}
			se utiliza para poder \textbf{reparar} los segmentos de un camino infactible, insertando un nodo entre el problema. (cambiar el elemento inválido por otro)
		\item \emph{Operador de Supresión:}
			se aplica a los caminos factibles e infactibles, para \textbf{disminuir costos}. (formas de disminuir el fitness por orden)
	 \end{itemize}
\end{itemize}
\end{block}
}

\frame
{
\frametitle{Estado del Arte}
\framesubtitle{Experiencias Similares}
\begin{block}{Scheduling Aircraft Landing}
\begin{itemize}
	\item Basar la selección clonal en:
	\begin{itemize}
		\item \emph{Infeasibility Degree (IFD):}
			maneja las restricciones de una buena manera y guía el proceso de optimización de manera efectiva.
		\item \emph{Excellent Gene Segment Spread (EGSS):}
			mejora la velocidad de convergencia del algoritmo.
	\end{itemize}
\end{itemize}
\end{block}
}

\frame
{
\frametitle{Estado del Arte}
\framesubtitle{Experiencias Similares}
\begin{block}{Vehicle Routing Problem}
\begin{itemize}
	\item Operadores de Mutación:
	\begin{itemize}
		\item \emph{EXC:}
			elige \textbf{esquinas} del camino y las trata de \textbf{unir}. (no sirve mucho)
		\item \emph{SUM:}
			que intenta \textbf{concatenar} dos rutas sin violar restricciones. (tipo de cruzamiento en un punto)
		\item \emph{NEW:}
			que \textbf{construye} nuevas ruta utilizando la heurística greedy a partir de un vértice aleatorio. (mejorar individuos particulares)
	\end{itemize}
\end{itemize}
\end{block}
}

\frame
{
\frametitle{Estado del Arte}
\framesubtitle{Experiencias Similares}
\begin{block}{Job-shop scheduling problem}
\begin{itemize}
	\item Variar la función objetivo en caso de encontrar mínimos locales, para escapar de los \emph{óptimos locales}.
	\begin{itemize}
		\item Primero, aumenta la función objetivo para hacer desaparecer los mínimos locales. (sumándole un factor con constantes positivas)
		\item Segundo, estira el vecindario de la variable auxiliar. (del paso anterior)
	\end{itemize}
\end{itemize}
\end{block}
}

\frame
{
\frametitle{Estado del Arte}
\framesubtitle{Experiencias Similares}
\begin{block}{Parallel Graph Coloring Problem}
\begin{itemize}
	\item Inicializar anticuerpos de forma aleatoria o utilizando \emph{greedy}.
	\item Utilizar conceptos de paralelismo.
	\item Concepto de \emph{migración}.
\end{itemize}
\end{block}
}
