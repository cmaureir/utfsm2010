% Como planea enfrentar el problema, que modelo planea implementar,
%  cosas del estado del arte le interesa usar (movimientos, operadores,
%  etc) o que le dan ideas de como enfrentar el problema.

% enfrentamiento
% modelo que planeo implementar
% cosas del estado del arte importante (mov, opera, ...)

\frame
{
\frametitle{Trabajo Futuro}
\framesubtitle{Enfrentamiento}
\begin{itemize}
	\item Realizar un modelamiento matemático simple,
		con el fin de poder enfocar la atención en el desempeño
		más que en la representación más rebuscada.
\end{itemize}
}

\frame
{
\frametitle{Trabajo Futuro}
\framesubtitle{Modelo}
\begin{itemize}
    \item Parámetros
    \begin{itemize}
        \item $cN$: Número total de autos.
        \item $oN$: Número total de opciones disponibles.
        \item $tN$: Número total de tipos/clases de autos.
    \end{itemize}
\end{itemize}
}

\frame
{
\frametitle{Trabajo Futuro}
\framesubtitle{Modelo}
\begin{itemize}
   \item Variables
    \begin{itemize}
        \item $nMax_{ij}$: Número máximo de autos con la opción $i$ en una subsecuencia $j$.
        \item $n_{ij}$: Número de autos con la opción $i$ en una subsecuencia $j$.
        \item $sMax_{i}$: Tamaño de la subsecuencia $j$ donde deben haber $nMax_{ij}$ autos.
        \item $q_{k}$: Cantidad de autos del tipo/clase $k$.
        \item $types_{il}$: Booleano que indica si la opción $i$ está presente en el auto $l$.
    \end{itemize}
\end{itemize}
}

\frame
{
\frametitle{Trabajo Futuro}
\framesubtitle{Modelo}
\begin{itemize}
    \item Función Objetivo
    $$FO\ :\ Min\ \sum\limits_{i=1}^{oN} \sum\limits_{l=1}^{cN} \sum\limits_{j=0}^{sMax_{i}} types_{il}\cdot (n_{ij} - nMax_{ij}), \forall n_{ij} > nM_{ij}$$
    \item Restricciones Duras
    $$\sum\limits_{k=1}^{cN} q_{k} = cN$$
    \item Restricciones Blandas
    $$\sum\limits_{i=1}^{oN} \sum\limits_{l=1}^{cN} \sum\limits_{j=0}^{sMax_{i}} types_{il}\cdot n_{ij} \leq nMax_{ij}$$
\end{itemize}
}

\frame
{
\frametitle{Trabajo Futuro}
\framesubtitle{Ideas del estado del arte}
\begin{itemize}
	\item Selección mediante \emph{Roullete Wheel}.
	\item Infeasibility Degree (IFD) y Excellent Gene Segment Spread (EGSS).
	\item Realizar un cruzamiento, cuidado las restricciones duras.
	\item Utilizar un operador de mutación para la hiper-mutación como el NEW (construir con greedy una nueva solución a partir de una existente)
	\item Quizás Paralelismo! 
\end{itemize}
}

\frame
{
\vspace{1cm}
\begin{center}
	\Huge ¿Preguntas?
\end{center}
}
