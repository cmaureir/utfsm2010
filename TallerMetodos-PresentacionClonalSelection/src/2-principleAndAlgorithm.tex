\section{Clonal Selection Principle and Algorithm}
\frame
{
\frametitle{Clonal Selection Principle and Algorithm}
\framesubtitle{A. Clonal Selection Principle}
\begin{itemize}
	\item Theory proposed by Burnet in 1978.
	\item Main features
	\begin{enumerate}
		\item The new cells are copies of their parents (clone) subjected to a mutation mechanism with high rates (somatic hyper-mutation).
		\item Elimination of newly differentiated lymphocytes.
		\item Proliferation and differentiation on contact of mature cells with antigens.
		\item The persistence of forbidden clones, resistant to early elimination by self-antigens, as the basis of autoimmune diseases.
	\end{enumerate}
\end{itemize}
}

\frame
{
\frametitle{Clonal Selection Principle and Algorithm}
\framesubtitle{B. Clonal Selection Algorithm}
\begin{itemize}
	\item[1.] Generate a set (P) of candidate solutions, composed of memory cells (M) added to the remaining (Pr) population (P = Pr + M )
	\item[2.] Determine the $n$ best individuals (Pn) of the population (P), based on an affinity measure.
	\item[3.] Clone (reproduce) these $n$ best individuals of the population, giving rise to a temporary population of clones (C).
\end{itemize}
}
\frame
{
\frametitle{Clonal Selection Principle and Algorithm}
\framesubtitle{B. Clonal Selection Algorithm}
\begin{itemize}
	\item[4.] Submit the population of clones to a hyper-mutation scheme (proportional to the affinity of the antibody). A maturated antibody population is generated (C*).
	\item[5.] Re-select the improved individuals from C* to compose the memory set. (Some replaces from C* to P, because the improvement)
	\item[6.] Replace $d$ low affinity antibodies of the population, maintaining its diversity.
\end{itemize}
}
