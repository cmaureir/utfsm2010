\documentclass[letter, 10pt]{article}
\usepackage[utf8]{inputenc}
\usepackage[spanish]{babel}
\usepackage{amsfonts}
\usepackage{amsmath}
\usepackage[dvips]{graphicx}
\usepackage{url}
\usepackage[top=3cm,bottom=3cm,left=3.5cm,right=3.5cm,footskip=1.5cm,headheight=1.5cm,headsep=.5cm,textheight=3cm]{geometry}


\begin{document}
\bibliographystyle{plain}
\pagestyle{empty}

\title{Inteligencia Artificial \\ \begin{Large}Estado del Arte: ``Car Sequencing Problem''\end{Large}}
\author{Cristián D. Maureira Fredes.}
\date{\today}
\maketitle


%--------------------No borrar esta sección--------------------------------%
\section*{Evaluación}

\begin{tabular}{ll}
Resumen (5\%): & \underline{\hspace{2cm}} \\
Introducción (5\%):  & \underline{\hspace{2cm}} \\
Definición del Problema (10\%):  & \underline{\hspace{2cm}} \\
Estado del Arte (35\%):  & \underline{\hspace{2cm}} \\
Modelo Matemático (20\%): &  \underline{\hspace{2cm}}\\
Conclusiones (20\%): &  \underline{\hspace{2cm}}\\
Bibliografía (5\%): & \underline{\hspace{2cm}}\\
 &  \\
\textbf{Nota Final (100)}:   & \underline{\hspace{2cm}}
\end{tabular}
%---------------------------------------------------------------------------%

\begin{abstract}
%Resumen del informe en no más de 10 líneas.
El presente documento tiene como finalidad presentar el estudio y análisis de un problema clásico, a lo que la Inteligencia Artificial respecta,
el \emph{Car Sequencing Problem}~\cite{parello}, que consiste en la planificación de una fila de vehículos en una línea de ensamblaje, 
mientras se satisfacen todas las restricciones de capacidad de las plantas ensambladores y de los propios vehículos.

La importancia de atacar el presente problema radica en el notorio aumento de la actividad en las fábricas automotrices a lo largo del mundo,
dejando como consecuencia que incluso para el pasado año 2002, ya existían 1 automóvil por cada 10 personas en el mundo~\cite{worldmapper}.

Se plantean además las técnicas más conocidas y utilizadas para poder enfrentar el presente problema, entre las cuales destacan la utilización
de heurísticas y métodos híbridos, dentro de los que son los acercamientos exactos, híbridos y mixtos.

Se presenta un modelamiento simplista utilizando lo que es Programación Lineal, y un planteamiento más teórico para poder comprender a fondo
el problema estudiado.



%problem
%solucion propuesta
%resultados
%keywords

\end{abstract}

\section{Introducción}
\label{sec:introduccion}
Una expresión regular,
a menudo llamada también patrón,
es una expresión que describe un conjunto de cadenas sin enumerar sus elementos.

Además,
normalmente representan otro grupo de caracteres mayor,
de tal forma que podemos comparar el patrón con otro conjunto de caracteres para ver las coincidencias.

La mayoría de las formalizaciones proporcionan los siguientes constructores:
una expresión regular es una forma de representar a los lenguajes regulares
(finitos o infinitos) y se construye utilizando caracteres del alfabeto
sobre el cual se define el lenguaje.

Específicamente,
las expresiones regulares se construyen utilizando los operadores
unión, concatenación y clausura de Kleene.

Las expresiones regulares en UNIX (RE's) están recogidas en el estándar POSIX 1003.2.



\section{Definición del Problema}
\label{sec:definicion}
%Descripción del problema, su complejidad, en que consiste, cuales son sus objetivos,
%restricciones, variantes más conocidas. 

El \emph{Car Sequencing Problem} es un problema de satisfacción de restricciones (CSP), posee la
característica de ser NP-duro~\footnote{
NP-duro es el conjunto de los problemas de decisión que contiene los problemas H
tales que todo problema L en NP puede ser transformado polinomialmente en H.
Esta clase puede ser descrita como conteniendo los problemas de decisión que son
al menos tan difíciles como un problema de NP.}
, además corresponde a un tipo de variación del problema NP-completo \emph{Job-Shop Scheduling}.%,
%pero con un uso de razonamiento automatizado, es decir, con un enfoque dedicado a estudiar y comprender
%diferentes características del razonamiento, permitiendo así construir programas que le den la posibilidad
%a los computadores para razonar en forma autónoma.
 
Siguiendo la misma idea, es válido señalar que el \emph{Car Sequencing Problem} es un tipo de problema de planificación
de tareas en una línea de ensamblaje de autos, donde cada uno es perteneciente a un clase de automóvil, debido al conjunto
de opciones y accesorios que posee (airo acondicionado, centralizado eléctrico, etc), y cada una de las opciones o
accesorios se instala en una planta distinta, por lo que el objetivo principal es el poder encontrar el orden en la
secuencia de los vehículos, preocupándonos de no exceder la capacidad de cada planta de ensamblaje y también cumplir con la demanda.

Por lo tanto, si realizamos una definición más formal de nuestro problema, podríamos decir lo siguiente:
Teniendo una lista de vehículos dada, cada uno con sus respectivas opciones requeridas,
necesitamos establecer un orden en la línea de ensamblaje, con el fin de que cada subsecuencia de $q$ vehículos
tengamos a lo más $p$ que requieren de una determinada opción. Es importante tener en consideración que los
valores de $p$ y $q$ están asociados a cada opción de los vehículos.

Con respecto a la información que el problema otorga, podemos decir que contamos con:
\begin{itemize}
	\item Cantidad de vehículos de cada tipo o clase a producir (demanda)
	\item Lista de las opciones con la cual se constituye cada tipo o clase de vehículo, la cual puede utilizar una representación
		booleana para saber si cierto tipo de automóvil posee o no una determinada opción.
	\item Capacidad de las plantas que se preocupan de instalar la determinada opción.
\end{itemize}

Nuestro objetivo principal es:
\begin{itemize}
	\item Encontrar un orden en nuestra secuencia, que sirva para minimizar el costo por cada restricción insatisfecha.
\end{itemize}

Con respecto a las restricciones, tenemos que:
\begin{itemize}
	\item En cada subsecuencia de los $q$ vehículos, a lo más pueden haber $p$ que requieran de la opción determinada.
		Donde $p$ y $q$ son valores asociados a cada opción.
	\item La capacidad de cada planta de ensamblaje no puede ser excedida, es decir, cumplir con la demanda de cada automóvil
		sin abusar de una planta determinada.
	\item Por cada tipo de auto, el numero de autos de ese tipo debe ser secuenciado, es decir, todos los automóviles de cada clase
		deben estar presente en una secuencia determinada.
\end{itemize}





\section{Estado del Arte}
\label{sec:estado}
%En que nivel se ha investigado la técnica aplicada al problema
%     (existe o no investigación, cuantas, cuales modelos, componentes
%     de la técnica, como es su desempeño, etc.).
%     En caso de no existir investigación o si a usted le parece que es
%     un aporte, experiencias con problemas similares o que podría ser útil
%     para su trabajo.

%existe o no?
%experiencias similares
\frame
{
\frametitle{Estado del Arte}
\framesubtitle{Existencia del acercamiento}
\begin{itemize}
	\item Actualmente en las grandes fuentes de publicaciones, como lo son:
	\begin{itemize}
		\item ACM.
		\item Springerlink.
		\item IEEE.
		\item Google Scholar.
	\end{itemize}
	\item \textbf{NO} existe un acercamiento utilizando \emph{Clonal Selection}
		para la resolución del \emph{Car Sequencing Problem}.
	\item Estado del arte del \emph{ROADEF}.
\end{itemize}
}

\frame
{
\frametitle{Estado del Arte}
\framesubtitle{Experiencias Similares}
\begin{block}{Mobile Robot Path Planning}
\begin{itemize}
	\item Operadores Inmunes:
	 \begin{itemize}
	 	\item \emph{Operador de Mutación:}
			consiste en elegir aleatoreamente un nodo del camino y \textbf{reemplazarlo} por otro nodo que no esté en el camino original. (intercambio de opciones válidas)
		\item \emph{Operador de Inserción:}
			se utiliza para poder \textbf{reparar} los segmentos de un camino infactible, insertando un nodo entre el problema. (cambiar el elemento inválido por otro)
		\item \emph{Operador de Supresión:}
			se aplica a los caminos factibles e infactibles, para \textbf{disminuir costos}. (formas de disminuir el fitness por orden)
	 \end{itemize}
\end{itemize}
\end{block}
}

\frame
{
\frametitle{Estado del Arte}
\framesubtitle{Experiencias Similares}
\begin{block}{Scheduling Aircraft Landing}
\begin{itemize}
	\item Basar la selección clonal en:
	\begin{itemize}
		\item \emph{Infeasibility Degree (IFD):}
			maneja las restricciones de una buena manera y guía el proceso de optimización de manera efectiva.
		\item \emph{Excellent Gene Segment Spread (EGSS):}
			mejora la velocidad de convergencia del algoritmo.
	\end{itemize}
\end{itemize}
\end{block}
}

\frame
{
\frametitle{Estado del Arte}
\framesubtitle{Experiencias Similares}
\begin{block}{Vehicle Routing Problem}
\begin{itemize}
	\item Operadores de Mutación:
	\begin{itemize}
		\item \emph{EXC:}
			elige \textbf{esquinas} del camino y las trata de \textbf{unir}. (no sirve mucho)
		\item \emph{SUM:}
			que intenta \textbf{concatenar} dos rutas sin violar restricciones. (tipo de cruzamiento en un punto)
		\item \emph{NEW:}
			que \textbf{construye} nuevas ruta utilizando la heurística greedy a partir de un vértice aleatorio. (mejorar individuos particulares)
	\end{itemize}
\end{itemize}
\end{block}
}

\frame
{
\frametitle{Estado del Arte}
\framesubtitle{Experiencias Similares}
\begin{block}{Job-shop scheduling problem}
\begin{itemize}
	\item Variar la función objetivo en caso de encontrar mínimos locales, para escapar de los \emph{óptimos locales}.
	\begin{itemize}
		\item Primero, aumenta la función objetivo para hacer desaparecer los mínimos locales. (sumándole un factor con constantes positivas)
		\item Segundo, estira el vecindario de la variable auxiliar. (del paso anterior)
	\end{itemize}
\end{itemize}
\end{block}
}

\frame
{
\frametitle{Estado del Arte}
\framesubtitle{Experiencias Similares}
\begin{block}{Parallel Graph Coloring Problem}
\begin{itemize}
	\item Inicializar anticuerpos de forma aleatoria o utilizando \emph{greedy}.
	\item Utilizar conceptos de paralelismo.
	\item Concepto de \emph{migración}.
\end{itemize}
\end{block}
}


\section{Modelo Matemático}
\label{sec:modelo}
%Uno o más modelos matemáticos para el problema, idealmente indicando el espacio de búsqueda para cada uno.
\subsection{Primer modelo}
El presente modelo, pese a utilizar notaciones matemáticas, trata de dar un enfoque más teórico,
para una mejor comprensión del problema.

Consideremos el \emph{Car Sequencing Problem} como una tupla $(C,O,p,q,r)$, tal que:
\begin{itemize}
	\item $C=\{c_1,\ldots,c_n\}$ es el conjunto de vehículos a producir.
	\item $O=\{o_1,\ldots,o_m\}$ es el conjunto de distintas opciones.
	\item $p:O\rightarrow\mathbb{N}$ and $q:O\rightarrow\mathbb{N}$ definen la capacidad de las restricciones,
		es decir, para cualquier opción $o_i \in O$, cada subsecuencia de $q_i$ vehículos consecutivos en la línea,
		deben no contener más que $p_i$ vehículos que requieran la opción $o_i$
	\item $r:C\times O \rightarrow \{0,1\}$ define los requerimientos de las opciones, es decir, para cada vehículo
		$c_i \in C$ y para cada opción $o_j \in O$, $r_{ij} = 1$ si $o_j$ debe ser instalado en $c_i$ y $r_{ij} = 0$
		en otro caso.
\end{itemize}

%Solving a car sequencing problem involves finding an arrangement of the cars
%in a sequence, defining the order in which they will pass along the assembly
%line, such that the capacity constraints are met. We shall use the following
%notations to denote and manipulate sequences:

\begin{itemize}
	\item Una \emph{secuencia}, definida $\pi =< c_{i_{1}}, c_{i_{2}}, \ldots, c_{i_{k}} >$, es una sucesión de vehículos.
	\item El conjunto de todas las secuencias que pueden ser construidas con un conjuntos de autos $C$ es denominada $\amalg_{C}$.
	\item El \emph{largo} de una secuencia $\pi$, denominado $|\pi|$, es el numero de vehículos que contiene.
	\item La \emph{concatenación} de dos secuencias, $\pi_1$ y $\pi_2$, denominada $\pi_1 \cdot \pi_2$, es una secuencia
		compuesta de los vehículos de $\pi_1$ seguidos de los vehículos de $\pi_2$.
	\item Una secuencia $\pi_1$ es una subsecuencia de otra secuencia $\pi_2$ , tomando en cuenta $\pi_1 \subseteq \pi_2$,
		si existen dos secuencias $\pi_3$ y $\pi_4$, tal que $\pi_2 = \pi_3 \cdot \pi_1 \cdot \pi_4$
	\item El \emph{cost} de una secuencia $\pi$ es el numero de las restricciones de capacidad que no se cumplen, es decir:
\end{itemize}

$$costo(\pi) = \sum_{o_{i}\in O} \ \ \ \sum_{\pi_{k}\subseteq \pi\\ \rightarrow |\pi_{k}|=q_i} violaci\acute{o}n(\pi_{k},o_{i})$$
$$\text{donde}\ violaci\acute{o}n(\pi_{k},o_{i}) = \left\{
 \begin{array}{l}
	0\ si\ \sum\limits_{<C_l>\subseteq\pi_k}r_{li}\leq p_i;\\
	1\ \text{en otro caso}\\
 \end{array} \right.$$


Por lo tanto se puede resolver ahora el ejercicio buscando la secuencia con el \textbf{mínimo costo}, compuesta por todos los autos
que serán producidos.

%

Respecto al espacio de búsqueda, si tenemos que $\alpha_o$ es la cantidad de vehículos que tienen que ser producidos
por cada clase/configuración o, con $o\in O$ y sea $n$ la cantidad total de vehículos a producir,
entonces el espacio de búsqueda está compuesto por todas las posibles permutaciones de los vehículos en $C$ que
requieren diferentes configuraciones, particionado en los $o$ tipos distintos tipos de vehículos a construir.

Por lo tanto el tamaño de las posibles soluciones sería:
$$E.B\ =\ \frac{n!}{\prod\limits_{o \in O}\alpha_o}$$
 
Claramente tenemos que darnos cuenta que varias soluciones dentro del presente espacio de búsqueda
no van a ser posibles, pues debemos aplicar las restricciones.

\subsection{Segundo modelo}
\textbf{Programación Lineal Entera (ILP)}

Definición de \emph{variables} y \emph{parámetros}:
\begin{itemize}
	\item \textbf{$opt$}	    Número total de opciones.
	\item \textbf{$cl$}	    Número total de tipos/clases de vehículos.
	\item \textbf{$n_i$}    Número de vehículos en la clase/tipo $i$.
	\item \textbf{$nc$}	    Número total de vehículos.
	\item \textbf{$o_{ik}$}	Parámetro \emph{booleano} que representa si los vehículos del tipo/clase $i$ requieren la opción $k$.
	\item \textbf{$s_k$}    Largo de una secuencia de vehículos consecutiva, donde algunas requieren la opción $k$
	\item \textbf{$r_k$}    Número máximo de vehículos que pueden tener la opción $k$ en una secuencia consecutiva de $s_k$ vehículos.
	\item \textbf{$M$}	 	Valor escalar grande que puede ser fijado a $(s_{k} - r_{k})$
	\item \textbf{$C_{ij}$}	Variable \emph{booleana} que representa si un vehículo del tipo/clase $i$ es asignado a la posición $j$ en la secuencia
		$(C_{ij} = 1)$
	\item \textbf{$Y_{kj}$}	Varibale \emph{booleana} que representa si el número de vehículos que requieren la opción $k$ en una secuencia
		de $s_k$ vehículos, comenzando de la posición $j$ en la secuencia respecto a $r_k$, el máximo permitido $(Y_{kj} = 0)$
	\item \textbf{$pp$}	    Número de vehículos en el periodo anterior, que permiten establecer la transición $(pp = max(s_{k}-1))$
	\item \textbf{$cpp_{mk}$}	Variable \emph{booleana} que representa si el vehículo $m$-ésimo del periodo previo tiene la opción $k$.
	\item \textbf{$Z_{kj}$}	Variable \emph{booleana} que representa si el número de vehículos que necesitan la opción $k$ en una secuencia
		de $s_k$ vehículos, partiendo en la posición $j$ de la secuencia en el periodo anterior, respetando el máximo $r_k$ permitido $(Z_{kj} = 0)$.
\end{itemize}

Por lo tanto, el modelo a continuación se establece para minimizar el número de violaciones de las restricciones de capacidad,
tanto como para opciones de alta prioridad, como de baja.

\textbf{Función objetivo:}
$$\text{Min} F = \sum\limits_{k=1}^{opt} \sum\limits_{j=1}^{nc-s_{k}+1} Y_{kj} + \sum\limits_{k=1}^{opt} \sum\limits_{j=1}^{s_{k}-11} Z_{kj}$$

\textbf{Restricciones}
\begin{itemize}
	\item $\sum\limits_{i=1}^{cl} C_{ij} = 1$, $\forall j = 1,\ldots,nc$
	\item $\sum\limits_{j=1}^{nc} C_{ij} = n_{i}$, $\forall i = 1,\ldots,cl$
	\item $\sum\limits_{i=1}^{cl} \sum\limits_{l=j}^{j+s_{k}-1} o_{ik} \cdot
		C_{il} \leq r_{k} + M\cdot Y_{kj}$,\\ $\forall k = 1,\ldots,opt$ y $\forall j=1,\ldots, nc-s_{k}+1$
	\item $\sum\limits_{i=1}^{cl} \sum\limits_{l=1}^{j} o_{ik} \cdot
		C_{il} + \sum\limits_{m=pp-S_{k}+j+1}^{pp} cpp_{mk} \leq r_{k} + M\cdot Z_{kj}$,\\ $\forall k = 1,\ldots,opt$ y $\forall j=1,\ldots, s_{k}-11$
\end{itemize}

Respecto al espacio de búsqueda, como ésta formulación es ILP igual que el primero modelamiento, sólo sería la permutación de todos
los vehículos que se necesitan,  particionados por las clases de vehículos que se producirán. Obviamente posicionándonos en el peor de los casos
donde tenemos que recorrer todo el espacio de búsqueda.

$$E.B\ =\ \frac{nc!}{\prod\limits_{i = 0}^{cl}\alpha_{n_i}}$$

Por ejemplo, si tuvieramos $30$ vehículos a ensamblar, $4$ clases de autos y
necesitamos $10$ autos de clase $1$, $5$ de clase $2$, $7$ de clase $3$ y $8$ de clase 4.

$$E.B\ =\ \frac{30!}{10\cdot 5\cdot 7\cdot 8}\ =\ 9.473316\times 10^{28}$$




\section{Conclusiones}
\label{sec:conclusiones}
%Conclusiones revelantes del estudio realizado.

En el presente informe se ha dado un estado del arte de un problema muy popular
en el área de la inteligencia artificial, el \emph{Car Sequencing Problem}, siendo éste
una variación de otro problema connotado llamado \emph{Job Shop Scheduling}.
Es tanto la importancia del presente problema, que la \emph{French Society of Operations
Research and Decision-Making Aid} ha decidido ya hace varios años, comenzar lo que se denomina
\emph{The ROADEF challenge} cada dos años, teniendo como objetivo central,  permitir a las personas
que se desarrollan en el área de la industria el presenciar todos los avances y evoluciones
en el ámbito de la Investigación de Operaciones y Análisis de Decisiones, pero no sólo eso
sino el poder enfrentar directamente problemas decisionales complejos, que ocurren en la industria.
Siguiendo la idea anterior, lo importante de éste \emph{Challenge} es que en el 2005, se consideró
como tema principal el \emph{Car Sequencing Problem} debido a la propuesta que realizó RENAULT,
por lo cual uno podrá imaginar la cantidad de avances que se produjeron, pues cada participante
abordaba el problema desde una metodología distinta.

Por otra parte, pareciera que un problema relacionado a \emph{ordenar} un conjunto de vehículos
para ser ensamblados y así obtener el orden más óptimo, no es una tarea difícil, pero claramente
debido a la complejidad que otorgan las restricciones y de que es un problema de la vida real,
presenta un grado de dificultad mayor, lo cual queda reflejado por la cantidad de publicaciones 
e investigaciones que hay al respecto.

Se dieron a conocer también, tres áreas para atacar el presente problema.
Por un lado tenemos los métodos heurísticos que como bien sabemos, es prácticamente jugar a la ruleta
rusa con nuestra investigación, pues la heurística solamente selecciona un objetivo de los dos provenientes
de la definición, una buena solución o un buen tiempo de ejecución. Pero también se presenta que la heurística
es un mecanismo confiable para decidir \emph{utilizarlo} como un apoyo, mas que utilizarlo solo.

Siguiendo con los mecanismos planteados, se vieron también los  métodos exactos,
es decir, técnicas de optimización, donde podemos encontrar la \emph{programación lineal entera},
\emph{branch and bound} y \emph{local search}, los cuales se dedicaban netamente a construir una
solución óptima a partir de los datos que el mismo problema nos entrega. El único problema que tienen
éstas técnicas es que la complejidad temporal va a crecer demasiado con respecto al tamaño de nuestro
\emph{input} del algoritmo.

Dentro de toda la lectura realizada para las distintas técnicas, pude percatarme que las mejores soluciones
siempre son variaciones de métodos o tomar dos técnicas como complementarias, por ejemplo uno de los
mejores resultados fue la combinación de un \emph{Ant Colony Optimization} con una heurística dinámica,
pues claramente se nos señala que el buen uso de una heurística es crucial, es decir, hay que preocuparse
de leer los estudios que se han publicado, par ver cual es la combinación más óptima.

Finalmente, es impresionante la cantidad de estudios con respecto a éste problema en particular,
por lo que podemos darnos cuenta que muchos centros de investigación han dedicado tiempo valioso
para la resolución óptima del \emph{Car Sequencing Problem}, pero no tanto la versión que se estudió,
que es la propuesta por Parello~\cite{parello}, sino mas bien al desafío de la ROADEF.


\section{Bibliografía}
\bibliography{informe1}

\end{document} 
