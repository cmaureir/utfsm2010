%Una explicación breve de lo que consiste el informe.
%Introducción al problema que se estudiará, motivación.
Actualmente la tecnología juega un papel importante en el desarrollo mundial,
permitiendo tener avances en variados temas, ya sean en el área de la salud,
las industrias, etc, pero por lo general el común de las personas asocia
tecnología a elementos como computadoras, teléfonos celulares, cámaras fotográficas,
y varios más, pero si somos formales y buscamos el significado de la tecnología,
nos daremos cuenta que se trata de un \emph{Conjunto de teorías y de técnicas que permiten
el aprovechamiento práctico del conocimiento científico}, siendo ésta la verdadera
motivación del presente trabajo, es decir, que todos los avances tecnológicos
con los que contamos hoy en día, se deben a la optimización de ciertas técnicas científicas.

La evolución notoria de las producciones de distintos productos del mercado, queda demostrada
por la cantidad de elementos que posee en promedio cada persona en el mundo, por lo que ya no es
exclusividad tener un computador, o un vehículo de último modelo, y es aquí donde me quiero detener.

A pesar de que el automóvil fue inventado en Alemania, es en los Estados unidos donde se encuentra
la mayor producción en el mundo. Según estudios (referencia) se habla de que en el 1900 en EEUU habían
unos 4142 vehículos, en 1985 habían 375 millones aproximadamente, lo cual nos entrega la información
de que en 85 años ha habido un crecimiento de 6000000\%.

Volviendo al tema central del presente trabajo, podemos inferir que la industria automotriz,
se debe netamente a las tecnologías que se han ido inventando a través del tiempo, por lo cual
una cadena de producción de automóviles no es un proceso simple, sino que es tan complejo
que se ha requerido de Inteligencia Artificial, para optimizar dichos procedimientos,
pero teniendo siempre en consideración tres puntos fundamentales: Incrementar la eficiencia,
controlar los costos de producción y por último, no descuidar la calidad del producto final

Por lo tanto nuestro estudio se centrará en un problema llamado \emph{Car Sequencing Problem},
que fue nombrado por primera vez por \emph{B.D Parello} en 1986~\cite{parello}, al cual se enfrentan
todas las líneas de ensamblajes de vehículos, que cómo bien sabemos, cada tipo de vehículo puede ser ensamblado
con distintas opciones, colores y accesorios, por lo tanto los vehículos que son producidos en un día,
o un periodo de tiempo no son todos iguales, y por otro lado maquinaría que poseen las grandes industrias son
finitas y es necesario poder aprovecharlas al máximo debido a que no es lo mismo el orden con el cual pasan
los vehículos en las distintas platas de ensamblaje y con ésto mismo se logra obtener un aumento en la producción, pues como
se señalo anteriormente el mercado crece cada día y es necesario poder utilizar todo de la forma
más óptima posible, para poder ganarle a la competencia y de la misma forma, aumentar las ganancias
de la empresa.

Debido a la importancia del presente problema, la marca francesa Renault propuso una variación del \emph{Car Sequencing Problem}
para el \emph{ROADEF challenge} del año 2005. La anterior variación del problema, llamada ``Renault CarSP'' posee una
diferencia bastante notoria, la cual es que contiene además las restricciones de la pintura final del vehículo.
Además éste problema incluye casos reales con más de 1300 vehículos, por lo que su propuesta parece ser un modelo un poco
más realista.

El presente trabajo se divide de la siguiente forma:
En la sección~\ref{sec:definicion}
se ve el \emph{Car Sequencing Problem} en profundidad, es decir, una descripción detalla, variables
relacionadas, las restricciones, y los objetivos.
La sección~\ref{sec:estado} muestra un resumen de los trabajos más importantes realizados hasta el momento,
señalando los métodos con los que fueron desarrollados, los algoritmos asociados, las representaciones
reales que han tenido los mejores resultados, etc.
En la sección~\ref{sec:modelo}
se muestran 2 modelos matemáticos para el presente problema, realizando una descripción detallada
de cada uno y analizando los espacios de búsqueda de cada uno.
Finalmente,
la sección \ref{sec:conclusiones} presenta un resumen
y consideraciones del presente trabajo,
además de dejar planteado el trabajo futuro de un problema tan importante como lo es el \emph{Car
Sequencing Problem}.

