\frame
{
\frametitle{Conclusiones}
\framesubtitle{Simulated Annealing}
\begin{itemize}
	\item La capacidad de {\bf encontrar} buenas soluciones y de poder {\bf escapar} de los mínimos locales depende de los parámetros definidos.
	\item Los {\bf tiempos} de ejecución para la misma cantidad de variables es {\bf similar}, por la forma en que se realizó el algoritmo. 
	\item Lo anterior también ocurre con las soluciones, que son semejantes unas a otras para una misma instancia.
\end{itemize}

}
\frame
{
\frametitle{Conclusiones}
\framesubtitle{Algoritmo Evolutivo + Simulated Annealing}
\begin{itemize}
	\item Mejorar la forma de {\bf generar} la población inicial\\ (Greedy, GRASP, etc)
	\item {\bf Sintonizar} mas profundamente los parámetros.
	\item Realizar un {\bf Control Adaptivo} de Tamaño de la población (edad y tiempo de vida).
\end{itemize}
}
\frame
{
\frametitle{Conclusiones}
\framesubtitle{Generales}
\begin{itemize}
	\item Existe una diferencia notoria entre las implementaciones.
	 \begin{itemize}
	 	\item AE + SA/AM vs SA/AM
		\item Movimiento del SA.
		\item Recalentamiento.
		\item Tiempo de ejecución.
	\item Comparativa con los óptimos conocidos (Métodos completos/incompletos).
	 \end{itemize}
\end{itemize}
}

\frame
{
\begin{center}
	\vspace{1.5cm}
	\Huge EOF
\end{center}
}
