% que es
\frame
{
\frametitle{Secuenciación de ADN}
\begin{center}
	\huge{¿Qué es?}
\end{center}
}

\frame
{
\frametitle{Secuenciación de ADN}
\framesubtitle{¿Qué es?}
\begin{center}
	\huge{Video}
\end{center}
}


\frame
{
\frametitle{Secuenciación de ADN}
\framesubtitle{¿Qué es?}
\begin{itemize}
	\item Conjunto de métodos y técnicas bioquímicas,
		cuya \red{finalidad} es la \blue{determinación del orden} de los nucleótidos (A, C, G y T) en un oligonucleótido de ADN.
	\item La secuencia de ADN constituye la \blue{información genética heredable} (en núcleo celular, plásmidos, mitocondria,
		y cloroplastos) \red{base de desarrollo} de los seres vivos.
\end{itemize}
}

% para que sirve

\frame
{
\frametitle{Secuenciación de ADN}
\begin{center}
	\huge{¿Para qué sirve?}
\end{center}
}
\frame
{
\frametitle{Secuenciación de ADN}
\framesubtitle{¿Para qué sirve?}
\begin{itemize}
	\item Comprensión.
	\item \red{Atajo!} ayuda a los científicos a encontrar a los genes más fácil y rápidamente.
	\item Hay \blue{pistas} para encontrar los genes.
	\item Se espera que sirva para comprender el \red{todo}.\\

	\begin{center}
	\emph{``...cómo los genes trabajan juntos para dirigir el crecimiento,
		desarrollo y mantenimiento de un organismo entero.''}
	\end{center}
\end{itemize}
}

\frame
{
\frametitle{Secuenciación de ADN}
\framesubtitle{¿Para qué sirve?}
\begin{itemize}
	\item Los genes son el $5\%$ del ADN, la secuencia \red{ayudará} a estudiar
		la parte del genoma por afuera de los genes.
	\begin{itemize}
		\item Sitios \blue{reguladores} que controlan la activación/desactivación de genes.
		\item ADN \blue{sin} sentido.
	\end{itemize}
\end{itemize}
}

% como se hace

\frame
{
\frametitle{Secuenciación de ADN}
\begin{center}
	\huge{¿Cómo se hace?}
\end{center}
}
\frame
{
\frametitle{Secuenciación de ADN}
\framesubtitle{¿Cómo se hace?}
\begin{itemize}
	\item Se secuencia en \emph{partes}.
	\item \red{No} se puede completo (\blue{limitación} métodos sólo con fragmentos)
	\item Fragmentar en pedazos \emph{pequeños}, secuenciarlos y reconstruirlos en orden.
	\item Dos técnicas más utilizadas \emph{(con ventajas y desventajas)}.
	\begin{itemize}
		\item Shotgun.
		\item Hierarchical Shotgun.
	\end{itemize}
	\item La secuenciación del genoma humano ha involucrado una combinación de \blue{ambas} técnicas. 
\end{itemize}
}

% como funciona
\frame
{
\frametitle{Secuenciación de ADN}
\begin{center}
	\huge{¿Cómo funciona?}
\end{center}
}
\frame
{
\frametitle{Secuenciación de ADN}
\framesubtitle{¿Cómo funciona?}
\begin{enumerate}
	\item<1-> Electroforesis para \blue{separar} los fragmentos de ADN.% que difieren en longitud por una base.
	\begin{itemize}
		\item \emph{Técnica para la separación de moléculas según movilidad en un campo eléctrico}.
	\end{itemize}
	\item<2-> Por lo que el ADN a secuenciar se coloca en un \blue{extremo} del gel.
	\item<3-> Electrodos en los extremos (aplicamos \blue{corriente}).
	\item<4-> Migración de fragmentos de ADN por \blue{tamaño} (a través del gel).
\end{enumerate}
}

\frame
{
\frametitle{Secuenciación de ADN}
\framesubtitle{¿Cómo funciona?}
\begin{itemize}
	\item Electroforesis es capaz \red{sólo} de separar fragmentos de 500 pares de bases (justificación anterior!).
	\item \red{Dato:} Hasta 1980, las electroforesis eran leídas por una \blue{persona}.
	\item Hoy en día el proceso es \emph{automático}.
	\item En producir una secuencia de 20.000 a 50.000 bases (Persona $\rightarrow$ 1 año, Máquina $\rightarrow$ horas).
	\item Máquinas con diseño \blue{basado} en el proceso manual.
\end{itemize}
}

\frame
{
\frametitle{Secuenciación de ADN}
\framesubtitle{¿Cómo funciona?}
\begin{itemize}
	\item Usando máquinas. 
	\begin{enumerate}
		\item<1-> Se ubica el gel en un espacio entre dos vidrios de medio milímetro.
		\item<2-> Se espera que el gel se endurezca
		\item<3-> El ADN se siembra en cada una de las 96 calles que corren a lo largo del gel.
		\item<4-> Se aplica corriente eléctrica.
		\item<5-> Los fragmentos de ADN migran de acuerdo a su tamaño.
		\item<6-> La máquina lee el orden de las bases de ADN y guarda la información. 
	\end{enumerate}
\end{itemize}
}

% como se reconoce una base

\frame
{
\frametitle{Secuenciación de ADN}
\begin{center}
	\huge{¿Cómo se reconoce una base?}
\end{center}
}

\frame
{
\frametitle{Secuenciación de ADN}
\framesubtitle{¿Cómo se reconoce una base?}
\begin{itemize}
	\item Secuenciadores automáticos \red{no} ven al ADN (hay que \blue{prepararlo})
	\item El ADN debe estar
	\begin{itemize}
		\item fragmentado
		\item copiado
		\item modificado químicamente
		\item unido a fragmentos fluorescentes (por bases)
	\end{itemize}
	\item Luego del procedimiento, se reconocen por la \blue{fluorescencia} (láser).
\end{itemize}
}

%\frame
%{
%\frametitle{Secuenciación de ADN}
%\framesubtitle{¿Cómo se reconoce una base?}
%\begin{itemize}
%	\item Antes de ser secuenciado, un fragmento de ADN se copia muchas veces,
%		luego se divide en cuatro partes para otra fase de copiado.
%	\item En esta segunda fase de copiado, se agrega una pequeña cantidad de base
%		químicamente modificada a cada parte (o sea, A modificada a una parte,
%		C modificada a otra y así sucesivamente).
%	\item Cuando una de estas bases modificadas es incorporada en la molécula de ADN,
%		la cadena de bases se frena.
%\end{itemize}
%}
%
%\frame
%{
%\frametitle{Secuenciación de ADN}
%\framesubtitle{¿Cómo se reconoce una base?}
%\begin{itemize}
%	\item El resultado de todo esto es que una parte contendrá solo fragmentos que
%		terminen en A, otra parte fragmentos que terminen en C y así sucesivamente.
%	\item Además, en la segunda fase de copiado, se le agrega un tinte fluorescente
%		distinto a cada parte. 
%	\item Luego se siembra en una misma calle una mezcla de las cuatro partes.
%\end{itemize}
%}
%
%\frame
%{
%\frametitle{Secuenciación de ADN}
%\framesubtitle{¿Cómo se reconoce una base?}
%\begin{itemize}
%	\item Como las moléculas más pequeñas migran más rápido,
%		los fragmentos de ADN se leen en orden creciente de tamaño,
%		cada fragmento con una base más que el anterior.
%	\item A medida que los fragmentos migran, un láser detecta los tintes de las
%		moléculas fluorescentes y así se corresponde con la base: A, C, T o G. 
%\end{itemize}
%}

% que ocurre despues

\frame
{
\frametitle{Secuenciación de ADN}
\begin{center}
	\huge{¿Qué ocurre después?}
\end{center}
}

\frame
{
\frametitle{Secuenciación de ADN}
\framesubtitle{¿Qué ocurre después?}
\begin{itemize}
	\item Salida del secuenciador automático.
	\begin{itemize}
		\item Llegan a una \blue{secuencia cruda} (agujeros, errores y ambigüedades).
		\item Proceso de \blue{limpieza} y \blue{ordenamiento} (cuidado con el tiempo)
	\end{itemize}
\end{itemize}
}

\frame
{
\frametitle{Secuenciación de ADN}
\framesubtitle{¿Qué ocurre después?}
\begin{itemize}
	\item Ordenamiento.
	\begin{itemize}
		\item Mediante un programa.
		\item El programa busca y analiza \blue{superposiciones} o secuencias \blue{idénticas} de ADN en diferentes fragmentos.
		\item Los fragmentos que contienen superposiciones se encuentran \red{juntos} en la secuencia final.
	\end{itemize}
\end{itemize}
}

\frame
{
\frametitle{Secuenciación de ADN}
\framesubtitle{¿Qué ocurre después?}
\begin{itemize}
	\item Aceptar o rechazar.
	\begin{itemize}
		\item Genoma no secuenciado \red{no} garantiza ser correcto.
		\item Pueden ocurrir \red{errores} (fragmenta el ADN, copia, secuencia u ordena).
		\item \blue{Truco:} secuenciar el genoma más de una vez. 
		\item Luego del ordenamiento, sigue el proceso de terminación. (no tenemos intuición)
	\end{itemize}
\end{itemize}
}

% distincion

\frame
{
\frametitle{Secuenciación de ADN}
\begin{center}
	\huge{¿Qué hace la secuenciación del genoma humano distinto?}
\end{center}
}

\frame
{
\frametitle{Secuenciación de ADN}
\framesubtitle{¿Qué hace la secuenciación del genoma humana distinto?}
\begin{itemize}
	\item El genoma humano es mucho más \blue{grande} que los genomas que se habían secuenciado anteriormente.
	\item Los anteriores secuenciados eran virus, bacterias u otros organismos simples.
	\item Genoma humano posee 25\% a 50\% de ADN repetitivo, bacterias y virus no mucho.
	\item El ADN repetitivo no sólo es difícil de ordenar sino también de secuenciar.
\end{itemize}
}
