%Las conclusiones deberían incluir:

Luego de las dos aproximaciones previas para poder sintonizar nuestro algoritmo,
en las cuales se obtuvieron bastantes mejoras, es necesario formalizar un poco
el desempeño de cada forma, lo que puede ser posible comparando tiempos de ejecución,
\emph{fitness}, etc.

%- Comparación de resultados de la sintonización manual vs sintonización automática.\\

Primero que todo es necesario realizar una pequeña comparación entre los \emph{fitness}
obtenidos en cada procedimiento por cada instancia.

\begin{center}
\begin{tabular}{|l|c|c|c|c|c|}
    \hline
    \textbf{Instancia} & \textbf{prueba1} & \textbf{prueba2} & \textbf{prueba3} & \textbf{prueba4} & \textbf{prueba5}\\\hline
    \texttt{pb\_200\_01.txt} & \red{167} & 186 & 200 & 192 & 203\\\hline 
    \texttt{pb\_200\_09.txt} & \red{152} & 166 & 178 & 180 & 189\\\hline
    \texttt{pb\_200\_10.txt} & \red{126} & 144 & 164 & 160 & 159\\\hline
\end{tabular}
\label{Fitness de cada prueba}
\end{center}

Podemos notar que los mejores resultados fueron obtenidos al momento de variar el tamaño de la población,
lo que claramente nos indica lo trascendental de su función dentro del algoritmo, pero de todas formas,
esto puede ser debido a que la ``configuración base'' que se utilizó ya se encontraba un poco sintonizada
y el único parámetro que faltaba mejorar era \texttt{POP}.

%- Comparación en términos de tiempo utilizado.\\

Luego es necesario tener presente el tiempo en total que demoró cada prueba por cada instancia.

\begin{center}
\begin{tabular}{|l|c|c|c|c|c|}
    \hline
    \textbf{Instancia} & \textbf{prueba1} & \textbf{prueba2} & \textbf{prueba3} & \textbf{prueba4} & \textbf{prueba5}\\\hline
    \texttt{pb\_200\_01.txt} & 1608.853 & 256.177 & 13.246 & 132.608 & 161\\\hline 
    \texttt{pb\_200\_09.txt} & 1593.739 & 262.906 & 12.027 & 132.068 & 195\\\hline
    \texttt{pb\_200\_10.txt} & 1662.580 & 256.546 & 12.133 & 132.124 & 169\\\hline
\end{tabular}
\label{Tiempo total de cada prueba [s]}
\end{center}

Nos damos cuenta que claramente al realizar cambios en el tamaño de la población, el tiempo que tomó la prueba fue
notablemente superior al resto por un un hecho totalmente predecible, ya que si consideramos poblaciones muy grandes
todo el procedimiento y trabajo que se realiza con ella va a tomar mucho más tiempo.

De la misma forma, hay que notar que el segundo lugar de los mejores \emph{fitness}, que lo tenía la \blue{prueba2},
se demora bastante menos tiempo y los resultados no son tan malos en comparación con el primer lugar, lo que nos podría
servir de consejo a la hora de buscar rapidez en la ejecución y buenos resultados, sólo cambiando la cantidad de generaciones
con una población no tan grande.


%- Conclusiones con respecto a la utilidad del sintonizador usado, ventajas desventajas.\\
Una de las ventajas principales al momento de utilizar REVAC es que no necesitamos una configuración inicial,
sólo necesitamos establecer cuales son los parámetros a sintonizar y cual es el rango de sus valores, lo cual
facilita mucho a la hora de poder sintonizar cualquier algoritmo, pues no es un gasto de tiempo muy significativo.

El problema que surge a la hora de seleccionar este sintonizador es su resultado, debido a que no nos entrega
un valor determinado para cada parámetro, sino que nos entrega un rango de valores en que nuestro algoritmo se comporta mejor,
pero de la mano de ésto es que REVAC considera que los valores de nuestros parámetros siguen una distribución uniforme
por lo que las mejores soluciones que nos entrega estaría todas cercanas al vecindario central de dicha distribución.

%- Conclusiones en relación a su algoritmo, características de los parámetros, características de la búsqueda.\\
Con respecto a la implementación del algoritmo de ``Selección Clonal''~\ref{sec:anexo2}, es correcto apreciar que ha tenido una notable
evolución respecto a sus origines, pues en el camino se han corregido bastantes errores de implementación y también
se ha considerado la posibilidad de ir realizando mejoras sustanciales de acuerdo a ciertos procedimiento internos,
pues como ya bien se sabe, los algoritmos de optimización poseen una estructura ``recomendable'' pero no es una regla
fundamental, por lo que cada implementación será distinta, y si no es en la estructura, será en la implementación de cada
paso del algoritmo, por ejemplo en el presente algoritmo se modifico más de tres veces la forma de realizar la hipermutación
de los individuos seleccionados, pasando de utilizar un factor, una constante hasta la guía actual que posee respecto
del índice del individuo en la población determinada.

Los parámetros utilizados podrían mejorarse en el sentido de  ampliar o modificar los rangos de sus valores, sobre todo
en el caso del tamaño de la población y de la cantidad de generaciones debido a que son parámetros esenciales al momento
de su ejecución. Por otro lado los demás parámetros fueron probado dentro de todo su rango posible, por lo que no sería
necesario poder variarlos un poco.

Un problema que posee la implementación actual, es que indirectamente efectúa un \emph{elitismo} muy fuerte, lo que se ha
podido mejorar respecto a las implementaciones pasadas, pero que es un factor importante que se podría seguir trabajando a futuro.

