\subsection{Anexo I}
Pseudocódigo de REVAC.
\label{sec:anexo1}

\begin{tabular}{|p{12cm}|}
\hline

\begin{verbatim}
    Procedure Revac
    oP <- generar M configuraciones candidatas
    Evaluar cada configuración en oP
    While not max_evaluations do
        oNew <- cruzamiento uniforme de los N mejores
        oNew <- mutar oNew en intervalo 2*H
        Evaluar configuración oNew
        Reemplazar la configuración mas antigua en oPp por oNew
        Calcular entropía para cada parámetro en oP
    End while
    Return media y entropía de cada parámetro en oP
\end{verbatim}\\

\hline
\end{tabular}

\subsection{Anexo II}
Pseudocódigo del algoritmo de Selección Clonal.
\label{sec:anexo2}

\begin{tabular}{|p{12cm}|}
\hline
\begin{verbatim}
    Inicio
    g <- 0 // numero de generaciones
    Leer datos de entrada
    Población <- Generar población inicial
    Evaluar población

    Mientras g < GENS
        Limpiar poblaciones
        Seleccionar individuos a clonal (ruleta)
        Clonación (fórmula)
        Hipermutación mediante Swap
        Evaluar población mutada
        Selección de clones (Mejores)
        Inserción de clones en nueva población
        Generar elementos nuevos aleatorios
        Inserción de elementos nuevos al población
        g <- g + 1
    Fin Mientras

    Imprimir resultados
    Fin
\end{verbatim}\\

\hline
\end{tabular}
