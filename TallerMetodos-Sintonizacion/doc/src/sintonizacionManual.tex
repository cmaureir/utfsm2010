%Se deben explicar todas las pruebas de sintonización manual realizadas. Para esto debe usar la estructura que se adjunta abajo.\\
%
%Ejemplo de análisis un parámetro (aislar comportamiento)\\ \\
%\textbf{Prueba}: prueba1 \\
%\textbf{Parámetros involucrados}: tama\~no de repertorio / partículas.\\
%\textbf{Objetivo}: Analizar el comportamiento de acuerdo a fitness y tiempo de ejecución del parámetro en un rango de valores.\\
%\textbf{Metodología}: Se probarán varios valores en el rango de valores del parámetro. 
%
%\begin{figure}[h!]
% \centering
%  \subfigure[]{
%      \includegraphics[scale=0.3]{imagenes/r_nl8_i4000}
%      \label{gra:repertorio vs fitness}}
%  \subfigure[]{
%      \includegraphics[scale=0.3]{imagenes/rt_nl8_i4000}
%      \label{gra:repertorio vs tiempo, iteraciones 4000}}
% \caption[Estudio de fitness y tama\~no de repetorio]{Estudio de fitness y tama\~no de repetorio }
% \label{fig:estudio repertorio y fitness NL8}
%\end{figure}
%
%Ejemplo de análisis más de un parámetro (comportamiento conjunto)\\ \\
%\textbf{Prueba}: prueba2 \\
%\textbf{Parámetros involucrados}: tasa de clonación - factor de clonación / factor cognitivo - factor social. \\
%\textbf{Objetivo}: Estudiar el efecto de los parámetros en conjunto de acuerdo a fitness en un rango de valores.\\
%\textbf{Metodología}: Se prueban varias combinaciones de valores para ver el efecto de los parámetros y buscar una relación entre sus valores.
%
%\begin{figure}[h!]
% \centering
%  \subfigure[]{
%      \includegraphics[scale=0.25, angle=270]{imagenes/nl6}}
%  \subfigure[]{
%      \includegraphics[scale=0.25, angle=270]{imagenes/nl12}}
%
% \caption[Estudio dos parámetros vs fitness]{Estudio dos parámetros vs fitness}
% \label{fig:estudio dos param}
%\end{figure}
%
%Finalmente en base a sus estudios de sintonización debe elegir una configuración de parámetros para cada instancia, indicar el fitness encontrado con ésta e identificar claramente las características del algoritmo que observa en base a estos estudios. Además debe incluir una estimación del tiempo que usted empleo en realizar esta sintonización.
