%Debe identificar el algoritmo sintonizador que elegirá
% y los parámetros que serán sintonizados por éste.
Luego del proceso manual anterior, es tiempo de automatizar el proceso,
que para algoritmos mucho más complejos y elaborados puede ser realmente un alivio,
a la hora de buscar los valores adecuados para los parámetros utilizados.

En la presente sección, utilizaremos un algoritmo sintonizador basado en búsqueda,
muy similar a un algoritmo genético, nos referimos a REVAC~\cite{REVAC},
el cual fue propuesto por Eiben y Nannen hace 3 años. (véase pseudocódigo en Anexo I~\ref{sec:anexo1})

La idea principal de REVAC es que nos entrega una herramienta de estimación
de distribución, la cual va a requerir un rango determinado de valores para
cada parámetro que deseemos sintonizar

Su parentesco con los algoritmos genéticos es que al comienzo utiliza un
conjunto de configuraciones de parámetros como población, la cual es inicializada
utilizando una distribución uniforme para los valores de los parámetros a sintonizar.

Algunas otras características son que en cada iteración del algoritmo, genera sólo
un hijo, a diferencia de los algoritmos evolutivos comunes, por otra parte posee
un cruzamiento uniforme considerando a los mejores elementos.
%Entropía → Relevancia de un parámetro
%Mutación con intervalo 2*H

Finalmente es necesario precisar que REVAC sólo necesitará un intervalo y precisión
de los parámetros a sintonizar, con los cuales después de cumplir todas las iteraciones,
o converger, nos entregará el valor e intervalo de los valores para cada parámetro.\\ 

\textbf{Prueba:} \blue{prueba5}\\

\textbf{Sintonizador:} \texttt{REVAC}\\

\textbf{Parámetros a Sintonizar:}\\

\begin{itemize}
	\item \texttt{clonationRate} TASA\_CL \blue{[0,1]}
	\item \texttt{clonationFactor} FACT\_CL \blue{[0,1]}
	\item \texttt{replaceRate} TASA\_RP \blue{[0,1]}
\end{itemize}

\textbf{Detalles del sintonizador:}\\

\begin{itemize}
	\item Iteraciones: \blue{200}
	\item Tamaño de población (M): \blue{18}
	\item Cantidad de padres (N): \blue{6}
	\item Intervalo de mutación (H) : \blue{3}
	\item Tipo de búsqueda: Algoritmo genético.
\end{itemize}

\textbf{Resultados:}

\begin{center}
\begin{tabular}{|l|c|c|c|c|c|}
	\hline
	\textbf{Instancia} & \textbf{clonationRate} & \textbf{clonationFactor} & \textbf{replaceRate} & \textbf{Fitness} & \textbf{Tiempo [s]} \\\hline
	\texttt{pb\_200\_01.txt} & 0.56 & 0.65 & 0.57 & 203 & 161 \\\hline
	\texttt{pb\_200\_09.txt} & 0.95 & 0.60 & 0.13 & 189 & 195 \\\hline
	\texttt{pb\_200\_10.txt} & 0.80 & 0.66 & 0.60 & 159 & 169\\\hline
\end{tabular}
\end{center}
