%Definición y análisis de cada uno de los parámetros que posea su algoritmo. \\
%
%Ej. \textit{Temperatura (Simulated Annealing); [0,1] parámetro que define la probabilidad con la cual se aceptan soluciones de peor calidad en cada iteración. Un valor alto para este parámetro involucra una fuerte e exploración, en cambio valores bajos indican una inclinación a la explotación.} \\
%Puede incluir diagramas o todo lo que sea necesario para dejar claras las caracteristicas y función que cumple el parámetro en su algoritmo.

\begin{itemize}
	\item \texttt{POP} \blue{(20)}, Parámetro que define el tamaño de cada población al momento de comenzar
			las iteraciones a través de la cantidad de generaciones.
			Este parámetros influye notablemente en el tiempo que demora la ejecución de nuestro algoritmo,
			pues si tenemos poblaciones muy grandes todo el tratamiento que le damos a los individuos,
			como la hipermutación y la clonación tomará mucho más tiempo.
	\item \texttt{GENS} \blue{(200)}, Parámetro que define el número total de generaciones en las cuales
			el algoritmo se mantendrá en ejecución. Este parámetro es la actual condición de término
			por lo cual es escencial a la hora de obtener buenos o malos resultados, ya que si tenemos
			muy pocas generaciones, para muchas variables puede que no se alcance a cumplir bien los
			objetivos del algoritmo y que se termine no con los mejores resultados.
	\item \texttt{selRate} POP*0.4  // Tasa para la cantidad de elementos seleccionados
	\item \texttt{swap} VARS*0.4 // Cantidad de swap realizados en el movimiento
	\item \texttt{mutRate} POP*0.02

	\item \texttt{clonationFactor} = 0.0;//0.5 // Factor para calcular individuos clonados
	\item \texttt{clonationRate} = 0.0; //POP*0.4 // Tasa para realizar la clonación
	\item \texttt{replaceRate} =  0.0; //POP*0.6  // Tasa para la cantidad de elementos reemplazados
\end{itemize}
