%Definición y análisis de cada uno de los parámetros que posea su algoritmo. \\
%
%Ej. \textit{Temperatura (Simulated Annealing); [0,1] parámetro que define la probabilidad con la cual se aceptan soluciones de peor calidad en cada iteración. Un valor alto para este parámetro involucra una fuerte e exploración, en cambio valores bajos indican una inclinación a la explotación.} \\
%Puede incluir diagramas o todo lo que sea necesario para dejar claras las caracteristicas y función que cumple el parámetro en su algoritmo.

\begin{itemize}
	\item \texttt{POP} \blue{[10,210]}, Parámetro que define el tamaño de cada población al momento de comenzar
			las iteraciones a través de la cantidad de generaciones.
			Este parámetros influye notablemente en el tiempo que demora la ejecución de nuestro algoritmo,
			pues si tenemos poblaciones muy grandes todo el tratamiento que le damos a los individuos,
			como la hipermutación y la clonación tomará mucho más tiempo.

	\item \texttt{GENS} \blue{[10,2000,30]}, Parámetro que define el número total de generaciones en las cuales
			el algoritmo se mantendrá en ejecución. Este parámetro es la actual condición de término
			por lo cual es esencial a la hora de obtener buenos o malos resultados, ya que si tenemos
			muy pocas generaciones, para muchas variables puede que no se alcance a cumplir bien los
			objetivos del algoritmo y que se termine no con los mejores resultados.

	\item \texttt{clonationFactor} \blue{[0,1]} \red{$(\beta)$}, Parámetro que sirve para definir la cantidad de clones
			que vamos a generar a partir de un elemento anteriormente seleccionado, considerando la
			fórmula propuesta por De Castro~\cite{decastro}, en el cual consideramos tres elementos:
			``factor de clonación ($\beta$)'', ``Total de anticuerpos($N$)'' y ``Afinidad (ranking) ($a$)''.
			
			La relación está dada por un número $m$ que equivale a la cantidad de clones que se generarán
            para cada anticuerpo ordenados por afinidad, partiendo del mejor al peor:

            $$m\ =\ \left\lceil\frac{\beta \cdot N}{a}\right\rceil$$

            Con ésto estamos siguiendo la idea central del algoritmo, pues estamos favoreciendo a que se clonen más
            los mejores elementos de nuestra población.

			Adicionalmente, cabe señalar que es posible considerar un $\beta > 1$, dependiendo del nivel
			de clonación que estemos interesados en utilizar.

	\item \texttt{clonationRate}   \blue{[0,1]}, Parámetro que define la cantidad de elementos que seleccionaremos
			para poder comenzar el proceso de clonación. Aquí se seleccionan \texttt{POP*clonationRate} elementos
			de una población que ya se encuentra ordenada de acuerdo al \emph{fitness}, es decir, de los mejores
			a los peores, por lo tanto nos aseguramos de que las futuras clonaciones sean a buenos individuos.

	\item \texttt{replaceRate}     \blue{[0,1]}, Parámetro que define la cantidad de nuevos elementos que
			integramos en nuestra población para poder así combatir el estancamiento en óptimos locales,
			de ésta forma se generan una cantidad de \texttt{POP*replaceRate} nuevos individuos, en las posiciones
			de los peores elementos, siendo éste el mecanismo para tener variación en nuestra población.

\end{itemize}
