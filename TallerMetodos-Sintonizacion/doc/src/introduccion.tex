% Una explicación breve de lo que consiste el informe.
El presente informe consiste en retomar nuestro trabajo principal
de poder resolver el ``Car Sequencing Problem'' con un algoritmo
inmune, más específico ``Selección Clonal'', pero fijándonos ahora
en la \textbf{sintonización} de nuestro algoritmo.

Se presentan los detalles de los parámetros del algoritmo,
para luego explicar el procedimiento y mostrar los resultados
de la sintonización \emph{manual} y \emph{automática} que se pudo
aplicar, obteniendo un mejor desempeño en los resultados al utilizar
alguna de las instancias más importantes.

Con respecto a la ``sintonización manual'' se han considerado cuatro
pruebas distintas, las cuales son variar el \emph{tamaño de la población},
el \emph{número de generaciones}, la \emph{tasa de reemplazo} y finalmente
realizar una sintonización en conjunto entre el \emph{factor de clonación}
y la \emph{tasa de clonación}.

Por otro lado respecto a la ``sintonización automática'' se utilizará
un algoritmo REVAC, para buscar los valores adecuados para la \emph{tasa
 de  reemplazo}, el \emph{factor de clonación} y la \emph{tasa de reemplazo}. 

Finalmente, en las conclusiones es posible obtener la información
pertinente, para dar a conocer al lector cual de los dos mecanismo
de sintonización funcionó mejor y algunos consejos al momento
de sintonizar.
