\documentclass[letter, 10pt]{article}
\usepackage[utf8]{inputenc}
\usepackage[spanish]{babel}
\usepackage{amsfonts}
\usepackage{amsmath}
\usepackage[dvips]{graphicx}
\usepackage{subfigure}
\usepackage{url}
\usepackage{hyperref}
\usepackage{color}
\usepackage[top=3cm,bottom=3cm,left=3.5cm,right=3.5cm,footskip=1.5cm,headheight=1.5cm,headsep=.5cm,textheight=3cm]{geometry}

\definecolor{red}{rgb}{1,0,0}
\definecolor{green}{rgb}{0,1,0}
\definecolor{blue}{rgb}{0,0,1}
\newcommand{\blue}{\textcolor{blue}}
\newcommand{\red}{\textcolor{red}}
\newcommand{\green}{\textcolor{green}}


\begin{document}
\bibliographystyle{plain}
\pagestyle{empty}


\title{Taller de Modelos y Métodos Cuantitativos \\ \begin{Large}Sintonización: Algoritmo \emph{Clonal Selection} para el problema \emph{Car Sequencing Problem}\end{Large}}
\author{Cristián D. Maureira Fredes.}
\date{\today}
\maketitle

\section{Introducción}
Una expresión regular,
a menudo llamada también patrón,
es una expresión que describe un conjunto de cadenas sin enumerar sus elementos.

Además,
normalmente representan otro grupo de caracteres mayor,
de tal forma que podemos comparar el patrón con otro conjunto de caracteres para ver las coincidencias.

La mayoría de las formalizaciones proporcionan los siguientes constructores:
una expresión regular es una forma de representar a los lenguajes regulares
(finitos o infinitos) y se construye utilizando caracteres del alfabeto
sobre el cual se define el lenguaje.

Específicamente,
las expresiones regulares se construyen utilizando los operadores
unión, concatenación y clausura de Kleene.

Las expresiones regulares en UNIX (RE's) están recogidas en el estándar POSIX 1003.2.



\section{Sintonización}

\subsection{Parámetros del algoritmo}
%Definición y análisis de cada uno de los parámetros que posea su algoritmo. \\
%
%Ej. \textit{Temperatura (Simulated Annealing); [0,1] parámetro que define la probabilidad con la cual se aceptan soluciones de peor calidad en cada iteración. Un valor alto para este parámetro involucra una fuerte e exploración, en cambio valores bajos indican una inclinación a la explotación.} \\
%Puede incluir diagramas o todo lo que sea necesario para dejar claras las caracteristicas y función que cumple el parámetro en su algoritmo.

\begin{itemize}
	\item \texttt{POP} \blue{(20)}, Parámetro que define el tamaño de cada población al momento de comenzar
			las iteraciones a través de la cantidad de generaciones.
			Este parámetros influye notablemente en el tiempo que demora la ejecución de nuestro algoritmo,
			pues si tenemos poblaciones muy grandes todo el tratamiento que le damos a los individuos,
			como la hipermutación y la clonación tomará mucho más tiempo.
	\item \texttt{GENS} \blue{(200)}, Parámetro que define el número total de generaciones en las cuales
			el algoritmo se mantendrá en ejecución. Este parámetro es la actual condición de término
			por lo cual es escencial a la hora de obtener buenos o malos resultados, ya que si tenemos
			muy pocas generaciones, para muchas variables puede que no se alcance a cumplir bien los
			objetivos del algoritmo y que se termine no con los mejores resultados.
	\item \texttt{selRate} POP*0.4  // Tasa para la cantidad de elementos seleccionados
	\item \texttt{swap} VARS*0.4 // Cantidad de swap realizados en el movimiento
	\item \texttt{mutRate} POP*0.02

	\item \texttt{clonationFactor} = 0.0;//0.5 // Factor para calcular individuos clonados
	\item \texttt{clonationRate} = 0.0; //POP*0.4 // Tasa para realizar la clonación
	\item \texttt{replaceRate} =  0.0; //POP*0.6  // Tasa para la cantidad de elementos reemplazados
\end{itemize}


\subsection{Sintonización Manual}
%Se deben explicar todas las pruebas de sintonización manual realizadas. Para esto debe usar la estructura que se adjunta abajo.\\

Para la presente sección, se requiere de una configuración inicial,
para poder comenzar a evaluar el rendimiento del algoritmo para
cada situación donde se quiera sintonizar manualmente algún parametro.

Los valores de la configuración inicial están dado en base a la experiencia
del trabajo pasado, es decir, en la \emph{primera implementación}.

\begin{itemize}
	\item \texttt{POP = 20}
	\item \texttt{GENS = 200}
	\item \texttt{clonationFactor = 0.4}
	\item \texttt{clonationRate = 0.5}
	\item \texttt{replaceRate = 0.6}
\end{itemize}

Complementariamente, se han considerado tres instancias provenientes de la CSPlib~\cite{CSP}
más precisamente de la sección \emph{``30 new hard problems from Caroline Gagne ''},
donde se han escogido tomando en cuenta el mejor resultado encontrado.
Las instancias son:

\begin{center}
	\begin{tabular}{|l|c|}
	\hline
	\textbf{Instancia} & \textbf{Mejor resultado conocido} \\\hline
	\texttt{pb\_200\_01.txt} & 0 \\\hline
	\texttt{pb\_200\_09.txt} & 10 \\\hline
	\texttt{pb\_200\_10.txt} & 19 \\\hline
	\end{tabular}
\end{center}
\newpage
\subsubsection{Tamaño de población}

\textbf{Prueba}: \blue{prueba1}\\

\textbf{Parámetros involucrados:} Tamaño de población \texttt{(POP)}.\\

\textbf{Objetivo:} Analizar el comportamiento de acuerdo a fitness y tiempo de ejecución del parámetro en un rango de valores.\\

\textbf{Metodología:} Se probarán varios valores en el rango de valores del parámetro \blue{[10,210]}.\\

\textbf{Gráfico:}\\

\begin{figure}[h!]
\begin{center}
	\includegraphics[width=0.95\textwidth]{img/1.pdf}
	\caption{Comparaci\'on de las tres instancias dado un cambio en la poblaci\'on}
	\label{fig:1}
\end{center}
\end{figure}

\textbf{Configuración escogida:}\\

\begin{center}
\begin{tabular}{|l|c|c|c|c|}
	\hline
	\textbf{Instancia} & \textbf{POP} & \textbf{Mejor resultado} & \textbf{Tiempo [s] } & \textbf{Tiempo total [s] }\\\hline
	\texttt{pb\_200\_01.txt} & 195 & 167 & 15.110 & 1608.853 \\\hline 
	\texttt{pb\_200\_09.txt} & 150 & 152 & 11.243 & 1593.739 \\\hline
	\texttt{pb\_200\_10.txt} & 158 & 126 & 11.210 & 1662.580 \\\hline
\end{tabular}
\end{center}

\newpage
\subsubsection{Número de generaciones}

\textbf{Prueba}: \blue{prueba2}\\

\textbf{Parámetros involucrados:} Número de generaciones \texttt{(GENS)}.\\

\textbf{Objetivo:} Analizar el comportamiento de acuerdo a fitness y tiempo de ejecución del número de generaciones entre un rango de valores.\\

\textbf{Metodología:} Se probarán varios valores en el rango de valores del parámetro \blue{[10,2000]} (iterando de 30 en 30).\\

\textbf{Gráfico:}\\

\begin{figure}[h!]
\begin{center}
	\includegraphics[width=0.95\textwidth]{img/2.pdf}
	\caption{Comparaci\'on de las tres instancias dado un cambio en el n\'umero de generaciones}
	\label{fig:2}
\end{center}
\end{figure}

\textbf{Configuración escogida:}\\

\begin{center}
\begin{tabular}{|l|c|c|c|c|}
	\hline
	\textbf{Instancia} & \textbf{GENS} &\textbf{Mejor resultado} & \textbf{Tiempo [s] } & \textbf{Tiempo total [s]}\\\hline
	\texttt{pb\_200\_01.txt} & 1810 & 186 & 7.782 & 256.177 \\\hline
	\texttt{pb\_200\_09.txt} & 1960 & 166 & 8.460 & 262.906 \\\hline
	\texttt{pb\_200\_10.txt} & 1630 & 144 & 7.562 & 256.546 \\\hline
\end{tabular}
\end{center}

\newpage
\subsubsection{Tasa de reemplazo}

\textbf{Prueba}: \blue{prueba3}\\

\textbf{Parámetros involucrados:} Tasa de reemplazo \texttt{(replaceRate)}.\\

\textbf{Objetivo:} Analizar el comportamiento de acuerdo a fitness y tiempo de ejecución de la tasa de reemplazo entre un rango de valores.\\

\textbf{Metodología:} Se probarán varios valores en el rango de valores del parámetro \blue{[0,1]}.
Para éste caso en particular, se ejecutó el algoritmo 10 veces por cada valor del parámetros y luego se seleccionó la mejor
para poder hacer el siguiente análisis.\\

\textbf{Gráfico:}\\

\begin{figure}[h!]
\begin{center}
	\includegraphics[width=0.95\textwidth]{img/3.pdf}
	\caption{Comparaci\'on de las tres instancias dado un cambio en la tasa de reemplazo}
	\label{fig:3}
\end{center}
\end{figure}

\textbf{Configuración escogida:}\\

\begin{center}
\begin{tabular}{|l|c|c|c|c|}
	\hline
	\textbf{Instancia} & \textbf{POP*replaceRate} & \textbf{Mejor resultado} & \textbf{Tiempo [s]} & \textbf{Tiempo total [s]}\\\hline
	\texttt{pb\_200\_01.txt} & 8 & 200 & 1.374 & 13.246 \\\hline
	\texttt{pb\_200\_09.txt} & 0 & 178 & 1.463 & 12.027 \\\hline
	\texttt{pb\_200\_10.txt} & 0 & 164 & 0.515 & 12.133   \\\hline
\end{tabular}
\end{center}


\newpage
\subsubsection{Factor de clonación y Tasa de clonación}

\textbf{Prueba}: \blue{prueba4} \\

\textbf{Parámetros involucrados}: Tasa de clonación y Factor de clonación. \\

\textbf{Objetivo}: Estudiar el efecto de la tasa y el factor de clonación en conjunto de acuerdo a fitness obtenido para la variación
de los dos parámetros en todo tu dominio.\\

\textbf{Metodología}: Se prueban varias combinaciones de valores para ver el efecto de los parámetros y poder observar su comportamienteo.\\

\textbf{Configuración escogida:}\\

\begin{small}
\begin{center}
\begin{tabular}{|l|c|c|c|c|c|}
	\hline
	\textbf{Instancia} & \textbf{clonationRate} & \textbf{clonationFactor} &\textbf{Mejor resultado} & \textbf{Tiempo [s]} & \textbf{Tiempo total [s]}\\\hline
	\texttt{pb\_200\_01.txt} & 0.6 & 1   & 192 & 1.241 & 132.608 \\\hline
	\texttt{pb\_200\_09.txt} & 0.6 & 1   & 180 & 1.378 & 132.068 \\\hline
	\texttt{pb\_200\_10.txt} & 0.5 & 0.9 & 160 & 1.379 & 132.124 \\\hline
\end{tabular}
\end{center}
\end{small}
\normalsize
\textbf{Gráfico:}\\

\begin{figure}[h!]
\begin{center}
	\includegraphics[width=0.95\textwidth]{img/01-4.pdf}
	\caption{Comparaci\'on de la instancia \texttt{pb\_200\_01.txt} variando \texttt{clonationRate} y \texttt{clonationFactor}}
	\label{fig:4-1}
\end{center}
\end{figure}

\begin{figure}[h!]
\begin{center}
	\includegraphics[width=0.95\textwidth]{img/09-4.pdf}
	\caption{Comparaci\'on de la instancia \texttt{pb\_200\_09.txt} variando \texttt{clonationRate} y \texttt{clonationFactor}}
	\label{fig:4-2}
\end{center}
\end{figure}

\newpage 
\begin{figure}[h!]
\begin{center}
	\includegraphics[width=0.95\textwidth]{img/10-4.pdf}
	\caption{Comparaci\'on de la instancia \texttt{pb\_200\_10.txt} variando \texttt{clonationRate} y \texttt{clonationFactor}}
	\label{fig:4-3}
\end{center}
\end{figure}


\newpage
%
%
%elegir una configuración para cada instancia
%	-indicar el fitness
%	-identificar claramente las características del algoritmo.
%	-estimación del tiempo en realizarla.


\newpage
\subsection{Sintonización Automática}
%Debe identificar el algoritmo sintonizador que elegirá y los parámetros que serán sintonizados por éste. Utilize la siguiente estructura: \\ \\
%\textbf{Sintonizador}: \\
%\textbf{Parámetros a Sintonizar}:\\
%\textbf{Detalles del sintonizador}: Todo lo que tenga que ver con la configuración usada del sintonizador (veáse en la materia los parámetros de cada sintonizador: iteraciones, individuos, tiempo, tipo de busqueda... etc.)\\ 
%
%Debe finalmente entregar el resultado del sintonizador para cada instancia analizada, junto con el resultado (fitness) de la ejecución del algoritmo con estos parámetros.


\section{Conclusiones}
%Conclusiones revelantes del estudio realizado.

En el presente informe se ha dado un estado del arte de un problema muy popular
en el área de la inteligencia artificial, el \emph{Car Sequencing Problem}, siendo éste
una variación de otro problema connotado llamado \emph{Job Shop Scheduling}.
Es tanto la importancia del presente problema, que la \emph{French Society of Operations
Research and Decision-Making Aid} ha decidido ya hace varios años, comenzar lo que se denomina
\emph{The ROADEF challenge} cada dos años, teniendo como objetivo central,  permitir a las personas
que se desarrollan en el área de la industria el presenciar todos los avances y evoluciones
en el ámbito de la Investigación de Operaciones y Análisis de Decisiones, pero no sólo eso
sino el poder enfrentar directamente problemas decisionales complejos, que ocurren en la industria.
Siguiendo la idea anterior, lo importante de éste \emph{Challenge} es que en el 2005, se consideró
como tema principal el \emph{Car Sequencing Problem} debido a la propuesta que realizó RENAULT,
por lo cual uno podrá imaginar la cantidad de avances que se produjeron, pues cada participante
abordaba el problema desde una metodología distinta.

Por otra parte, pareciera que un problema relacionado a \emph{ordenar} un conjunto de vehículos
para ser ensamblados y así obtener el orden más óptimo, no es una tarea difícil, pero claramente
debido a la complejidad que otorgan las restricciones y de que es un problema de la vida real,
presenta un grado de dificultad mayor, lo cual queda reflejado por la cantidad de publicaciones 
e investigaciones que hay al respecto.

Se dieron a conocer también, tres áreas para atacar el presente problema.
Por un lado tenemos los métodos heurísticos que como bien sabemos, es prácticamente jugar a la ruleta
rusa con nuestra investigación, pues la heurística solamente selecciona un objetivo de los dos provenientes
de la definición, una buena solución o un buen tiempo de ejecución. Pero también se presenta que la heurística
es un mecanismo confiable para decidir \emph{utilizarlo} como un apoyo, mas que utilizarlo solo.

Siguiendo con los mecanismos planteados, se vieron también los  métodos exactos,
es decir, técnicas de optimización, donde podemos encontrar la \emph{programación lineal entera},
\emph{branch and bound} y \emph{local search}, los cuales se dedicaban netamente a construir una
solución óptima a partir de los datos que el mismo problema nos entrega. El único problema que tienen
éstas técnicas es que la complejidad temporal va a crecer demasiado con respecto al tamaño de nuestro
\emph{input} del algoritmo.

Dentro de toda la lectura realizada para las distintas técnicas, pude percatarme que las mejores soluciones
siempre son variaciones de métodos o tomar dos técnicas como complementarias, por ejemplo uno de los
mejores resultados fue la combinación de un \emph{Ant Colony Optimization} con una heurística dinámica,
pues claramente se nos señala que el buen uso de una heurística es crucial, es decir, hay que preocuparse
de leer los estudios que se han publicado, par ver cual es la combinación más óptima.

Finalmente, es impresionante la cantidad de estudios con respecto a éste problema en particular,
por lo que podemos darnos cuenta que muchos centros de investigación han dedicado tiempo valioso
para la resolución óptima del \emph{Car Sequencing Problem}, pero no tanto la versión que se estudió,
que es la propuesta por Parello~\cite{parello}, sino mas bien al desafío de la ROADEF.


\section{Bibliografía}
\bibliography{sintonizacion}.

\newpage
\section{Anexos}
\subsection{Anexo I}
\label{sec:anexo1}
\lstset{
  literate={á}{{\'a}}1
           {é}{{\'e}}1
           {í}{{\'i}}1
           {ó}{{\'o}}1
           {ú}{{\'u}}1
}
\begin{small}
	\lstinputlisting{scripts/pregunta-2.py}
\end{small}

\newpage
\subsection{Anexo II}
\label{sec:anexo2}
\begin{small}
	\lstinputlisting{scripts/pregunta-3.py}
\end{small}

\newpage
\subsection{Anexo III}
\label{sec:anexo3}
\begin{small}
	\lstinputlisting{src/4.1-alineamiento.txt}
\end{small}

\newpage
\subsection{Anexo IV}
\label{sec:anexo4}
\begin{small}
	\lstinputlisting{src/4.2-alineamiento.txt}
\end{small}

\newpage
\subsection{Anexo V}
\label{sec:anexo5}
\includegraphics[angle=270,scale=0.55]{img/hmm}


\end{document} 
