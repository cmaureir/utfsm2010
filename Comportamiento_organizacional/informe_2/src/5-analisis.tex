% Análisis de la Aplicación de la Entrevista. (30%)
% Este análisis constituye la parte esencial del informe. Debe incluir:
%  Citar principios de la Entrevista Apreciativa aplicados u
% observados.
%  Se deben analizar las encuestas realizadas citando a los
% entrevistados para apoyar el análisis. También debe incorporar las
% observaciones que haya hecho in situ.
%  Explicitar los elementos descubiertos a través de la actividad
% (entorno de la organización, recursos, funcionamiento, etc.).


La mayoría de los entrevistados sentían que el mejor día de la semana era
cuando salían del trabajo, y podían realizar otras actividades recreativas o
con su familia.

Todavía se sentía la pérdida de un compañero de trabajo que habían despedido
hace 3 años atrás.

Con respecto al entorno de trabajo, pudimos percatarnos que existía un marcado
exceso en las horas de trabajo, lo que provocaba malestar e indiferencia en los trabajadores,
más lo que podría ser una presión social por parte de la dirección a que esta se
mantuviera (sin pagar las horas extras).

Por lo general, no existían problemas con los pagos de las cotizaciones, y todos los trámites
estaban al día, además no hubo ninguna queja con que faltara algún implemento para el trabajo
ya que para el almuerzo, se tiene un refrigerador, microondas, cada taller tiene una radio e
incluso algunos tienen un televisor, por lo que el ambiente es grato con respecto a elementos
tecnológicos que puedan necesitar.

Algunas actividades (salidas de fin de año, actividades en conjunto con todos
los empleados) que se realizaban en la empresa, se dejaron de desarrollar
hace unos 2 o 3 años atrás, y a todos les gustaría volver a ellas, o realizar
actividades nuevas que integren a todos como equipo.

Se ve al Jefe como a una figura paterna. Unas veces preocupado por sus
empleados, y otras veces como, citando a un entrevistado "dictador", lo cual
significa dos cosas, quizás problemas personales entre el Jefe y el entrevistado,

Algunos entrevistados se referían "al grupo en general", y utilizaban frases como
"son preocupados" pero otros se referían diciendo "cada uno mata su chancho",
lo que a pesar de no ser tantos trabajadores, nos muestra que tenemos los dos lados
de la moneda, personas que tienen un sentimiento negativo con el jefe por los malos
tratos y otros que dicen estar felices, aunque pueden ser de las personas que solo
bajan la cabeza y aceptan lo que el jefe diga, independiente que sea justo o no.

Una pregunta realizada, que se relaciona directamente con uno de los principios de la
entrevista apreciativa\cite{apreciativa} que nos dice "lo que se hace bien se puede hacer mejor",
descolocaba a las personas y pocos entrevistados pudieron realmente responderla,
ya que cuando les decíamos que si pudieran mejorar algún aspecto positivo, que harían,
nos respondían de inmediato que no podían y nosotros insistíamos que era un caso
hipotético y se desviaban del tema o nos señalaban que sería bueno agregar algún otro
elemento, pero las pocas personas que nos referíamos anteriormente, señalaron que 
quizás el ambiente sería mucho mejor con más actividades que unieran al grupo de trabajo.

Otra de las características importantes en la organización es que lamentablemente absolutamente
todos los trabajadores tienen la idea de que sin el Jefe, la empresa no va a funcionar, lo cual
es un aspecto negativo pues el Jefe está viejo y de a poco sus hijos han ido tomando el control
de la organización, a lo que los trabajadores no se contentan del todo.

Otro principio que se presenta en la entrevista, es el principio "principio anticipatorio", el cual 
esencialmente afirma que las imágenes del futuro pueden afecta la manera como nos comportamos en el presente. 
En particular, si tenemos una imagen particularmente deseable del futuro es muy probable que nos conduzcamos 
de maneras que lo cristalizarán. Es exactamente lo que se ve en algunos entrevistados, los cuales se veían
mas fuera de la empresa que dentro, por lo que su forma de describir su estado actual dentro de élla, demostraba
en gran medida lo que deseaban para su futuro.

Finalmente, la organización funciona sin problemas, aunque varios trabajadores les gustaría un poco
más de motivación, aumentos de sueldos o por último que se cumplieran los horarios.

% escribir mas
