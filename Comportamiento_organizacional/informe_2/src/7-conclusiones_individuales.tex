%Cada alumno se debe identificar y dar una conclusión personal acerca de los
%temas tratados.

\begin{itemize}
	\item Rodrigo Fernández\\ \\
        Es bastante complejo el lograr reacciones positivas a diferentes
        personas bajo un sólo cuestionario en común, ya que las experiencias de
        diferentes grupos de personas condicionan de diferente forma la
        emotividad de cada una. Aún así, considero que la encuesta fue exitosa, ya
        que nos permitió identificar claramente la situación actual de la
        empresa, los puntos positivos que la hacen mantenerse con éxito, y las
        problemáticas que ha tenido a través del tiempo y que se pueden mejorar.
        Tomando en consideración la actividad realizada anteriormente con la
        empresa, este tipo de entrevista logró identificar y sacar a relucir a
        más detalle el estado actual de la organización, junto con las
        fortalezas, debilidades y posibles amenazas que pudieran afectarle en
        un futuro.
        Por último, el tema de las relaciones interpersonales dentro de los
        grupos de trabajo se logró destacar como un factor importante a la
        hora de evaluar la satisfacción de los trabajadores con su trabajo, y
        es posible que, como dicen algunos estudios, esté intrínsecamente
        relacionado con el desempeño laboral.
	\item Cristián Maureira \\ \\
		Al momento de redactar la entrevista, fue más difícil de lo habitual
	poder escribir las preguntas, porque realmente nos dimos cuenta que por lo
	general no pensamos "apreciativamente".
	Con respecto a los entrevistados, la experiencia fue muy similar a la que
	ya habíamos realizado en el ramo anterior, pero me pareció muy llamativo
	que a las personas les costaba mucho o simplemente no podían responder
	algunas preguntas, ya que no entendían lo que había que responder, como
	por ejemplo ¿Qué sientes al trabajar acá?, nos preguntaban a que nos
	referíamos y que diéramos un ejemplo, quizás las personas no están acostumbradas
	a responder preguntas de como se sienten o que consten en buscar entre sus
	propios logros.
	Finalmente comprobé lo que el profesor decía en clases acerca de que depende
	de la forma en que uno pregunta las cosas, puede interferir en lo que se responde,
	por ejemplo, malamente en varias ocasiones, preguntaba, ¿Pero usted se siente cómoda
	con su grupo de trabajo, no es cierto? y me decían "sí, por supuesto".

	\item Gabriel Zamora \\ \\
		La forma en que influyó el método apreciativo que tratamos de ejercer
	sobre los entrevistados, en cierta medida resulto bastante bien, ya que 
	algunos pocos se sentían bastante bien trabajando en aquel lugar. Sin
	embargo, una gran mayoría estaba predispuesta a responder en forma de
	demostrar su malestar sobre ciertas condiciones laborales, incluso 
	demostrándolo con sus gestos, lo que provocó que sus respuestas fueran
	bastante poco explicativas y de aspecto negativo, logrando finalmente
	que la entrevista fuera casi una queja a un tercero, para que se le 
	hiciera llegar a su jefe. Creo que a pesar de ésto último, se logró
	sacar ciertas cosas positivas en las respuestas y entre los pocos que 
	respondían positivamente, logramos sacar ideas escondidas que un cierto
	grupo poseía, como el hecho de independizarse o buscar un trabajo que le
	ofrezca mejores condiciones.


\end{itemize}

