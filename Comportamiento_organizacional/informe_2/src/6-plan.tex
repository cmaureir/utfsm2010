%Diseño de un plan de acción con propuestas para el futuro de la empresa, de
%acuerdo a lo analizado.

De acuerdo a lo analizado en los puntos anteriores, claramente se ve una 
cierta disconformidad en las personas que se desenvuelven en esta empresa,
por lo que hay que tener presente que en el normal funcionamiento, es 
fundamental establecer un ambiente acorde a ello, ya sea físico como
psicológico. Para poder establecer dicho objetivo, planteamos lo siguiente:

% Esto está demasiado VERDE!!!! agreguen o modifiquen cosas (gzamora)

% De administracion general:
%1.-PREVISION
%2.-PLANEACION
%3.-ORGANIZACION
%4.-INTEGRACION
%5.-DIRECCION
%6.-CONTROL

\begin{itemize}
	\item Prever problemas. \\
	Se debe claramente prever los problemas que actualmente se han
	detectado dentro de la empresa. Para dicho propósito, se debe detectar las causas
	que provocan dichos problemas, con tal de mitigar aquello.
	Sería muy positivo, poder constar con algún mecanismo donde se puedan conversar
	todas las situaciones que ocurren en la empresa, como por ejemplo
	reuniones cada 2 semanas (se explica más adelante) donde todos puedan
	tocar algún tema determinado y aclarar todo, de ésta forma se pueden prever
	futuros conflictos. 

	\item Planear soluciones. \\
	La planeación de las soluciones va de la mano con actividades en las cuales todos puedan dar
	a conocer su opinión con respecto a una temática en particular, es necesario que la dirección
	de la organización no haga oídos sordos a éstas inquietudes, para poder mejorar el ambiente
	y el bienestar de cada miembro de la organización.
	Se tiene muy claro que no se le puede dar \emph{todo} lo que los trabajadores quieran, pero es
	necesario conversar las cosas y llegar a un acuerdo.

    \item Mejorar la organización.\\
	Actualmente existe una clara preferencia a las gestiones del Jefe, y se le ve como un
	elemento fundamental para que la organización funcione, hay que buscar con urgencia que
	no exista tal dependencia, puesto que la estadía del Jefe no es eterna, y por lo apreciado
	en las entrevistas, va a ser muy difícil el día que no esté para la organización seguir
	en igual funcionamiento como lo ha hecho hasta el momento. Es necesario más integrar y hacer
	que se simpatice con los futuros encargados, como en éste caso son los hijos.

	\item Actividades de integración.\\
	Variadas personas se referían a los momentos en que todos estaban unidos, como los buenos
	momentos de la organización, si bien es cierto algunas personas realmente no disfrutan de estar
	en grupo, es válido crear instancias más constantes, como por ejemplo, si se le van a celebrar los cumpleaños
	a los trabajadores, que sean a \emph{todos} y no sólo cuando el jefe dé la
    autorización\cite{integracion}.
	De la misma forma, sería recomendable retomar los paseos de fin de año para que todos los trabajadores se
	logren unir más.
	Finalmente, varios trabajadores se veían muy interesados en poder participar de charlas sobre distintas
	temáticas de actualidad para poder nutrirse de conocimiento y estar al día con respecto a un tema en particular,
	ya que nos plantearon las ganas de aprender cosas nuevas, no necesariamente relacionadas con la joyería.
	
	\item Aspectos de la dirección de la organización. \\
	La dirección de la organización se ha realizado de buena forma, pues ha tenido un crecimiento sustancial desde
	su comienzo, por lo que se percibe el esfuerzo detrás, pero hay que tener en cuenta que se deben cumplir
	las leyes del trabajo al pie de la letra, si bien es cierto hay mucho trabajo algunos días, los trabajadores
	están en todo su derecho para poder trabajar las horas que les corresponde, y si llegan trabajar horas extras
	estas deben ser remuneradas.
	
	\item Control de cada elemento de la organización. \\
	Actualmente no existe un control de cada miembro de la organización, aparte de que el Jefe algunas veces
	se acerca a las personas a preguntar como están.
	Sería muy recomendable que existiera una persona dedicada a analizar el comportamiento de cada miembro,
	quizás un especialista como un asesor externo, para poder estudiar a cada persona con lo cual se puede
	justificar quizás un bajo desempeño y al buscar la solución, poder tener a un personal trabajando cómodamente,
	sabiendo que se preocupan de sus problemas y con lo mismo la productividad de la organización se verá en aumento.

	% ESTO NO ES DE LA LISTA ORIGINAL PROPUESTA POR gzamora.

    \item Mejorar sistema de incentivos.\\
    Actualmente, los trabajadores no se sienten motivados~\cite{motivacion} a innovar y ser más
    creativos en su trabajo~\cite{creatividad}. Esto puede que se vea como algo positivo en
    cuanto se siguen las reglas impuestas al pie de la letra, pero se
    transforma en una fuente de disconformidad laboral al hacer más monótona
    la actividad diaria. Se recomienda tomar en consideración este punto, y
    generar un sistema de incentivos bien definido y que de verdad sienta en
    las personas que su trabajo y creatividad personal es reconocido.\\

    \item Plan de crecimiento personal.\\
    Estando trabajando muchos años en una misma empresa, logra especializar a
    las personas en lo que hace, lo cual es bueno en un comienzo, pero que
    hace que la persona deje de aprender cosas nuevas y sienta la inquietud de
    tener nuevas experiencias y aprender cosas nuevas. La idea es mantener el
    interés de las personas por su trabajo\cite{interes}, dándoles la oportunidad de poder
    perfeccionarse y/o aprender cosas nuevas complementarias a su área de
    trabajo. Ésto se puede realizar a través de pequeños cursos,
    capacitaciones o charlas libres para ciertos empleados de la empresa.

    \item Reuniones periódicas.\\
    Se recomienda realizar reuniones que den la libertad de
    discutir temas concernientes a las dinámicas internas de la empresa,
    problemas entre los miembros del personal o disconformidades varias que
    puedan tener, en donde se busque proponer los problemas y la solución a
    ellos, discutiendo todos de par-a-par dejando de lado por un momento los
    rangos de cada uno. Para ello se recomienda utilizar dinámicas de grupo
    focal (focus group~\cite{Grupo_focal}), con un moderador imparcial que
    controle que las conversaciones no se vallan por otros temas y den un
    orden a la actividad.

\end{itemize}
