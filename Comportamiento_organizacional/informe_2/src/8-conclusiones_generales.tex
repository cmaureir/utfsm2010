%Conclusión elaborada por todos los miembros del equipo en conjunto.

La experiencia fue muy enriquecedora al poder comprobar la reacción de las personas
al enfrentarse a una entrevista apreciativa, era fácil pode ver cuando se hacían sonrisas
en sus rostros y no había una tensión que empañara la entrevista, no había tensión al responder,
la gente lo paso muy bien y no fue como la entrevista pasada, donde algunos rostros quedaron
serios.

Algo muy importante que quedó muy en claro en el trabajo, es que muy pocas personas
tienen una autoestima alta que les permita decir "si, soy bueno en lo que hago",
ya que por otro lado hay personas que simplemente no podían decir ningún logro personal.

La organización escogida, no posee tantos trabajadores, por lo que se pudo comprobar el buen ambiente
que existía entre los trabajadores de las distintas áreas, sin dejar de lado algunos conflictos entre
un par de personas, que provocaban indiferencia, pero no un mayor conflicto.
Los espacios para almorzar juntos, por ejemplo, están pero no todas las personas se dan el tiempo
de almorzar en grupo, algunos almuerzan en su mismo lugar de trabajo, lo que hace alejarse más
la posibilidad de tener un gran equipo de trabajo.

Existen muy generalmente dos perfiles de trabajadores en la organización, aquellos que sólo acatan
todo lo que se les dice, y aquellos que están muy pendiente de si son pasados a llevar en sus
derechos, por ejemplo con el tema de las sobre-horas de trabajos, algunos se referían con rabi a la
situación y otros sólo decían que era necesario por la pega.

Además, hay ciertas figuras en la empresa, aparte del Jefe, que son de vital importancia para algunas
personas. Por ejemplo en el taller, existe una mujer que la mayoría ve casi como una madre y le
agradece por toda la preocupación, y por otro lado en las salas de ventas, las vendedoras han construido
una relación más de amistad que sólo de compañeras de trabajo, lo cual es positivo ya que el ambiente
es mucho más cálido al llegar a cualquier local y por lo menos al realizar las entrevistas así lo notamos.

Finalmente, pese a todos los problemas, pensamos que éste grupo de trabajo puede llegar a convertirse
en un equipo con un poco de esfuerzo de cada uno y algunas nuevas medidas realizadas por el jefe para
aumentar el bienestar de sus trabajadores.
