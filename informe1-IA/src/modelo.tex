%Uno o más modelos matemáticos para el problema, idealmente indicando el espacio de búsqueda para cada uno.
\subsection{Primer modelo}
El presente modelo, pese a utilizar notaciones matemáticas, trata de dar un enfoque más teórico,
para una mejor comprensión del problema.

Consideremos el \emph{Car Sequencing Problem} como una tupla $(C,O,p,q,r)$, tal que:
\begin{itemize}
	\item $C=\{c_1,\ldots,c_n\}$ es el conjunto de vehículos a producir.
	\item $O=\{o_1,\ldots,o_m\}$ es el conjunto de distintas opciones.
	\item $p:O\rightarrow\mathbb{N}$ and $q:O\rightarrow\mathbb{N}$ definen la capacidad de las restricciones,
		es decir, para cualquier opción $o_i \in O$, cada subsecuencia de $q_i$ vehículos consecutivos en la línea,
		deben no contener más que $p_i$ vehículos que requieran la opción $o_i$
	\item $r:C\times O \rightarrow \{0,1\}$ define los requerimientos de las opciones, es decir, para cada vehículo
		$c_i \in C$ y para cada opción $o_j \in O$, $r_{ij} = 1$ si $o_j$ debe ser instalado en $c_i$ y $r_{ij} = 0$
		en otro caso.
\end{itemize}

%Solving a car sequencing problem involves finding an arrangement of the cars
%in a sequence, defining the order in which they will pass along the assembly
%line, such that the capacity constraints are met. We shall use the following
%notations to denote and manipulate sequences:

\begin{itemize}
	\item Una \emph{secuencia}, definida $\pi =< c_{i_{1}}, c_{i_{2}}, \ldots, c_{i_{k}} >$, es una sucesión de vehículos.
	\item El conjunto de todas las secuencias que pueden ser construidas con un conjuntos de autos $C$ es denominada $\amalg_{C}$.
	\item El \emph{largo} de una secuencia $\pi$, denominado $|\pi|$, es el numero de vehículos que contiene.
	\item La \emph{concatenación} de dos secuencias, $\pi_1$ y $\pi_2$, denominada $\pi_1 \cdot \pi_2$, es una secuencia
		compuesta de los vehículos de $\pi_1$ seguidos de los vehículos de $\pi_2$.
	\item Una secuencia $\pi_1$ es una subsecuencia de otra secuencia $\pi_2$ , tomando en cuenta $\pi_1 \subseteq \pi_2$,
		si existen dos secuencias $\pi_3$ y $\pi_4$, tal que $\pi_2 = \pi_3 \cdot \pi_1 \cdot \pi_4$
	\item El \emph{cost} de una secuencia $\pi$ es el numero de las restricciones de capacidad que no se cumplen, es decir:
\end{itemize}

$$costo(\pi) = \sum_{o_{i}\in O} \ \ \ \sum_{\pi_{k}\subseteq \pi\\ \rightarrow |\pi_{k}|=q_i} violaci\acute{o}n(\pi_{k},o_{i})$$
$$\text{donde}\ violaci\acute{o}n(\pi_{k},o_{i}) = \left\{
 \begin{array}{l}
	0\ si\ \sum\limits_{<C_l>\subseteq\pi_k}r_{li}\leq p_i;\\
	1\ \text{en otro caso}\\
 \end{array} \right.$$


Por lo tanto se puede resolver ahora el ejercicio buscando la secuencia con el \textbf{mínimo costo}, compuesta por todos los autos
que serán producidos.

%

Respecto al espacio de búsqueda, si tenemos que $\alpha_o$ es la cantidad de vehículos que tienen que ser producidos
por cada clase/configuración o, con $o\in O$ y sea $n$ la cantidad total de vehículos a producir,
entonces el espacio de búsqueda está compuesto por todas las posibles permutaciones de los vehículos en $C$ que
requieren diferentes configuraciones, particionado en los $o$ tipos distintos tipos de vehículos a construir.

Por lo tanto el tamaño de las posibles soluciones sería:
$$E.B\ =\ \frac{n!}{\prod\limits_{o \in O}\alpha_o}$$
 
Claramente tenemos que darnos cuenta que varias soluciones dentro del presente espacio de búsqueda
no van a ser posibles, pues debemos aplicar las restricciones.

\subsection{Segundo modelo}
\textbf{Programación Lineal Entera (ILP)}

Definición de \emph{variables} y \emph{parámetros}:
\begin{itemize}
	\item \textbf{$opt$}	    Número total de opciones.
	\item \textbf{$cl$}	    Número total de tipos/clases de vehículos.
	\item \textbf{$n_i$}    Número de vehículos en la clase/tipo $i$.
	\item \textbf{$nc$}	    Número total de vehículos.
	\item \textbf{$o_{ik}$}	Parámetro \emph{booleano} que representa si los vehículos del tipo/clase $i$ requieren la opción $k$.
	\item \textbf{$s_k$}    Largo de una secuencia de vehículos consecutiva, donde algunas requieren la opción $k$
	\item \textbf{$r_k$}    Número máximo de vehículos que pueden tener la opción $k$ en una secuencia consecutiva de $s_k$ vehículos.
	\item \textbf{$M$}	 	Valor escalar grande que puede ser fijado a $(s_{k} - r_{k})$
	\item \textbf{$C_{ij}$}	Variable \emph{booleana} que representa si un vehículo del tipo/clase $i$ es asignado a la posición $j$ en la secuencia
		$(C_{ij} = 1)$
	\item \textbf{$Y_{kj}$}	Varibale \emph{booleana} que representa si el número de vehículos que requieren la opción $k$ en una secuencia
		de $s_k$ vehículos, comenzando de la posición $j$ en la secuencia respecto a $r_k$, el máximo permitido $(Y_{kj} = 0)$
	\item \textbf{$pp$}	    Número de vehículos en el periodo anterior, que permiten establecer la transición $(pp = max(s_{k}-1))$
	\item \textbf{$cpp_{mk}$}	Variable \emph{booleana} que representa si el vehículo $m$-ésimo del periodo previo tiene la opción $k$.
	\item \textbf{$Z_{kj}$}	Variable \emph{booleana} que representa si el número de vehículos que necesitan la opción $k$ en una secuencia
		de $s_k$ vehículos, partiendo en la posición $j$ de la secuencia en el periodo anterior, respetando el máximo $r_k$ permitido $(Z_{kj} = 0)$.
\end{itemize}

Por lo tanto, el modelo a continuación se establece para minimizar el número de violaciones de las restricciones de capacidad,
tanto como para opciones de alta prioridad, como de baja.

\textbf{Función objetivo:}
$$\text{Min} F = \sum\limits_{k=1}^{opt} \sum\limits_{j=1}^{nc-s_{k}+1} Y_{kj} + \sum\limits_{k=1}^{opt} \sum\limits_{j=1}^{s_{k}-11} Z_{kj}$$

\textbf{Restricciones}
\begin{itemize}
	\item $\sum\limits_{i=1}^{cl} C_{ij} = 1$, $\forall j = 1,\ldots,nc$
	\item $\sum\limits_{j=1}^{nc} C_{ij} = n_{i}$, $\forall i = 1,\ldots,cl$
	\item $\sum\limits_{i=1}^{cl} \sum\limits_{l=j}^{j+s_{k}-1} o_{ik} \cdot
		C_{il} \leq r_{k} + M\cdot Y_{kj}$,\\ $\forall k = 1,\ldots,opt$ y $\forall j=1,\ldots, nc-s_{k}+1$
	\item $\sum\limits_{i=1}^{cl} \sum\limits_{l=1}^{j} o_{ik} \cdot
		C_{il} + \sum\limits_{m=pp-S_{k}+j+1}^{pp} cpp_{mk} \leq r_{k} + M\cdot Z_{kj}$,\\ $\forall k = 1,\ldots,opt$ y $\forall j=1,\ldots, s_{k}-11$
\end{itemize}

Respecto al espacio de búsqueda, como ésta formulación es ILP igual que el primero modelamiento, sólo sería la permutación de todos
los vehículos que se necesitan,  particionados por las clases de vehículos que se producirán. Obviamente posicionándonos en el peor de los casos
donde tenemos que recorrer todo el espacio de búsqueda.

$$E.B\ =\ \frac{nc!}{\prod\limits_{i = 0}^{cl}\alpha_{n_i}}$$

Por ejemplo, si tuvieramos $30$ vehículos a ensamblar, $4$ clases de autos y
necesitamos $10$ autos de clase $1$, $5$ de clase $2$, $7$ de clase $3$ y $8$ de clase 4.

$$E.B\ =\ \frac{30!}{10\cdot 5\cdot 7\cdot 8}\ =\ 9.473316\times 10^{28}$$


