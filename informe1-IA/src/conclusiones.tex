%Conclusiones revelantes del estudio realizado.

En el presente informe se ha dado un estado del arte de un problema muy popular
en el área de la inteligencia artificial, el \emph{Car Sequencing Problem}, siendo éste
una variación de otro problema connotado llamado \emph{Job Shop Scheduling}.
Es tanto la importancia del presente problema, que la \emph{French Society of Operations
Research and Decision-Making Aid} ha decidido ya hace varios años, comenzar lo que se denomina
\emph{The ROADEF challenge} cada dos años, teniendo como objetivo central,  permitir a las personas
que se desarrollan en el área de la industria el presenciar todos los avances y evoluciones
en el ámbito de la Investigación de Operaciones y Análisis de Decisiones, pero no sólo eso
sino el poder enfrentar directamente problemas decisionales complejos, que ocurren en la industria.
Siguiendo la idea anterior, lo importante de éste \emph{Challenge} es que en el 2005, se consideró
como tema principal el \emph{Car Sequencing Problem} debido a la propuesta que realizó RENAULT,
por lo cual uno podrá imaginar la cantidad de avances que se produjeron, pues cada participante
abordaba el problema desde una metodología distinta.

Por otra parte, pareciera que un problema relacionado a \emph{ordenar} un conjunto de vehículos
para ser ensamblados y así obtener el orden más óptimo, no es una tarea difícil, pero claramente
debido a la complejidad que otorgan las restricciones y de que es un problema de la vida real,
presenta un grado de dificultad mayor, lo cual queda reflejado por la cantidad de publicaciones 
e investigaciones que hay al respecto.

Se dieron a conocer también, tres áreas para atacar el presente problema.
Por un lado tenemos los métodos heurísticos que como bien sabemos, es prácticamente jugar a la ruleta
rusa con nuestra investigación, pues la heurística solamente selecciona un objetivo de los dos provenientes
de la definición, una buena solución o un buen tiempo de ejecución. Pero también se presenta que la heurística
es un mecanismo confiable para decidir \emph{utilizarlo} como un apoyo, mas que utilizarlo solo.

Siguiendo con los mecanismos planteados, se vieron también los  métodos exactos,
es decir, técnicas de optimización, donde podemos encontrar la \emph{programación lineal entera},
\emph{branch and bound} y \emph{local search}, los cuales se dedicaban netamente a construir una
solución óptima a partir de los datos que el mismo problema nos entrega. El único problema que tienen
éstas técnicas es que la complejidad temporal va a crecer demasiado con respecto al tamaño de nuestro
\emph{input} del algoritmo.

Dentro de toda la lectura realizada para las distintas técnicas, pude percatarme que las mejores soluciones
siempre son variaciones de métodos o tomar dos técnicas como complementarias, por ejemplo uno de los
mejores resultados fue la combinación de un \emph{Ant Colony Optimization} con una heurística dinámica,
pues claramente se nos señala que el buen uso de una heurística es crucial, es decir, hay que preocuparse
de leer los estudios que se han publicado, par ver cual es la combinación más óptima.

Finalmente, es impresionante la cantidad de estudios con respecto a éste problema en particular,
por lo que podemos darnos cuenta que muchos centros de investigación han dedicado tiempo valioso
para la resolución óptima del \emph{Car Sequencing Problem}, pero no tanto la versión que se estudió,
que es la propuesta por Parello~\cite{parello}, sino mas bien al desafío de la ROADEF.
