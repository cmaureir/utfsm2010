\documentclass[letter, 10pt]{article}
\usepackage[utf8]{inputenc}
\usepackage[spanish]{babel}
\usepackage{amsfonts}
\usepackage{amsmath}
\usepackage[top=3cm,bottom=3cm,left=3.5cm,right=3.5cm,footskip=1.5cm,headheight=1.5cm,headsep=.5cm,textheight=3cm]{geometry}

\begin{document}
%\pagestyle{empty}


\title{Bioinformática\\``Tarea Genoma y Ensamblado''}
\author{Rodrigo Fernández \and Cristián Maureira \and Gabriel Zamora\\\texttt{\{rfernand,cmaureir,gzamora\}@inf.utfsm.cl}}
\date{\today}
\maketitle

\section{Problema}
Como le explicaría a mi sobrina, que entiende de puzzles y rompecabezas,
la diferencia entre las técnicas \emph{Shotgun Sequencing} y la \emph{Hierarchical
Shotgun Sequencing}.

Debe quedar clara la idea central, objetivos, mecanismos de trabajo,
ventajas y desventajas, pero sobre todo, el por qué en un caso molesta más
la repetición de secuencias.

\vfill\hfill
RF/CM/GZ/\LaTeX
\end{document} 
