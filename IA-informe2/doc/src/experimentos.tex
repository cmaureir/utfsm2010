% Se necesita saber como experimentaron, como definieron par\'ametros,
%  como los fueron modificando, cuales problemas se trataron, instancias,
%  por que ocuparon esos problemas.

Los valores iniciales, eran menores a los señalados al final de la presente sección,
y para poder definirlo se fueron variando por separado.

Primero se fueron variando por separado los parámetros del \emph{algoritmo evolutivo},
donde se llegó a la conclusión, de que si la cantidad de las generaciones va aumentando,
y el tamaño de la población es muy grande, aparte de tardar demasiado en converger,
el desempeño no es bueno.

Luego se estableció la idea de tener poblaciones pequeñas, pero muchas generaciones,
así obtendremos mejores resultados, pero dependerá netamente, si las soluciones iniciales
son buenas, aquí el elitismo juega un papel sumamente importante.

En segundo lugar, las pruebas y modificaciones se realizaron con los parámetros de la
mutación \emph{simulated annealing}, donde se probaron distintos valores, que hicieron
llegar a la conclusión, de que es más apropiado, poseer un número no tan grande de iteraciones,
con respecto a la temperatura, pero la temperatura debe ser a lo más 1 orden de magnitud mayor
que el número de iteraciones.

De la misma forma, la temperatura, no debe tener valores muy elevados, porque no ayuda mucho
en encontrar mejores individuos, sobre todo en éste caso que era un \emph{simulated annealing}
con \emph{alguna mejora}.

Las instancias en las que se ejecutó el algoritmo, fueron los \emph{30 problemas difíciles} de
Caroline Gagne que se encuentran en el sitio de CSPLib~\cite{gagne}, por cada instancia se ejecutó
el presente algoritmo $100$ veces, para obtener una amplia gama de soluciones a distintas situaciones.

Los experimentos fueron realizados en un computador con las siguientes características:
\begin{itemize}
	\item Intel(R) Core(TM)2 Duo CPU 2.66\,[GHz]
	\item 4 Gigabyte RAM
	\item Sistema Operativo Fedora 12 para i686
\end{itemize}

Los parámetros utilizados en los experimentos, no fueron cambiados por cada población,
lo cual puede significar un desempeño inferior al sintonizarlos por en cada caso.

Los valores son:
\begin{itemize}
	\item VARS: 200, 300, 400 (dependiendo de cada caso)
	\item POP: 12
	\item GENS: 5000
	\item PMUT: 0.3
	\item TMAX: 100
	\item IMAX: 10
\end{itemize}
