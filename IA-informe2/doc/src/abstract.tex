%Resumen del informe en no más de 10 líneas.
El presente documento tiene como finalidad presentar el estudio y análisis de un problema clásico, a lo que la Inteligencia Artificial respecta,
el \emph{Car Sequencing Problem}~\cite{parello}, que consiste en la planificación de una fila de vehículos en una línea de ensamblaje, 
mientras se satisfacen todas las restricciones de capacidad de las plantas ensambladores y de los propios vehículos.

La importancia de atacar el presente problema radica en el notorio aumento de la actividad en las fábricas automotrices a lo largo del mundo,
dejando como consecuencia que incluso para el pasado año 2002, ya existían 1 automóvil por cada 10 personas en el mundo~\cite{worldmapper}.

Se plantean además las técnicas más conocidas y utilizadas para poder enfrentar el presente problema, entre las cuales destacan la utilización
de heurísticas y métodos híbridos, dentro de los que son los acercamientos exactos, híbridos y mixtos.

Se presenta un modelamiento simplista utilizando lo que es Programación Lineal, y un planteamiento más teórico para poder comprender a fondo
el problema estudiado.



%problem
%solucion propuesta
%resultados
%keywords
