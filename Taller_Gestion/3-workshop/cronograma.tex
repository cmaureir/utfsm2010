\documentclass[letter, 10pt]{article}
\usepackage[utf8]{inputenc}
\usepackage[spanish]{babel}
\usepackage{amsfonts}
\usepackage{amsmath}
\usepackage[dvips]{graphicx}
\usepackage{url}
\usepackage[top=2cm,bottom=2cm,left=3cm,right=3cm,footskip=1.5cm,headheight=1.5cm,headsep=.5cm,textheight=3cm]{geometry}
\usepackage{listings}
\usepackage{color}
\usepackage{fancyvrb}
\usepackage{fancyhdr}

\pagestyle{fancyplain}

\lhead{Workshop Taller de Gestión} %Parte superior izquierda
\rhead{\bf \it Cronograma} %Parte superior derecha
\lfoot{} %Parte inferior izquierda.
\cfoot{} %Parte inferior central
\rfoot{\bf \thepage} %Parte inferior derecha
\renewcommand{\footrulewidth}{0.4pt} %Linea de separacion inferior


\begin{document}

\title{Cronograma}
\author{
Rodrigo Fernández\\ \small{\texttt{rfernand@inf.utfsm.cl}} \and
Cristián Maureira\\ \small{\texttt{cmaureir@inf.utfsm.cl}} \and
Gabriel Zamora   \\ \small{\texttt{gzamora@inf.utfsm.cl}}
}
%\titlegraphic{\includegraphics[height=1cm]{img/logos}}
\date{\today}

\maketitle

\section{Introducción}
En éste workshop se presentará una metodología de desarrollo de proyectos,
utilizando las herramientas actualmente en uso por distintos proyectos
dentro del Computer Systems Research Group (CSRG).

\section{Cronograma}

\small{
\begin{tabular}{lllll}
\hline
\textbf{Hora} & \textbf{Actividad} & \textbf{Duración} & \textbf{Objetivo} & \textbf{Enbcargado}\\
\hline
3:40pm  & Presentación     & 15 min & Conocimiento de la temática				& Cristián\\
3:55pm  & Tutorial Trac    & 20 min & Funcionalidades y creación de una página  & Rodrigo \\
4:15pm  & Tutorial Planner & 10 min & Creación carta gantt						& Rodrigo \\
4:25pm  & Tutorial Git     & 15 min & Descentralización (comandos básicos)		& Gabriel \\
4:40pm  & Tutorial Latex   & 20 min & Creación documento simple					& Cristián\\
5:00pm  & Tutorial Skype   & 10 min & Llamada conferencia 						& Gabriel \\
\end{tabular}
}
\end{document}
