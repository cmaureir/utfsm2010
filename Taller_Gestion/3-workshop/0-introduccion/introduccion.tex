\documentclass[letter, 10pt]{article}
\usepackage[utf8]{inputenc}
\usepackage[spanish]{babel}
\usepackage{amsfonts}
\usepackage{amsmath}
\usepackage{verbatim}
\usepackage{url}
\usepackage{hyperref}
\usepackage[top=3cm,bottom=3cm,left=3.5cm,right=3.5cm,footskip=1.5cm,headheight=1.5cm,headsep=.5cm,textheight=3cm]{geometry}
\begin{document}

\title{Instructivo General\\Taller de Gestión de Proyectos Informáticos}
\author{Rodrigo Fernández \and Cristián Maureira \and Gabriel Zamora\\ \small{\texttt{\{rfernand,cmaureir,gzamora\}@inf.utfsm.cl}}}
\date{\today}
\maketitle

\section{Objetivo}
Poder configurar adecuadamente las herramientas de trabajo del workshop a
realizar.

\section{Configuraciones}
Para utilizar \textbf{Git} es necesario que en la ubicación \texttt{~/.ssh/},
se descargen los siguientes archivos:
\begin{itemize}
	\item Llave pública y privada: \url{http://csrg.inf.utfsm.cl/~cmaureir/key.zip}
\end{itemize}

Se debe crear un archivo llamado \texttt{~/.ssh/config} que contenga la siguiente información.
\begin{verbatim}
Host csrg
Hostname csrg.inf.utfsm.cl
User git
IdentityFile ~/.ssh/llavegit_taller
\end{verbatim}

\section{Información}
\begin{itemize}
	\item TRAC: \url{http://csrg.inf.utfsm.cl/taller_trac/}
	\item Git: \url{git@csrg:taller_repo}
	\item Usuario: \texttt{taller}
	\item Password: \texttt{taller1234}
\end{itemize}

\end{document} 
