\documentclass[letter, 10pt]{article}
\usepackage[utf8]{inputenc}
\usepackage[spanish]{babel}
\usepackage{amsfonts}
\usepackage{amsmath}
\usepackage[dvips]{graphicx}
\usepackage{url}
\usepackage[top=3cm,bottom=3cm,left=3.5cm,right=3.5cm,footskip=1.5cm,headheight=1.5cm,headsep=.5cm,textheight=3cm]{geometry}
\usepackage{listings}
\usepackage{color}
\usepackage{fancyvrb}
\usepackage{fancyhdr}
\usepackage{verbatim}

\pagestyle{fancyplain}

\lhead{Workshop Taller de Gestión} %Parte superior izquierda
\rhead{\bf \it Tutoriales} %Parte superior derecha
\lfoot{} %Parte inferior izquierda.
\cfoot{} %Parte inferior central
\rfoot{\bf \thepage} %Parte inferior derecha
\renewcommand{\footrulewidth}{0.4pt} %Linea de separacion inferior


\begin{document}

\title{Tutorial Git}
\author{Rodrigo Fernández \and Cristián Maureira \and Gabriel Zamora\\ \small{\texttt{\{rfernand,cmaureir,gzamora\}@inf.utfsm.cl}}}
%\titlegraphic{\includegraphics[height=1cm]{img/logos}}
\date{\today}

\maketitle

\section{Qué es GIT}

GIT es un software de control de versiones (SCM) open source, diseñado por Linus Torvalds para ser utilizado de manera distribuida. Al ser descentralizado, posee
la gran ventaja de poder trabajar sobre copias locales sin depender de un nodo central.

\section{Instrucciones de uso}

Existen una serie de comandos que son los más utilizados para trabajar con git, sin embargo por el tiempo solo se verán los que utilizaremos en el tutorial.

\begin{itemize}
	\item git init: Permite iniciar un repositorio git. No será utilizado en este tutorial.
	\item git clone $<$repositorio$>$: Permite crear una copia local de un repositorio remoto.
	\item git add $<$archivo$>$: Permite agregar bajo el control de git a un archivo.
	\item git commit -sm 'Mensaje': Permite comprometer un cambio en el repositorio.
	\item git pull: Obtiene cambios remotos en la copia local.
	\item git push: Envía cambios locales a repositorio remoto.
\end{itemize}


\section{Ejercicio}
\subsection{Objetivo}

 En 3 equipos, se realizará un pequeño programa en lenguaje ANSI C de manera colaborativa. La estructura del código será:

\begin{itemize}
	\item ssg.c: Archivo principal, debe llamar a la función plot
	\item plot.c: Definición de función plot, debe mostrar un mensaje del tipo: ``Plotting data...''. Incluye a su declaración (.h)
	\item plot.h: Declaración del prototipo de la función plot
	\item Makefile: Archivo makefile
\end{itemize}

\subsection{Desarrollo}

\begin{enumerate}

	\item Clonar repositorio git (git clone)
	\item Desarrollo de código (disponible en la siguiente sección)

		\begin{itemize}
			\item Equipo 1: Desarrollo de ssg.c
			\item Equipo 2: Desarrollo de plot.c y plot.h
			\item Equipo 3: Desarrollo de Makefile
		\end{itemize}

	\item Agregar archivos bajo el control de git (git add)
	\item Comprometer cambios locales (git commit)
	\item Comprobar cambios globales (git pull)
	\item Enviar cambios locales a rama principal (git push)

\end{enumerate}

\subsection{Código fuente}
\subsubsection{ssg.c}
\verbatiminput{../src/git/ssg.c}
\subsubsection{plot.c}
\verbatiminput{../src/git/plot.c}
\subsubsection{plot.h}
\verbatiminput{../src/git/plot.h}
\subsubsection{Makefile}
\verbatiminput{../src/git/Makefile}

\end{document}
