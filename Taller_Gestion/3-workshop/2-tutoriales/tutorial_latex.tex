\documentclass[letter, 10pt]{article}
\usepackage[utf8]{inputenc}
\usepackage[spanish]{babel}
\usepackage{amsfonts}
\usepackage{amsmath}
\usepackage[dvips]{graphicx}
\usepackage{url}
\usepackage[top=3cm,bottom=3cm,left=3.5cm,right=3.5cm,footskip=1.5cm,headheight=1.5cm,headsep=.5cm,textheight=3cm]{geometry}
\usepackage{listings}
\usepackage{color}
\usepackage{fancyvrb}
\usepackage{fancyhdr}

\pagestyle{fancyplain}

\lhead{Workshop Taller de Gestión} %Parte superior izquierda
\rhead{\bf \it Tutoriales} %Parte superior derecha
\lfoot{} %Parte inferior izquierda.
\cfoot{} %Parte inferior central
\rfoot{\bf \thepage} %Parte inferior derecha
\renewcommand{\footrulewidth}{0.4pt} %Linea de separacion inferior

\begin{document}

\title{Tutorial \LaTeX}
\author{Rodrigo Fernández \and Cristián Maureira \and Gabriel Zamora\\ \small{\texttt{\{rfernand,cmaureir,gzamora\}@inf.utfsm.cl}}}
\date{\today}

\maketitle

\section{Qué es Latex}
Sistema de composición de textos, orientado especialmente a la
creación de libros, documentos científicos y técnicos que contengan
fórmulas matemáticas.

\section{Instrucciones de uso}

Para ésta herramienta sólo nos limitaremos a utilizar un archivo \textbf{tex},
y un comando para compilarlo a \textbf{pdf}.
Si nuestro archivo fuera \texttt{ejemplo.tex}, el procedimiento sería:
\begin{verbatim}
	$ pdflatex ejemplo.tex
\end{verbatim}
Donde finalmente podremos obtener un archivo \texttt{ejemplo.pdf}

Explicación comandos dentro del documento:
\begin{itemize}
	\item \texttt{$\backslash$documentclass}: Declaración del documento.
	\item \texttt{$\backslash$usepackage}: Uso de un paquete determinado (como un import, include, etc)
	\item \texttt{$\backslash$title}: Declaración del título del documento.
	\item \texttt{$\backslash$author}: Declaración de el o los autores.
	\item \texttt{$\backslash$section}: Declaración de una nueva sección del documento.
	\item \texttt{$\backslash$subsection}: Declaración de una nueva subsección del documento.
\end{itemize}

\section{Ejercicio}
Generar un informe~\footnote{Archivo principal \texttt{informe.tex}} que contenga:

\begin{itemize}
	\item Descripción Proyecto. (\texttt{src/descripcion.tex})
	\item Requerimientos. (\texttt{src/requerimientos.tex})
	\item Información de cada desarrollador. (\texttt{src/informacion.tex})
\end{itemize}




\end{document}
