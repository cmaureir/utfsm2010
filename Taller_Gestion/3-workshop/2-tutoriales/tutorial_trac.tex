\documentclass[letter, 10pt]{article}
\usepackage[utf8]{inputenc}
\usepackage[spanish]{babel}
\usepackage{amsfonts}
\usepackage{amsmath}
\usepackage[dvips]{graphicx}
\usepackage{url}
\usepackage[top=3cm,bottom=3cm,left=3.5cm,right=3.5cm,footskip=1.5cm,headheight=1.5cm,headsep=.5cm,textheight=3cm]{geometry}
\usepackage{listings}
\usepackage{color}
\usepackage{fancyvrb}
\usepackage{fancyhdr}

\pagestyle{fancyplain}

\lhead{Workshop Taller de Gestión} %Parte superior izquierda
\rhead{\bf \it Tutoriales} %Parte superior derecha
\lfoot{} %Parte inferior izquierda.
\cfoot{} %Parte inferior central
\rfoot{\bf \thepage} %Parte inferior derecha
\renewcommand{\footrulewidth}{0.4pt} %Linea de separacion inferior


\begin{document}

\title{Tutorial Trac}
\author{Rodrigo Fernández \and Cristián Maureira \and Gabriel Zamora\\ \small{\texttt{\{rfernand,cmaureir,gzamora\}@inf.utfsm.cl}}}
%\titlegraphic{\includegraphics[height=1cm]{img/logos}}
\date{\today}

\maketitle

\section{Qué es Trac}
Trac es un manejador de proyectos y bugs.
Permite enlazar información entre una base de datos de errores de software, un sistema de control de versiones y el contenido de un wiki.\\
Está escrito en Python y publicado bajo la licencia BSD modificada.

\subsection{Características}
\begin{itemize}
    \item Administración de proyectos (Roadmap, Milestones, etc.).
    \item Sistema de tickets (bug tracking, tareas, etc.).
    \item Administración fina de permisos de acceso (desde la versión 0.11).
    \item Linea de tiempo para toda la actividad reciente.
    \item Wiki.
    \item Reporte customizado.
    \item Interfaz de web VCS.
    \item RSS Feeds.
    \item Soporte de múltiples proyectos.
    \item Aumento de funcionalidades por extensiones o plugins.
    \item Formato iCalendar.
    \item Soporte de múltiples repositorios por ambiente (desde la 0.12)
\end{itemize}


\section{Instrucciones de uso}
Crear una página wiki:
  \begin{itemize}
	\item Editamos una página y escribimos una ``WikiWord''. Al guardar la página, la WikiWord sera presentada como un link a una página nueva y, siguiendo este link, podrás pasar a editar la nueva página.
	\item Otra forma de crear una página en la wiki es accediendo a ella a
    través del navegador,por ejemplo\\
    \url{http://csrg.inf.utfsm.cl/taller_trac/wiki/newpage}\\ y presionar en
    ``Edit This page link'', o explicitando específicamente el nombre que
    quieres que tenga el link visualizado en la página wiki:
		\begin{center}\textbf{\lbrack wiki:NombreDePagina Nombre de Link\rbrack}\end{center}
	\item Para evitar que Trac reconozca una WikiWord, se le antepone ! a la palabra.
  \end{itemize}


\section{Ejercicio}
\begin{description}
  \item[Creación de las Fichas de usuarios]:
  \begin{itemize}
	\item Creamos una página con la información personal de uno:\\
    \url{http://csrg.inf.utfsm.cl/taller_trac/wiki/NombreApellido/}
	\item Dentro de ella, llenamos nuestros datos utilizando la siguiente
    plantilla:\\
\begin{verbatim}
== Nombre Apellido ==

|| '''Nombre:''' || Nombre Apellido ||
||'''Foto:'''|| [[Image(nombreimagen.png)]] ||
|| '''Cumpleaños:''' || fecha de tu cumpleaños ||
|| '''Celular:''' ||  ||
|| '''Telefono:''' ||  ||
||'''MSN:'''|| ||
||'''SkypeID:'''||  ||
||'''YahooID:'''||  ||
||'''Dirección:'''|| ||
||'''GMail:'''|| ||
|| '''Mail:''' ||  ||
\end{verbatim}
	\item Crear un link a su página creada dentro de la página principal, de
    la siguiente forma:
		\begin{center}\textbf{\ \ \ *\ \lbrack wiki:NombreApellido Nombre Apellido\rbrack}\end{center}

  \end{itemize}

  \item[Asignación de tareas]:
\begin{enumerate}
    \item Cree 2 tareas cualquiera y asigne las a alguno de sus compañeros
    (anotando el usuario de los responsables, separados por comas).
    \item Pruebe diferentes atributos que puede asignarle a una tarea.
\end{enumerate}

\end{description}
\end{document}
