\documentclass[letter, 10pt]{article}
\usepackage[utf8]{inputenc}
\usepackage[spanish]{babel}
\usepackage{amsfonts}
\usepackage{amsmath}
\usepackage{url}
\usepackage[top=3cm,bottom=3cm,left=3.5cm,right=3.5cm,footskip=1.5cm,headheight=1.5cm,headsep=.5cm,textheight=3cm]{geometry}
\begin{document}

\title{Sampling System GUI (SSG)}
\author{ALMA-UTFSM Group\\\texttt{alma@inf.utfsm.cl}}
\date{\today}
\maketitle

\section{Description}

\subsection{About the Sampling System}
The ACS Sampling System is a collection of objects designed to easily sample an
ACS Component's Property value over time. For more information about ACS
Components and Properties, please refer to Java references.

The Sampling System is composed by two kind of entities: \textbf{Sampling\ Objects}
and \textbf{Sampling\ Managers}. A Sampling Object is a distributed object which
represents and manages one, and only one, sample. A Sampling Object is composed
by the following elements:

\begin{itemize}
    \item The ACS \textbf{Characteristic} Component containing the desired property.
    \item The Property to be sampled on the given Component.
    \item A period of time between sampling intervals.
\end{itemize}

Sampling Objects manage the sampling of the given property. They communicate
with the property by acquiring a direct CORBA reference of it, and sample it
with the given frequency. The acquired values are then sent through a
Notification Channel (for details refer to Notification Channel).
Sampling Objects allow to start/pause/resume/stop the sampling process by
making remote calls over them.

The Sampling Manager is an ACS Component that offers the functionality to create
Sampling Objects. This functionality, then, can be accessed by other ACS
Components, or by ACS Clients. SSG is an example of such clients.

\subsection{About SSG}

SSG is a client to the Sampling System. It communicates with one or several
Sampling Managers to create one or several Sampling Objects, group them as
needed, in order to sample one or several properties, and finally plot the
values in user-time (i.e., as they arrive through the Notification Channel).
Therefore, it allows an easy, quick visualization of a system's behavior during
a period of time, or under certain circumstances, and gives the possibility of
visually correlate the values of different properties of the system.

\section{Requeriments}
\begin{tabular}{|l|p{9.5cm}|}
\hline
\textbf{Phase} & \textbf{Item} \\\hline
Phase1& Initial implementation\\\hline
 	  & Implement a GUI\\\hline
 	  & Select sampling period\\\hline
 	  & Select sampling frequency\\\hline
 	  & Dump sampled data to file\\\hline
 	  & Select the sampling manager\\\hline
 	  & Save sampling status\\\hline
 	  & Improve read components performance\\\hline
 	  & Improve sampling thread performance\\\hline
Phase 2& Provide plotting features\\\hline
 	   & Provide on-line plotting\\\hline
 	   & Plotting of several properties at the same time.\\\hline
 	   & Select time window or volume of data to plotted.\\\hline
 	   & Plotting has to be allowed at different frequencies.\\\hline
 	   & Change frequency dynamically.\\\hline
 	   & Plotting of two properties in the same plot.\\\hline
 	   & Automatically adjust the plotting range.\\\hline
 	   & Provide the capability to store data into a file.\\\hline
 	   & Show the actual state of the system.\\\hline
Phase 3.1& Mission at the OSF.\\\hline
 	     & Replace jfreechart with jchart2d.\\\hline
 	     & Maximun trending window of 15 minutes.\\\hline
 	     & Make saving to a file optional, but on by default.\\\hline
 	     & Fix/change time windows.\\\hline
 	     & Script executor: Start along with the sampling a specified script, and when it returns (ends), finish the sampling.\\\hline
 	     & Save the sampling configuration (complete).\\\hline
 	     & Less is more (keep the UI clean and lean).\\\hline
Phase 3.2& Nice to have features.\\\hline
 	     & Allow (de)activation of a specific trend (continue with samples, but does not present the whole trend).\\\hline
 	     & Zoom in/out, reset zoom.\\\hline
 	     & Pause: Allow to stop all trending, but not the data capture. Just to focus on details for a moment (or to use the crosshair function).\\\hline
Phase 4& Documentation and Bugs.\\\hline
 	   & Bug fixes.\\\hline
 	   & Documentation with ALMA standars.\\\hline
\end{tabular}

\section{Links}

\begin{itemize}
	\item \url{https://csrg.inf.utfsm.cl/twiki4/bin/view/ACS/SamplingSystem}
	\item \url{https://csrg.inf.utfsm.cl/twiki4/bin/view/ACS/SamplingSystemRequirements}
	\item \url{http://csrg.inf.utfsm.cl/ssg/api}
	\item \url{http://csrg.inf.utfsm.cl/ssg/doc/SamplingSystemGUI-Documentation.pdf}
\end{itemize}

\end{document} 
