\frame
{
\frametitle{Introducción}
\framesubtitle{Definición de riesgo}

\begin{block}{Riesgo}
	El riesgo en un proyecto es un \textbf{evento incierto} o \textbf{condición incierta} que si ocurre,
	tiene un efecto \textbf{positivo} o \textbf{negativo} sobre el proyecto.
\end{block}

\begin{itemize}
	\item <2->\emph{Riesgos conocidos:}
		Son aquellos que fueron \textbf{identificados}, analizados, y que es posible encontrar
		una minimización de su probabilidad de ocurrencia o de su impacto.
	\item <3->\emph{Riesgos desconocidos:}
		Son aquellos que \textbf{no pueden} ser administrados, lo máximo que se puede hacer es
		basarse en experiencias similares anteriores para mejorar la situación en el
		momento en que ocurren.
\end{itemize}
}

\frame
{
\frametitle{Introducción}
\frametitle{Gestión de riesgo}

\begin{block}{Gestión de riesgo}
	\textbf{Identificación}, \textbf{evaluación} y \textbf{priorización} de riesgos, ya sean positivos o negativos,
	seguida de la aplicación coordinada y económica de los recursos para \emph{minimizar},
	\emph{monitorear} y controlar la probabilidad y/o \emph{impacto} de eventos desafortunados o para
	maximizar la obtención de \emph{oportunidades}.
\end{block}

}
