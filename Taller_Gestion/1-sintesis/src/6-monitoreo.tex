Monitorear y controlar los riesgos es el proceso de ejecución de planes de respuesta ante el riesgo, el seguimiento de los riesgos identificados, supervisar los riesgos residuales, identificando nuevos riesgos, y evaluar la eficacia del proceso de riesgo en todo el proyecto.

Respuestas de Planificación de riesgo que se incluyen en el plan de gestión de proyectos se ejecutan durante el ciclo de vida del proyecto, pero el trabajo del proyecto debe ser objeto de control permanente para las nuevos, cambiantes y actualizados riesgos.

El Monitor de Riesgos y Control de Procesos aplica técnicas, tales como la varianza y análisis de tendencias, que requieren el uso de información sobre el rendimiento generado durante la ejecución del proyecto:

\subsection{Entradas}

\begin{itemize}
	\item Registros de Riesgo
	\item Plan de Administración de Proyectos
	\item Información de Rendimiento en el Trabajo
	\item Reportes de Rendimiento
\end{itemize}

\subsection{Herramientas y Técnicas}

\begin{itemize}
	\item Revaloración de Riesgos
	\item Auditorías de Riesgos
	\item Varianza y Análisis de Tendencias
	\item Medición de Rendimiento Técnico
	\item Análisis de Reserva
	\item Reuniones de Estado
\end{itemize}

\subsection{Salidas}

\begin{itemize}
	\item Actualizaciones de Registro de Riesgos
	\item Actualizaciones de los Activos de Proceso de Organización
	\item Solicitudes de Cambio
	\item Actualizaciones al Plan de Administración de Proyecto
	\item Actualizaciones al Documento de Proyecto
\end{itemize}


