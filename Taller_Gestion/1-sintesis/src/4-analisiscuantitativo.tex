Es el proceso de analizar numéricamente el efecto de la
identificación de riesgos dentro de los objetivos del proyecto.
Se realiza sobre riesgos que ya se priorizaron en el análisis
de riesgo cualitativo y que se estableció que son potenciales
impactos a las demandas del proyecto.

Analiza el efecto de los riesgos, asignando un rating numérico
a cada riesgo o para evaluar el efecto agregado de todos los
riesgos que afectan el proyecto.

Sirve para tomar decisiones, ya que viene después del
Análisis Cualitativo de Riesgos, pero en algunos casos éste análisis
puede no requerir desarrollar respuestas efectivas para los riesgos.

También los resultados nos pueden servir como una especie de monitoreo
y control de riesgos para ver si la cantidad total de riesgos va disminuyendo.

\subsection{Entradas}
\begin{enumerate}
	\item Registro de Riesgos.
	\item Plan de administración de Riesgos.
	\item Plan de administración de costos. (Criterios para planificar, estructurar, estimar, etc)
	\item Plan de administración de programación.
	\item Activos de proceso de organización.(Información de proyectos similares
        anteriores, estudios por especialistas, bases de datos) 
\end{enumerate}

\subsection{Herramientas y Técnicas}
\begin{enumerate}
	\item Obtener datos y técnicas de representación
		\begin{itemize}
			\item Entrevistando.
			\item Distribuciones de probabilidad. ( modelamiento y simulación en valores de incertidumbre, duración actividades, costos, etc)
		\end{itemize}
	\item Análisis cuantitativo de riesgos y técnicas de modelamiento.
		\begin{itemize}
		\item Análisis de sensibilidad. (determinar riesgos con mayor potencial de impacto)
		\item Análisis de valor monetario esperado. (resultados medios de situaciones que pueden o no ocurrir) 
		\item Modelamiento y Simulación. (especificación detallada de incertidumbres y su impacto)
		\end{itemize}
	\item Juicio Experto. (interpretación y experiencia)
\end{enumerate}

\subsection{Salidas}
\begin{enumerate}
	\item Actualizaciones de registro de riesgos.\\
        Se realiza un registro cuando se identifican los procesos de riesgos.
        Esta información se actualiza con la información del Análisis de Riesgo Cuantitativo, que contiene:
		\begin{itemize}
			\item Análisis probabilístico del proyecto. (fechas, costos, etc)
			\item Probabilidad de obtener los objetivos de costo y tiempo.
			\item Lista priorizada de riesgos cuantificados.
			\item Tendencias en los resultados cuantitativos de análisis de riesgos.
		\end{itemize}
\end{enumerate}                                                                                                              
