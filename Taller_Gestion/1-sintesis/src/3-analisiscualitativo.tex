Realizar "análisis de riesgos cualitativo" es el proceso de
priorización de riesgos para realizar análisis adicionales o
determinar el grado de evolución de los actos de una organización,
asociados a una cierta probabilidad de ocurrencia y su impacto.

Las organizaciones pueden mejorar el rendimiento de los proyectos
concentrándose en riesgos de mayor prioridad.
No hay que dejar de lado que existe una Probabilidad de ocurrencia
para cada riesgo, tanto de mayor o menos prioridad.

La presente evaluación refleja también la actitud del equipo del proyecto y
los stakeholders a los riesgos estudiados.
Los tiempos críticos de las acciones relacionadas al riesgo pueden
magnificar la importancia de un riesgo.

La evaluación de la calidad de la información de un proyecto
puede ayudar a la claridad de la evaluación de riesgos.
Realizar esto es una forma rápida y de bajo costo para establecer
las prioridades para las respuestas de un plan de riesgo.

Se realiza durante todo el ciclo de vida del proyecto,
ya que hay ciertos cambios a lo largo del desarrollo del proyecto
que pueden llevar a poseer nuevos riesgos o a eliminar algunos.

\subsection{Entradas}
\begin{enumerate}
	\item Registro de Riesgos.
	\item Plan de gestión de riesgos. (roles, responsabilidades, planificación
		de actividades, categorías de riesgos, definiciones, definición de probabilidad
		e impacto, matriz de impacto, tolerancia de riesgo de los stakeholders.
	\item Declaración del ámbito del proyecto. (proyectos parecidos, tienen los mismos
		riesgos y los que son nuevos tienen más incertidumbre)
	\item Activos de procesos de organización. (Información de proyectos similares
		anteriores, estudios por especialistas, bases de datos)
\end{enumerate}

\subsection{Herramientas y Técnicas}
\begin{enumerate}
	\item Probabilidad de riesgo y evaluación del impacto.
	\item Probabilidad y matriz de impacto. (oportunidades y amenazas, ranking)
	\item Datos de los riesgo en la evaluación de calidad.
	\item Categorización de riesgos.
	\item Evaluación de urgencia de riesgo.
	\item Juicio Experto.
\end{enumerate}


\subsection{Salidas}
\begin{enumerate}
	\item Actualizaciones de registro de riesgos
        Se realiza un registro cuando se identifican los procesos de riesgos.
        Esta información se actualiza con la información del Análisis de Riesgo Cualitativo.
        Cada actualización contiene:
		\begin{itemize}
            \item Ranking relativo o lista de prioridades de los riesgos del proyecto.
			\item Riesgos agrupados por categorías.
			\item Causas de riesgos o áreas de proyecto que requieren una atención particular.
			\item Lista de riesgos que requieren respuesta cercana al equipo de trabajo.
			\item Lista de riesgos para un análisis y respuesta adicional.
			\item Lista de seguimiento de riesgos de baja prioridad.
			\item Tendencias en el resultado del análisis  de riesgos cualitativo.
		\end{itemize}
\end{enumerate}

