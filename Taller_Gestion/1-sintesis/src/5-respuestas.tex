Las respuestas de planificación de riesgo deben ser adecuadas a la importancia del riesgo, el costo efectivo para enfrentarse al reto, realista en el contexto del proyecto, acordado por todas las partes involucradas, y es propiedad de una persona responsable. También debe ser oportuna, para seleccionar la mejor respuesta entre varias opciones de riesgo que se requieren a menudo.

Es el proceso de desarrollo de opciones y oportunidades para mejorar y reducir las amenazas a los objetivos del proyecto. Incluye la identificación y asignación de una persona (el dueño de respuesta al riesgo) para asumir la responsabilidad de cada uno de acuerdo a los riesgos y la respuesta de capitalización. Plan de respuestas al riesgo dirige los riesgos por su prioridad, inserta recursos y actividades en el presupuesto, controla cronograma y gestiona proyectos según sea necesario.

\subsection{Entradas}

\begin{itemize}
	\item Registro de Riesgos
	\item Plan de Administración de Riesgos
\end{itemize}

\subsection{Herramientas y Técnicas}

\begin{itemize}
	\item Estrategias para Riesgos Negativos o Amenazas
	\item Estrategias para Riesgos Positivos u Oportunidades
	\item Estrategias de Respuesta de Contingencia
	\item Juicio Experto
\end{itemize}

\subsection{Salidas}

\begin{itemize}
	\item Actualizaciones de Registro de Riesgos
	\item Decisiones de Contrato relacionadas a Riesgos
	\item Actualizaciones a Plan de Administración de Proyecto
	\item Actualizaciones al Documento de Proyecto
\end{itemize}

