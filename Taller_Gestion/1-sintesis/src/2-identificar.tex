Es el proceso de identificar los riesgo que puedan afectar al proyecto,
documentando sus características principales. Esto es iterativo, ya que nuevos
riesgos pueden evolucionar o aparecer en el transcurso del proyecto. Por ello,
debe involucrar a los miembros del proyecto para inculcar un sentido de
responsabilidad en ellos.

En este proceso, se deben considerar los planes realizados anteriormente,
junto con los costos y duración de cada una de las actividades propuestas y el
lineamiento base sobre los alcances del proyecto. Por ello, los registros con
información sobre los stakeholders y los planes de administración de costos,
tiempos y calidad, junto con los documentos del proyecto, deben ser revisados
exhaustivamente para poder identificara los principales riesgos.

En esta etapa, se debería revisar la documentación que se tenga del proyecto,
y utilizar técnicas de recopilación de la información como Brainstorming,
técnica de Delphi, entrevistas, etc. para poder contar con una buena
información de todos los niveles de trabajo del proyecto. El realizar análisis
con listas de chequeo, analizar las suposiciones y el desarrollo de diagramas
de causa y efecto, flujo del sistema y influencia pueden ayudar a la
identificación de riesgos. Por último, vale la pena nombrar el análisis SWOT
(FODA) y el juicio de expertos para una revisión más profunda de los riesgos.

El resultado de la identificación de riesgos es una documentación con el
registro de riesgos, identificados apropiadamente y con acciones potenciales
para la mitigación de cada uno de ellos.
\subsection{Valores de Entrada}

\begin{enumerate}
    \item Plan de administración de Riesgos.
    \item Costos estimados de las actividades.
    \item Duración estimada de las actividades.
    \item Lineamiento base sobre los alcances del proyecto.
    \item Registros de los stakeholders.
    \item Plan de administración de Costos.
    \item Plan de administración de Tiempos.
    \item Plan de administración de Calidad.
    \item Documentos del proyecto
    \begin{itemize}
        \item Diagramas de red, lineamientos base, eficiencia de trabajo, etc.
    \end{itemize}
    \item Factores del ambiente empresarial.
    \begin{itemize}
        \item Información pública, bases de datos, estudios académicos,
        benchmarking, estudios de la industria, etc.
   
    \end{itemize}
    \item Información organizacional de los procesos.
    \begin{itemize}
        \item Información del proyecto, procesos de control del proyecto,
        templates, lecciones aprendidas, etc.
    \end{itemize}

\end{enumerate}


\subsection{Técnicas y Herramientas}

\begin{enumerate}
    \item Revisiones de la documentación.
    \item Técnicas para la recopilación de la información.
    \begin{itemize}
        \item Brainstorming.
        \item Técnica de Delphi.
        \item Entrevistas.
        \item Análisis de las fuentes del problema.
    \end{itemize}
    \item Análisis con listas de chequeo.
    \item Análisis de las suposiciones.
    \item Técnicas de desarrollo de diagramas.
    \begin{itemize}
        \item Diagramas de causa y efecto, flujo del sistema y influencia.
    \end{itemize}
    \item Análisis SWOT (FODA).
    \item Juicio de Expertos.
\end{enumerate}

\subsection{Valores de Salida}
\begin{enumerate}
    \item Registro de riesgos
    \begin{itemize}
        \item Lista de riesgos identificados.
        \item Lista de acciones potenciales.
    \end{itemize}
\end{enumerate}
