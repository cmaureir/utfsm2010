%APORTE RODRIGO:
La administración de los riesgos de un proyecto es un proceso que involucra a
todas las partes que o conforman, y que es muy necesario para poder estimar
los tiempos de desarrollo del mismo y de poder ayudar a evitar retrasos o el
fracaso del mismo. Además, las herramientas para poder realizar esto requieren de
personas que las manejen de buena forma y que puedan ver la visión holística
del problema para encontrar acciones de mitigación lo más adecuadas posibles.

%APORTE CRISTIAN:
Como equipo de trabajo tenemos muy clara, la importancia del todos los tópicos
que fueron repartidos entre los grupos de trabajos, pero creemos que la gestión
de los riesgos de un proyecto es fundamental, pues si no se tiene un especial cuidado
puedes llegar a acabar con un proyecto determinado.
En nuestra experiencia personal en la "Feria de Software", tuvimos la suerte de ser
los tres parte del mismo equipo, y al finalizar nos dimos cuenta que la parte más
dificil fué poder controlar los distintos riesgos de nuestro proyecto, tanto
del nivel técnico como a niver personal.

%APORTE GABRIEL:
Finalmente, se hace indispensable definir las respuestas que se pueden obtener al 
plantear el plan de riesgos, aplicando tanto herramientas, metodologías y 
técnicas, con el fin de manejar un conjunto de opciones a la hora de mitigar dichos
riesgos. Por otro lado todo proceso debe ser monitoreado y controlado, para ver 
que efectos tiene sobre el todo, con tal de hacer un seguimiento en el tiempo y
verificar que siga el camino establecido en los planes de gestión.
