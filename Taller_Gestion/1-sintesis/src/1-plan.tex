
Es el proceso de definir como realizar las actividades de gestión de riesgo del proyecto.
En el se debe realizar una planificación cuidadosa y bien definida, ya que
ésta mejora las probabilidades de triunfar en los 5 otros procesos de la
administración del riesgo. Por ello, es que éste proceso comienza cuando el proyecto es concebido.

La administración del plan de riesgo requiere considerar los diferentes planes
y factores que posee la empresa. Por ello, hay que considerar los alcances del
proyecto, los planes de administración de costos, tiempos y comunicaciones,
los factores del ambiente empresarial y la información organizacional de los
procesos, para poder definir bien las categorías de riesgo, definiciones en
común de términos y conceptos, formatos de presentación de riesgos, templates
estandarizados, roles y responsabilidades, niveles de autoridad, etc.

Para poder llevar a cabo de buena forma el proceso, es recomendable tener en
cuenta diferentes técnicas y herramientas, dentro de las que se encuentra el
realizar reuniones de análisis y elaboración del plan de administración de
riesgo. En estas reuniones se requieren todos los involucrados del proyecto, y
se busca definir las actividades que se seguirán para controlar los posibles
riesgos y sus costos respectivos, las medidas de contingencia y asignación de
responsabilidades. Éstas se deben categorizar por nivel, probabilidad,
impacto, etc. del riesgo, y generar los templates necesarios para el correcto
control de los mismos.

Al final del proceso se obtendrían los documentos que definen la metodología,
roles y responsabilidades, presupuesto, periodos de evaluación, categorías de
riesgos, definiciones de probabilidad y impacto de los riesgos, la matriz de
probabilidad e impacto, la tolerancia de los stakeholders, los formatos de
reporte y por último la documentación y seguimiento del proceso.
\subsection{Valores de entrada}

\begin{itemize}
    \item Alcances del proyecto.
    \item Plan de la administración de costos.
    \item Plan de administración de tiempos.
    \item Plan de administración de Comunicaciones.
    \item Factores del ambiente empresarial.
    \item Información organizacional de los procesos.
    \begin{itemize}
        \item Categorías de riesgo.
        \item Definiciones en común de términos y conceptos.
        \item Formatos de presentación de riesgos.
        \item Templates estandarizados.
        \item Roles y responsabilidades.
        \item Niveles de autoridad para la toma de decisiones.
        \item Lecciones aprendidas.
        \item Registros de los stakeholders.
    \end{itemize}
\end{itemize}

\subsection{Técnicas y Herramientas}
\begin{itemize}
    \item Reuniones de análisis y elaboración del plan de administración de riesgo.
    \begin{itemize}
        \item Requieren de todos los involucrados del proyecto.
        \item Se busca definir las actividades para la administración del
        riesgo.
        \item Se calculan los costos respectivos.
        \item Revisiones de las medidas de contingencia.
        \item Asignación de responsabilidades.
        \item Categorización de riesgos por nivel, probabilidad, impacto, etc.
        \item Generación de templates.
    \end{itemize}
\end{itemize}

\subsection{Valores de Salida}
\begin{itemize}
    \item Metodología.
    \item Roles y responsabilidades.
    \item Presupuesto.
    \item Periodos de evaluación.
    \item Categorías de riesgos.
    \item Definiciones de probabilidad y impacto de los riesgos.
    \item Matriz de probabilidad e Impacto.
    \item Tolerancia de los stakeholders.
    \item Formatos de reporte.
    \item Documentación y seguimiento.
\end{itemize}
