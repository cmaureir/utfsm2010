% Analisis de la relacion entre publico objetivo y tema tecnologico
% 
% A partir de los datos obtenidos en el Estudio de Campo, uds.
%  deberan analizar los datos cuantitativos y cualitativos del
%  publico objetivo, para determinar los impactos del tema
%  tecnologico con relación al publico objetivo

La herramienta más utilizada entre los encuestados resultó ser de mensajería
instantánea y de videoconferencia (Skype), pasando en segundo lugar a las
llamadas telefónicas y teleconferencias. La preferencia por Skype se puede
comprender a causa de que las tarifas asociadas al servicio de llamadas
internacionales posee un costo muy menor en comparación con los servicios de
telefonía convencional. Un lugar importante fue abarcado por las aplicaciones
de gestión de proyectos y Wiki's, las cuales se puede concluir que son de
alguna forma necesarias por el tipo de proyectos que los encuestados realizan,
que requieren de una buena documentación y detalle de las tareas realizadas
por cada miembro. Todos encuentran necesario el uso de tecnologías para
comunicarse, y todos han tenido experiencias con al menos una de ellas en su
trabajo.

Entre algunas de las respuestas a las problemáticas más comunes al comunicarse
a través de las tecnologías, se nombró el hecho de que las personas no
llegaban a la hora de la reunión. Éste hecho puede ser un problema
considerable, ya que si la reunión se hacen de forma remota, las personas de
una u otra localidad normalmente desconocen las causas del retraso, generando
situaciones de estrés y pérdidas de tiempo de desarrollo de los proyectos. De
todas formas, ese tipo de problemas se puede considerar ajeno a las
tecnologías, siendo inherente a las personas con las que uno trabaje.
Otra problemática que era independiente de las tecnologías que se utilizaran,
pero propia de los trabajos con personas de diferentes países, eran las
diferencias culturales de cada persona. Éstas pueden ser solucionadas a
través de una buena comunicación, realizando reuniones presenciales cada
cierto tiempo para mejorar los lazos entre cada uno y así poder comprender más
las diferencias de cada uno. De todas formas, una cantidad considerable se
encontró con fallas técnicas a la hora de comunicarse, pero la mayoría logró
superarlas profundizando su conocimiento del uso de las herramientas o
cambiando las tecnologías utilizadas. Considerando que las personas
encuestadas poseen un alto nivel técnico, estas soluciones puede que no tengan
la misma efectividad para grupos de trabajo de todo tipo.

Se puede apreciar como la mayoría de los problemas de comunicación eran por
dificultades de idioma, en donde lo común era que se necesitara repetir lo
último dicho, junto con una dificultad para explicar y/o transmitir conceptos,
lo cual se dificulta al no poder utilizar la comunicación no verbal para
expresarse. La mayoría de los encuestados comparten la opinión de que es
necesario el uso de tecnologías para comunicarse hoy en día. 

La mayoría de los encuestados han tenido que integrarse a equipos de trabajo
sin presentar grandes dificultades para aprender nuevas tecnologías. Las
tendencias indican que éstas cada vez son más utilizadas, y las habilidades de
adaptación a las mismas son cada vez mayores, especialmente por el hecho que
las personas desde cada vez más temprano empiezan a interactuar con las
tecnologías de la información, ya sea en la escuela o en sus casas. El
problema normalmente se encuentra con las personas vivieron el proceso de
tecnocratización de los medios de comunicación en una edad más temprana, y
presentan por ello una mayor dificultad o rechazo hacia las nuevas
tecnologías.

En general se concuerda que las tecnologías son imparciales a la hora de
mantener el respeto entre los integrantes, pero dependiendo de que herramienta
se esté utilizando, al disminuir los canales de comunicación, aumentan las
probabilidades de que sucedan interpretaciones erróneas sobre lo que se
intenta comunicar, pudiendo provocar roces innecesarios entre los integrantes
más sensibles del equipo. Por ello, a veces es bueno abstenerse de realizar
recomendaciones, dichos o bromas que puedan ser malentendidas estando fuera de
contexto.
