%Uds. deben realizar conclusiones con relacion a los impactos del tema tecnológico
% en el publico objetivo, al estudio realizado y a sus proyecciones.
%Ademas deben incluir:
%   a) conclusiones sobre el significado que tiene para ustedes el trabajo realizado.

El hecho de realizar este trabajo, no solo significo un aspecto netamente de investigación, sino de auto-conocimiento.
Esto se debe a que todo lo estudiado, lo hemos experimentado en el día a día, en menor o mayor medida, se está logrando
un aporte al uso responsable sobre la utilización de las tecnologías de la información, demostrando no solo que son posibles
su utilización para fines colaborativos y positivos, sino también que el comprender que muchas otras personas que hacen 
utilización de ellas, nos dan la retroalimentación necesaria para indicarnos de que vamos por buen camino a ojos de nuestros similares.
El hecho de comprender que nuestras actividades al momento de comunicarnos, se ven fuertemente soportadas otras personas, incluso de lugares
distantes, nos da la fuerza necesaria para continuar en la misma dirección.

%   b) los conocimientos, habilidades y actitudes desarrollados

Debido a que nuestro trabajo de investigación, se aplica sobre un área con la cual convivimos a diario, los conocimientos y habilidades adquiridas,
tienen neta relación con una interiorización más fuerte de nuestras actividades, de como las organizamos y de los grandes resultados que se logran
al continuar de esta manera, que una vez descubierto la favorable tendencia en cuanto a resultados obtenidos, al utilizar dichos beneficios, encontramos
inminente el desarrollo de actitudes que conlleven a la mejora de estas metodologías, por lo que un trabajo en el cual se centra netamente en la reflexión
de lo que se hace a diario, permite un espacio necesario para la retroalimentación de lo generado, con consiguientes conclusiones, para bien o para mal, pero 
que a fin de cuentas, permite ganar conocimiento nuevo al respecto, permite mejorar nuestras habilidades aplicadas al contexto y nos permite interiorizar
una actitud de constante mejora a nuestro trabajo.

%   c) utilidad de lo realizado para la vida profesional.

En cuanto a lo que nuestro trabajo implica para nuestra futura vida profesional, 
claramente podemos sacar conclusiones bastante positivas, a lo que comunicación
se refiere. Debido a que en los tiempos actuales, el trabajo se hace cada vez más 
colaborativo, esto implica que se trata de buscar e integrar los mejores elementos
para la realización de tareas específicas, por lo que el trabajo a distancia se vuelve
inminente. Por lo mismo, al futuro que probablemente nos encontremos, el comprender del 
por que se hace indispensable la utilización de tecnologías de la información, permite que
tengamos una visión más amplia de lo que ello implica, desde los fundamentos de su funcionamiento, hasta
el como poder mejorar dichas tecnologías, ya sea creando nuevas o integrarlas de mejor manera. Esto último
implica que no solo se genere grandes avances a lo que vida laboral se refiere, sino también a la calidad de 
vida dentro del trabajo, debido a que se pueden utilizar dichas tecnologías para el mejoramiento de las relaciones
interpersonales entre trabajadores a distancia. Finalmente, una de las mayores lecciones que se pueden conseguir con
éstos trabajos de análisis e investigación, es generar en nuestras mentes una linea de tiempo, en donde comprendamos que
tendencia se va generando en el uso de las tecnologías, por lo que solamente por ser futuros profesionales en el manejo 
de la información, implica una responsabilidad a la hora de corregir y dar soluciones al momento de que se equivoque el
camino, algo que no está totalmente lejano a lo estudiado sobre la ``Dictadura Tecnológica''.

