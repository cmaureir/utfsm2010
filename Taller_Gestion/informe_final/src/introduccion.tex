La comunicación es un pilar fundamental no sólo en nuestra sociedad,
sino que en todo el mundo y a lo largo de toda la historia de la humanidad,
por lo cual desde las míticas señales de humo, pasando por la invención del
teléfono hasta la masificación de internet, que la sociedad ha buscado formas
con las cuales mejorar dicha comunicación.

La tecnología por su lado ha sido fundamental para fomentar la ``globalización
comunicacional'', pues es gracias al avance exponencial que ha tenido durante
las últimas décadas, que podemos fácilmente hablar con una persona de cualquier
país en el mundo, ya sea mediante mensajería instantánea, teleconferencia y 
videoconferencia.

Enfocándonos nuevamente en la ``comunicación'', nos podemos dar cuenta que se
aplica en distintas situaciones de nuestra vida cotidiana, pero juega un papel
fundamental a la hora de poder expresar nuestras ideas con personas que no se
relacionan todo el tiempo con nosotros mismos, es decir, cuando uno conoce a
alguien, busca expresarse de la mejor forma posible, para evitar malos entendidos.

El presente trabajo entonces se basa en el último punto señalado, pero enfocándolo
netamente en lo que nosotros conocemos como ``colaboraciones'', es decir, ningún país
por si sólo es capaz de evolucionar por si sólo, siempre se necesitan colaboraciones
para poder afiatar las relaciones, incluso entre instituciones.

En nuestro caso trataremos de ver la temática de un caso de éxito real, como lo es
el grupo ALMA-UTFSM, que siendo parte de CSRG (Computer Systems Research Group) 
del Departamento de Informática de nuestra Universidad, ha logrado llevar a cabo
una colaboración actualmente activa y de renombre, con el proyecto ALMA (Atacama Large
Millimeter/submillimeter Array), siendo una importante agrupación, la cual aporta 
en el desarrollo e investigación, tanto de software como nuevas técnicas 
para dicho proyecto.

Tomaremos como idea central el cómo las tecnologías de la información han ayudado
fundamentalmente para poder tener el nivel de comunicación actual con los colaboradores
del proyecto ALMA, en todo el mundo.
