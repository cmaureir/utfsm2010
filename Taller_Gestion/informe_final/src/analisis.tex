Se describen a continuación los análisis a distintos factores que hemos
decidido como equipo, que son dignos de estudiar.

\subsection{Idioma}
Es conocido que el idioma universal, nos guste o no,
es el inglés, por lo tanto al momento de participar en
cualquier reunión con personas de distintos países,
la única solución es poder comunicarnos con dicho lenguaje.

Por lo tanto cada miembro presente, debe preocuparse no tanto
por saber la traducción de palabras en español al inglés,
sino en algo mucho más complejo, saber expresar las ideas
en un lenguaje distinto que el idioma del país natal.

%Agregar más

\subsection{Zonas Horarias}
Si bien es cierto, para cualquier persona, fijar una reunión
a una cierta hora no es una tarea difícil, tenemos que darnos
cuenta que al momento de pensar en una reunión internacional,
la mayoría de los países poseen una zona horaria distinta, por
lo cual, las personas encargadas tanto de organizar como de
participar en alguna reunión de esa índole, debe poseer el
conocimiento del horario de los países en cuestión,
y saber cuantas horas de diferencia se tienen, pues, mientras
en un lugar el día puede estar comenzando, el otro, puede estar
acabando.

Actualmente en dichas reuniones, uno de los mayores problemas
que se han tenido, son las diferencias muy grandes entre EEUU, 
Alemania y Chile, pues si tomamos en consideración a Chile,
EEUU tiene 2 horas menos y Alemania tiene 6 horas más, lo cual
claramente es un problema, pues se necesita encontrar un balance.

Para nuestro uso, en la Twiki que nuestro grupo de trabajo
utiliza para gestionar proyectos, se encuentra una sección en la cual
se muestran las zonas horarias de los distintos países involucrados
en los proyectos, a través de relojes, lo que a simple vista parecería
algo sencillo, facilita enormemente en situaciones en donde el tiempo
es escaso y se necesita contactar a alguien de estos lugares remotos.

% CORRECCION 4
Para poder resolver el tema de diferencias horarias, se fijan reuniones
en forma de que todos los posibles participantes, señalen que día tienen
disponible, para luego fijar un horario bastante acotado. Generalmente
se suele definir antes del medio día de Chile (11:00 hrs), para que en EEUU puedan 
tener dicha reunión en el inicio de la mañana (9:00 hrs) y Alemania alrededor
de las 17:00 hrs.
% CORRECCION 4

%Agregar más

\subsection{Costumbres}
Es conocido de que lamentablemente en Chile, nos caracterizamos
por ser un poco impuntuales, independiente de si se trata de una
reunión formal, a una fiesta de cumpleaños. Por lo que debido a lo
anteriormente señalado, en todos los países hay distintas costumbres
y es en la puntualidad que nos hemos fijado, en el cual Alemania
destaca por cumplir su palabra completamente, pues al momento
de iniciar una reunión, si es a las 11:00 am, no es ni un par
de minutos antes ni un par de minutos después, pero lamentablemente
tanto en EEUU como en Chile, ocurre lo contrario.

Es por lo anteriormente señalado que cada vez que se comienza una
reunión ocurren los problemas de ``Estamos esperando a una persona
para poder comenzar.''

% índices de productividad en una tabla
\subsubsection{Índices 2009}
\begin{tabular}{|l|r|r|r|l|c|}
\hline
País        & PIB (*)     & PIB per cápita    & Fuerza Laboral    & Desempleo \\%& Productividad \\
\hline
Chile       & \$164.615                        & \$9.672           &  7.30 millones    & 9.7\% (2009)    \\%&   \\
Alemania    & \$3.020.050                      & \$44.729          & 43.51 millones    & 8.5\% (Marzo 2010) \\%& \\
EEUU        & \$14.266.000                     & \$46.442          & 154.50 millones   & 9.7\% (Marzo 2010) \\%& \\
\hline
\end{tabular}
\begin{center}
\vspace{-0.3cm}
\hspace{-10cm}
\emph{(*) en billones de dolares.}
\end{center}


Si se analizan los índices de productividad de Chile y los compara con los de
otros países, como EEUU y Alemania, queda a la vista las grandes diferencias
existentes entre ambos, que más allá de las diferencias económicas de cada uno
de ellos, se relacionan con la cultura y condiciones laborales, como la
duración de las jornadas de trabajo.

Mundialmente, Chile se encuentra en el puesto 54 con \$9,525 de PIB per cápita,
en el International Monetary Fund (2009)\cite{IMF}\cite{IMFpercapita}, EEUU en
el 9 con \$46,381, y Alemania en el 16 con \$40,875.

% CORRECCION 5
% Podriamos buscar estudios en el cual se demuestre estadisticas de estos paises
%y sus tasas de trabajo y ausentismo laboral // Más que ausentismo, tasas de
%productividad, ya que asuentismo se refiere más a las licencias médicas que a
%las sacadas de vuelta
% CORRECCION 5


\subsection{Experiencia Técnica}
Si bien es cierto, expresar las ideas en otro idioma, no es algo
trivial que se le dé innatamente a cualquier persona, para participar
en distintas reuniones donde se discuten temáticas de una índole técnica
es necesario tener el conocimiento técnico necesario para, aparte de
entender la idea que la otra persona está queriendo expresar, entender
que significa lo que está diciendo.
Es por ésto, que las personas que participan en las reuniones,
es gente que conoce lo que es ACS (ALMA Common Software) a un nivel
intermedio, para poder sobrevivir en un lugar donde se discutan ideas
en torno al dicho software.
%Agregar más

\subsection{Medios de comunicación}
Cuando uno realiza teleconferencias con distintas personas,
nunca al finalizar, quedan todas las ideas completamente claras,
o mirado desde otro punto de vista, siempre hay nuevos temas
o dudas respecto a lo que se habló, por lo cual es necesario
comunicarse con las personas que plantearon dicha idea, o que
simplemente están a cargo.

Para poder realizar una comunicación más fluida entonces,
es necesario tener otros medios para poder comunicarnos,
es decir, aparte de Skype~\cite{Skype} para realizar
dichas teleconferencias, se necesitan otras formas, como lo son:
\begin{itemize}
   \item Mensajería Instantánea: Es necesario establecer un protocolo de comunicación
		único para dicho objetivo, en el cual se pueda agregar a las personas de interés, logrando
		una interacción rápida y expedita. En nuestro caso se utiliza Yahoo Messenger.
   \item Correo Electrónico: Si bien es cierto existen las reuniones, es necesario también
		poder tener una comunicación un poco indirecta para comentar problemas, o presentar
		avances de un cierto trabajo a todas las personas interesadas mediante listas de correo.
   \item Acceso Telefónico: Las reuniones se realizan mediante teleconferencias, por lo cual
		es necesario tener acceso telefónico con el extranjero, o en el caso nuestro, poder
		contar con una cuenta Skype para llamadas a todo el mundo.
\end{itemize}

\subsection{Medios de almacenamiento de información}
Para cada reunión apropiadamente desarrollada,
es necesario poseer una minuta para tener un orden
cronológico de los temas a tratar o el orden
en que las personas tienen la palabra, para poder
expresar sus ideas y consultas.

Por lo tanto es muy positivo, que todas las personas
estén familiarizadas con dichos medios, en este caso hablamos
de TWiki (referencia a la TWiki), un medio de almacenamiento
comunitario de comunicación, que es utilizado tanto como para
presentar las minutas, como para el desarrollo de los distintos
proyectos colaborativos alrededor de lo que es ACS.

% CORRECCION 3
Dicha herramienta Twiki, es de carácter online, muy similar al 
utilizado por la enciclopedia Wikipedia (Wiki), por lo que para
todas las personas que cuenten con acceso a ella, les es bastante
útil como herramienta de almacenamiento y trabajo colaborativo.
% CORRECCION 3

%Agregar mas

\subsection{Relación con Tecnologías de la Información}

Si nos damos cuenta, en todos los puntos señalados anteriormente
la tecnología juega un papel fundamental, ya sea desde el
almacenamiento de información, como para una comunicación directa.

% Agregar mas
