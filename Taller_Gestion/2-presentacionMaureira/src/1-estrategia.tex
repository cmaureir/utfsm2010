\frame
{
\frametitle{Estrategias de liderazgo}
\vspace{1cm}
\begin{center}
	``El liderazgo al final es \blue{meramente} la forma
		en que el \blue{jefe} (de cualquier nivel) \red{vive}
		cada momento ''
\end{center}
\begin{columns}
\begin{column}{0.6\textwidth}
\end{column}
\begin{column}{0.4\textwidth}
	\includegraphics[width=0.9\textwidth]{img/liderazgo}
\end{column}
\end{columns}
}

\frame
{
\frametitle{Estrategias de liderazgo}
\framesubtitle{Contar Historias}
\begin{itemize}
	\item Característica \blue{necesaria} para ser un buen líder (Maestría).
	\item \red{Tarea número 1}:
		$$ \text{Celebrar a los renegados, pequeños triunfos, fallos útiles.}$$
	\item El \blue{truco}, encontrarlos (Difícil)
	\item Aplaudir, celebrar, recompensar a las personas que lo hacen.
	\item Es la esencia del liderazgo para \red{estimular la innovación}. 
\end{itemize}
}

\frame
{
\frametitle{Estrategias de liderazgo}
\framesubtitle{Hable sin rodeos de la innovación}
\begin{itemize}
	\item La estrategia de la empresa está en el \blue{informe anual} (enfatiza lo que el \red{jefe} quiere ventilar).
	\item El informe debe \blue{oler a innovación} (favorecer a la empresa).
	\item La innovación tiene que ser una \red{obsesión}, así el equipo también lo hará.
	\item La innovación debería estar en todos \blue{medios} de la empresa (boletines, circulares, memorias, ...)
\end{itemize}
}

\frame
{
\frametitle{Estrategias de liderazgo}
\framesubtitle{Predique el ``fallo como parte de la vida''}
\begin{itemize}
	\item Variación de hablar sin rodeos.
	\item Hablar sin rodeos de los \blue{fallos} y \blue{errores}, grandes o pequeños.
	\item Es un \blue{atributo} de un líder obsesionado por la innovación.
	\item \red{No} se está induciendo a la gente para que meta la pata.
	\item El fallo es \red{esencial} para el progreso.
\end{itemize}
}

\frame
{
\frametitle{Estrategias de liderazgo}
\framesubtitle{Proclame que en la ``vida hay que arriesgarse e intentarlo''}
\begin{itemize}
	\item \emph{``Nosotros no planeamos, hacemos''} (Quadracci)
	\item Filosofía, \blue{Cualquier} cosa que merece la pena, merece la pena incluso mal hecha.
	\item Hay que \blue{empezar} de alguna manera, aunque \blue{no} sea óptimo.
	\item En la vida hay que \red{arriesgarse} e intentar!
	\item \red{Fomente} experimentos! de todos, vicepresidentes o recepcionistas, Pida mas!
\end{itemize}
}
