\frame
{
\frametitle{Estudiar ``innovación''}
\vspace{1cm}
\begin{center}
	``¿Cómo \green{pensar} en el confuso mundo de la innovación?''\\
	``¿Cómo \green{introducirnos} nosotros mismos en ese mundo?''
\end{center}
\begin{columns}
\begin{column}{0.6\textwidth}
\end{column}
\begin{column}{0.4\textwidth}
	\includegraphics[width=0.8\textwidth]{img/innovacion}
\end{column}
\end{columns}
}

\frame
{
\frametitle{Estudiar ``innovación''}
\framesubtitle{Aprenda de los mejores}
\begin{itemize}
	\item Utilizar el \red{Robo creativo}.
	\item Tener \blue{opiniones} y adaptarlas al negocio.
	\item Hacer que cada empleado se \blue{entere}, \blue{aprenda}, \blue{adapte}, \blue{ensaye} ideas de algún lado.
\end{itemize}
\begin{columns}
\begin{column}{0.6\textwidth}
\end{column}
\begin{column}{0.4\textwidth}
	\includegraphics[width=0.8\textwidth]{img/robo}
\end{column}
\end{columns}
}

\frame
{
\frametitle{Estudiar ``innovación''}
\framesubtitle{Lea los libros adecuados}
\begin{itemize}
	\item Seguir las técnicas explicadas anteriormente...hasta que \red{no} se \blue{entienda} todo bien.
	\item \blue{Antídoto} es comenzar a \red{leer}!
	\item Los mejores libros sobre innovación \red{no} son los textos convencionales.
	\item Revisar estudios como:
	\begin{itemize}
		\item ``The soul of a new machine'' % proceso de inventar un producto que salvara una empresa
		\item ``Final cut'' % relato del mayor fiesco mundo en el rodaje de una pelicula
		\item ``Fumbling the future'' % historia burocratica de xerox y su perdida de liderazgo en el negocio informatico
	\end{itemize}
	\item La innovación es un proceso \blue{desalentador}.
	\item Hay que tener \green{genes perturbadores}.
\end{itemize}
}
