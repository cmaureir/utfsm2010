\frame
{
\frametitle{Solución}
\framesubtitle{}
\begin{itemize}
	\item<1-> Main
	\item<2-> Función en el \blue{host} para preparar los datos y variables.
	\item<3-> Función en el \blue{device} para realizar las operaciones.
\end{itemize}
}


\frame
{
\frametitle{Solución}
\framesubtitle{Main}
\begin{itemize}
	\item<1-> Declaración y asignación de memoria de \red{C}.
	\item<2-> Leer archivo de entrada y generar matrices (\red{A} y \red{B}).
	\item<3-> Llamar a función en el \blue{host} (\texttt{Mul()}).
\end{itemize}
}

\frame
{
\frametitle{Solución}
\framesubtitle{Función en el \emph{Host} (\texttt{Mul()})}
\begin{itemize}
	\item<1-> Cargar las matrices \red{A} y \red{B}, copiando sus valores.
	\item<2-> Cargar la matriz vacía \red{C} en el \blue{device}.
	\item<3-> Calcular la dimensión del Bloque y del Grid.
	\item<4-> Llamada al kernel \texttt{Muld()}.
	\item<5-> Leer \red{C} del \blue{device}.
\end{itemize}
}

\frame
{
\frametitle{Solución}
\framesubtitle{Función en el \emph{Device} (\texttt{Muld()})}
\begin{itemize}
	\item<1->  Calcular índices
	\begin{itemize}
		\item<2-> Determinar la posición de \blue{inicio} y \blue{fin} de submatrices.
		\item<3-> Determinar tamaño de \red{paso} por cada iteración.
	\end{itemize}
 	\item<4-> Bucle de cálculo.
\end{itemize}
}
