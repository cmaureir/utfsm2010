\documentclass[letter, 10pt]{article}
\usepackage[utf8]{inputenc}
\usepackage[spanish]{babel}
\usepackage{amsfonts}
\usepackage{amsmath}
\usepackage[pdftex]{graphicx}
\usepackage{epstopdf}
\usepackage{url}
\usepackage{hyperref}
\usepackage{verbatim}
\usepackage[top=3cm,bottom=3cm,left=3.5cm,right=3.5cm,footskip=1.5cm,headheight=1.5cm,headsep=.5cm,textheight=3cm]{geometry}


\begin{document}
\bibliographystyle{plain}
\pagestyle{empty}

\title{Inteligencia Artificial \\ \begin{Large}Informe Final: ``Car Sequencing Problem''\end{Large}}
\author{Cristián D. Maureira Fredes.}
\date{\today}
\maketitle


%--------------------No borrar esta sección--------------------------------%
\section*{Evaluación}

\begin{tabular}{ll}
Mejoras 1ra Entrega (10 \%): &  \underline{\hspace{2cm}}\\
Codigo Fuente (10 \%): &  \underline{\hspace{2cm}}\\
Representaci\'on (15 \%):  & \underline{\hspace{2cm}} \\
Descripci\'on del algoritmo (20 \%):  & \underline{\hspace{2cm}} \\
Experimentos (10 \%):  & \underline{\hspace{2cm}} \\
Resultados (10 \%):  & \underline{\hspace{2cm}} \\
Conclusiones (20 \%): &  \underline{\hspace{2cm}}\\
Bibliograf\'ia (5 \%): & \underline{\hspace{2cm}}\\
&  \\
\textbf{Nota Final (100)}:   & \underline{\hspace{2cm}}
\end{tabular}
%---------------------------------------------------------------------------%

\begin{abstract}
%Resumen del informe en no más de 10 líneas.
El presente documento tiene como finalidad presentar el estudio y análisis de un problema clásico, a lo que la Inteligencia Artificial respecta,
el \emph{Car Sequencing Problem}~\cite{parello}, que consiste en la planificación de una fila de vehículos en una línea de ensamblaje, 
mientras se satisfacen todas las restricciones de capacidad de las plantas ensambladores y de los propios vehículos.

La importancia de atacar el presente problema radica en el notorio aumento de la actividad en las fábricas automotrices a lo largo del mundo,
dejando como consecuencia que incluso para el pasado año 2002, ya existían 1 automóvil por cada 10 personas en el mundo~\cite{worldmapper}.

Se plantean además las técnicas más conocidas y utilizadas para poder enfrentar el presente problema, entre las cuales destacan la utilización
de heurísticas y métodos híbridos, dentro de los que son los acercamientos exactos, híbridos y mixtos.

Se presenta un modelamiento simplista utilizando lo que es Programación Lineal, y un planteamiento más teórico para poder comprender a fondo
el problema estudiado.



%problem
%solucion propuesta
%resultados
%keywords

\end{abstract}

\section{Introducción}
\label{sec:introduccion}
Una expresión regular,
a menudo llamada también patrón,
es una expresión que describe un conjunto de cadenas sin enumerar sus elementos.

Además,
normalmente representan otro grupo de caracteres mayor,
de tal forma que podemos comparar el patrón con otro conjunto de caracteres para ver las coincidencias.

La mayoría de las formalizaciones proporcionan los siguientes constructores:
una expresión regular es una forma de representar a los lenguajes regulares
(finitos o infinitos) y se construye utilizando caracteres del alfabeto
sobre el cual se define el lenguaje.

Específicamente,
las expresiones regulares se construyen utilizando los operadores
unión, concatenación y clausura de Kleene.

Las expresiones regulares en UNIX (RE's) están recogidas en el estándar POSIX 1003.2.



\section{Definición del Problema}
\label{sec:definicion}
%Descripción del problema, su complejidad, en que consiste, cuales son sus objetivos,
%restricciones, variantes más conocidas. 

El \emph{Car Sequencing Problem} es un problema de satisfacción de restricciones (CSP), posee la
característica de ser NP-duro~\footnote{
NP-duro es el conjunto de los problemas de decisión que contiene los problemas H
tales que todo problema L en NP puede ser transformado polinomialmente en H.
Esta clase puede ser descrita como conteniendo los problemas de decisión que son
al menos tan difíciles como un problema de NP.}
, además corresponde a un tipo de variación del problema NP-completo \emph{Job-Shop Scheduling}.%,
%pero con un uso de razonamiento automatizado, es decir, con un enfoque dedicado a estudiar y comprender
%diferentes características del razonamiento, permitiendo así construir programas que le den la posibilidad
%a los computadores para razonar en forma autónoma.
 
Siguiendo la misma idea, es válido señalar que el \emph{Car Sequencing Problem} es un tipo de problema de planificación
de tareas en una línea de ensamblaje de autos, donde cada uno es perteneciente a un clase de automóvil, debido al conjunto
de opciones y accesorios que posee (airo acondicionado, centralizado eléctrico, etc), y cada una de las opciones o
accesorios se instala en una planta distinta, por lo que el objetivo principal es el poder encontrar el orden en la
secuencia de los vehículos, preocupándonos de no exceder la capacidad de cada planta de ensamblaje y también cumplir con la demanda.

Por lo tanto, si realizamos una definición más formal de nuestro problema, podríamos decir lo siguiente:
Teniendo una lista de vehículos dada, cada uno con sus respectivas opciones requeridas,
necesitamos establecer un orden en la línea de ensamblaje, con el fin de que cada subsecuencia de $q$ vehículos
tengamos a lo más $p$ que requieren de una determinada opción. Es importante tener en consideración que los
valores de $p$ y $q$ están asociados a cada opción de los vehículos.

Con respecto a la información que el problema otorga, podemos decir que contamos con:
\begin{itemize}
	\item Cantidad de vehículos de cada tipo o clase a producir (demanda)
	\item Lista de las opciones con la cual se constituye cada tipo o clase de vehículo, la cual puede utilizar una representación
		booleana para saber si cierto tipo de automóvil posee o no una determinada opción.
	\item Capacidad de las plantas que se preocupan de instalar la determinada opción.
\end{itemize}

Nuestro objetivo principal es:
\begin{itemize}
	\item Encontrar un orden en nuestra secuencia, que sirva para minimizar el costo por cada restricción insatisfecha.
\end{itemize}

Con respecto a las restricciones, tenemos que:
\begin{itemize}
	\item En cada subsecuencia de los $q$ vehículos, a lo más pueden haber $p$ que requieran de la opción determinada.
		Donde $p$ y $q$ son valores asociados a cada opción.
	\item La capacidad de cada planta de ensamblaje no puede ser excedida, es decir, cumplir con la demanda de cada automóvil
		sin abusar de una planta determinada.
	\item Por cada tipo de auto, el numero de autos de ese tipo debe ser secuenciado, es decir, todos los automóviles de cada clase
		deben estar presente en una secuencia determinada.
\end{itemize}





\section{Estado del Arte}
\label{sec:estado}
%En que nivel se ha investigado la técnica aplicada al problema
%     (existe o no investigación, cuantas, cuales modelos, componentes
%     de la técnica, como es su desempeño, etc.).
%     En caso de no existir investigación o si a usted le parece que es
%     un aporte, experiencias con problemas similares o que podría ser útil
%     para su trabajo.

%existe o no?
%experiencias similares
\frame
{
\frametitle{Estado del Arte}
\framesubtitle{Existencia del acercamiento}
\begin{itemize}
	\item Actualmente en las grandes fuentes de publicaciones, como lo son:
	\begin{itemize}
		\item ACM.
		\item Springerlink.
		\item IEEE.
		\item Google Scholar.
	\end{itemize}
	\item \textbf{NO} existe un acercamiento utilizando \emph{Clonal Selection}
		para la resolución del \emph{Car Sequencing Problem}.
	\item Estado del arte del \emph{ROADEF}.
\end{itemize}
}

\frame
{
\frametitle{Estado del Arte}
\framesubtitle{Experiencias Similares}
\begin{block}{Mobile Robot Path Planning}
\begin{itemize}
	\item Operadores Inmunes:
	 \begin{itemize}
	 	\item \emph{Operador de Mutación:}
			consiste en elegir aleatoreamente un nodo del camino y \textbf{reemplazarlo} por otro nodo que no esté en el camino original. (intercambio de opciones válidas)
		\item \emph{Operador de Inserción:}
			se utiliza para poder \textbf{reparar} los segmentos de un camino infactible, insertando un nodo entre el problema. (cambiar el elemento inválido por otro)
		\item \emph{Operador de Supresión:}
			se aplica a los caminos factibles e infactibles, para \textbf{disminuir costos}. (formas de disminuir el fitness por orden)
	 \end{itemize}
\end{itemize}
\end{block}
}

\frame
{
\frametitle{Estado del Arte}
\framesubtitle{Experiencias Similares}
\begin{block}{Scheduling Aircraft Landing}
\begin{itemize}
	\item Basar la selección clonal en:
	\begin{itemize}
		\item \emph{Infeasibility Degree (IFD):}
			maneja las restricciones de una buena manera y guía el proceso de optimización de manera efectiva.
		\item \emph{Excellent Gene Segment Spread (EGSS):}
			mejora la velocidad de convergencia del algoritmo.
	\end{itemize}
\end{itemize}
\end{block}
}

\frame
{
\frametitle{Estado del Arte}
\framesubtitle{Experiencias Similares}
\begin{block}{Vehicle Routing Problem}
\begin{itemize}
	\item Operadores de Mutación:
	\begin{itemize}
		\item \emph{EXC:}
			elige \textbf{esquinas} del camino y las trata de \textbf{unir}. (no sirve mucho)
		\item \emph{SUM:}
			que intenta \textbf{concatenar} dos rutas sin violar restricciones. (tipo de cruzamiento en un punto)
		\item \emph{NEW:}
			que \textbf{construye} nuevas ruta utilizando la heurística greedy a partir de un vértice aleatorio. (mejorar individuos particulares)
	\end{itemize}
\end{itemize}
\end{block}
}

\frame
{
\frametitle{Estado del Arte}
\framesubtitle{Experiencias Similares}
\begin{block}{Job-shop scheduling problem}
\begin{itemize}
	\item Variar la función objetivo en caso de encontrar mínimos locales, para escapar de los \emph{óptimos locales}.
	\begin{itemize}
		\item Primero, aumenta la función objetivo para hacer desaparecer los mínimos locales. (sumándole un factor con constantes positivas)
		\item Segundo, estira el vecindario de la variable auxiliar. (del paso anterior)
	\end{itemize}
\end{itemize}
\end{block}
}

\frame
{
\frametitle{Estado del Arte}
\framesubtitle{Experiencias Similares}
\begin{block}{Parallel Graph Coloring Problem}
\begin{itemize}
	\item Inicializar anticuerpos de forma aleatoria o utilizando \emph{greedy}.
	\item Utilizar conceptos de paralelismo.
	\item Concepto de \emph{migración}.
\end{itemize}
\end{block}
}


\section{Modelo Matemático}
\label{sec:modelo}
%Uno o más modelos matemáticos para el problema, idealmente indicando el espacio de búsqueda para cada uno.
\subsection{Primer modelo}
El presente modelo, pese a utilizar notaciones matemáticas, trata de dar un enfoque más teórico,
para una mejor comprensión del problema.

Consideremos el \emph{Car Sequencing Problem} como una tupla $(C,O,p,q,r)$, tal que:
\begin{itemize}
	\item $C=\{c_1,\ldots,c_n\}$ es el conjunto de vehículos a producir.
	\item $O=\{o_1,\ldots,o_m\}$ es el conjunto de distintas opciones.
	\item $p:O\rightarrow\mathbb{N}$ and $q:O\rightarrow\mathbb{N}$ definen la capacidad de las restricciones,
		es decir, para cualquier opción $o_i \in O$, cada subsecuencia de $q_i$ vehículos consecutivos en la línea,
		deben no contener más que $p_i$ vehículos que requieran la opción $o_i$
	\item $r:C\times O \rightarrow \{0,1\}$ define los requerimientos de las opciones, es decir, para cada vehículo
		$c_i \in C$ y para cada opción $o_j \in O$, $r_{ij} = 1$ si $o_j$ debe ser instalado en $c_i$ y $r_{ij} = 0$
		en otro caso.
\end{itemize}

%Solving a car sequencing problem involves finding an arrangement of the cars
%in a sequence, defining the order in which they will pass along the assembly
%line, such that the capacity constraints are met. We shall use the following
%notations to denote and manipulate sequences:

\begin{itemize}
	\item Una \emph{secuencia}, definida $\pi =< c_{i_{1}}, c_{i_{2}}, \ldots, c_{i_{k}} >$, es una sucesión de vehículos.
	\item El conjunto de todas las secuencias que pueden ser construidas con un conjuntos de autos $C$ es denominada $\amalg_{C}$.
	\item El \emph{largo} de una secuencia $\pi$, denominado $|\pi|$, es el numero de vehículos que contiene.
	\item La \emph{concatenación} de dos secuencias, $\pi_1$ y $\pi_2$, denominada $\pi_1 \cdot \pi_2$, es una secuencia
		compuesta de los vehículos de $\pi_1$ seguidos de los vehículos de $\pi_2$.
	\item Una secuencia $\pi_1$ es una subsecuencia de otra secuencia $\pi_2$ , tomando en cuenta $\pi_1 \subseteq \pi_2$,
		si existen dos secuencias $\pi_3$ y $\pi_4$, tal que $\pi_2 = \pi_3 \cdot \pi_1 \cdot \pi_4$
	\item El \emph{cost} de una secuencia $\pi$ es el numero de las restricciones de capacidad que no se cumplen, es decir:
\end{itemize}

$$costo(\pi) = \sum_{o_{i}\in O} \ \ \ \sum_{\pi_{k}\subseteq \pi\\ \rightarrow |\pi_{k}|=q_i} violaci\acute{o}n(\pi_{k},o_{i})$$
$$\text{donde}\ violaci\acute{o}n(\pi_{k},o_{i}) = \left\{
 \begin{array}{l}
	0\ si\ \sum\limits_{<C_l>\subseteq\pi_k}r_{li}\leq p_i;\\
	1\ \text{en otro caso}\\
 \end{array} \right.$$


Por lo tanto se puede resolver ahora el ejercicio buscando la secuencia con el \textbf{mínimo costo}, compuesta por todos los autos
que serán producidos.

%

Respecto al espacio de búsqueda, si tenemos que $\alpha_o$ es la cantidad de vehículos que tienen que ser producidos
por cada clase/configuración o, con $o\in O$ y sea $n$ la cantidad total de vehículos a producir,
entonces el espacio de búsqueda está compuesto por todas las posibles permutaciones de los vehículos en $C$ que
requieren diferentes configuraciones, particionado en los $o$ tipos distintos tipos de vehículos a construir.

Por lo tanto el tamaño de las posibles soluciones sería:
$$E.B\ =\ \frac{n!}{\prod\limits_{o \in O}\alpha_o}$$
 
Claramente tenemos que darnos cuenta que varias soluciones dentro del presente espacio de búsqueda
no van a ser posibles, pues debemos aplicar las restricciones.

\subsection{Segundo modelo}
\textbf{Programación Lineal Entera (ILP)}

Definición de \emph{variables} y \emph{parámetros}:
\begin{itemize}
	\item \textbf{$opt$}	    Número total de opciones.
	\item \textbf{$cl$}	    Número total de tipos/clases de vehículos.
	\item \textbf{$n_i$}    Número de vehículos en la clase/tipo $i$.
	\item \textbf{$nc$}	    Número total de vehículos.
	\item \textbf{$o_{ik}$}	Parámetro \emph{booleano} que representa si los vehículos del tipo/clase $i$ requieren la opción $k$.
	\item \textbf{$s_k$}    Largo de una secuencia de vehículos consecutiva, donde algunas requieren la opción $k$
	\item \textbf{$r_k$}    Número máximo de vehículos que pueden tener la opción $k$ en una secuencia consecutiva de $s_k$ vehículos.
	\item \textbf{$M$}	 	Valor escalar grande que puede ser fijado a $(s_{k} - r_{k})$
	\item \textbf{$C_{ij}$}	Variable \emph{booleana} que representa si un vehículo del tipo/clase $i$ es asignado a la posición $j$ en la secuencia
		$(C_{ij} = 1)$
	\item \textbf{$Y_{kj}$}	Varibale \emph{booleana} que representa si el número de vehículos que requieren la opción $k$ en una secuencia
		de $s_k$ vehículos, comenzando de la posición $j$ en la secuencia respecto a $r_k$, el máximo permitido $(Y_{kj} = 0)$
	\item \textbf{$pp$}	    Número de vehículos en el periodo anterior, que permiten establecer la transición $(pp = max(s_{k}-1))$
	\item \textbf{$cpp_{mk}$}	Variable \emph{booleana} que representa si el vehículo $m$-ésimo del periodo previo tiene la opción $k$.
	\item \textbf{$Z_{kj}$}	Variable \emph{booleana} que representa si el número de vehículos que necesitan la opción $k$ en una secuencia
		de $s_k$ vehículos, partiendo en la posición $j$ de la secuencia en el periodo anterior, respetando el máximo $r_k$ permitido $(Z_{kj} = 0)$.
\end{itemize}

Por lo tanto, el modelo a continuación se establece para minimizar el número de violaciones de las restricciones de capacidad,
tanto como para opciones de alta prioridad, como de baja.

\textbf{Función objetivo:}
$$\text{Min} F = \sum\limits_{k=1}^{opt} \sum\limits_{j=1}^{nc-s_{k}+1} Y_{kj} + \sum\limits_{k=1}^{opt} \sum\limits_{j=1}^{s_{k}-11} Z_{kj}$$

\textbf{Restricciones}
\begin{itemize}
	\item $\sum\limits_{i=1}^{cl} C_{ij} = 1$, $\forall j = 1,\ldots,nc$
	\item $\sum\limits_{j=1}^{nc} C_{ij} = n_{i}$, $\forall i = 1,\ldots,cl$
	\item $\sum\limits_{i=1}^{cl} \sum\limits_{l=j}^{j+s_{k}-1} o_{ik} \cdot
		C_{il} \leq r_{k} + M\cdot Y_{kj}$,\\ $\forall k = 1,\ldots,opt$ y $\forall j=1,\ldots, nc-s_{k}+1$
	\item $\sum\limits_{i=1}^{cl} \sum\limits_{l=1}^{j} o_{ik} \cdot
		C_{il} + \sum\limits_{m=pp-S_{k}+j+1}^{pp} cpp_{mk} \leq r_{k} + M\cdot Z_{kj}$,\\ $\forall k = 1,\ldots,opt$ y $\forall j=1,\ldots, s_{k}-11$
\end{itemize}

Respecto al espacio de búsqueda, como ésta formulación es ILP igual que el primero modelamiento, sólo sería la permutación de todos
los vehículos que se necesitan,  particionados por las clases de vehículos que se producirán. Obviamente posicionándonos en el peor de los casos
donde tenemos que recorrer todo el espacio de búsqueda.

$$E.B\ =\ \frac{nc!}{\prod\limits_{i = 0}^{cl}\alpha_{n_i}}$$

Por ejemplo, si tuvieramos $30$ vehículos a ensamblar, $4$ clases de autos y
necesitamos $10$ autos de clase $1$, $5$ de clase $2$, $7$ de clase $3$ y $8$ de clase 4.

$$E.B\ =\ \frac{30!}{10\cdot 5\cdot 7\cdot 8}\ =\ 9.473316\times 10^{28}$$




%%%

\section{Representación}
\label{sec:representacion}
\frame
{
\frametitle{Representaciones}
\begin{center}
	\includegraphics[width=0.7\textwidth]{img/representacion}
\end{center}
}

\frame
{
\frametitle{Representaciones}
\framesubtitle{Simulated Annealing}
\begin{itemize}
	\item Variables utilizadas
		\begin{itemize}
			\item {\tt T\_max:} Temperatura máxima.
        	\item {\tt N\_recal:} Número de recalentamientos.
			\item {\tt alpha:} Porcentaje de la temperatura máxima a recalentar.
			\item {\tt Delta\_t:} Temperatura a decrecer, el decrecimiento es aritmético.
			\item {\tt X:} Almacena la solución actual del problema, además se define {\tt x\_gen} y {\tt x\_opt}.
			\item {\tt Delta\_x:} Variación en la función objetivo.
		\end{itemize}
\end{itemize}
}

\frame
{
\frametitle{Representaciones}
\framesubtitle{Simulated Annealing}
\begin{itemize}
	\item Movimiento
	\begin{itemize}
		\item El movimiento escogido es el {\bf swap}, que realiza el cambio de dos variables.
		\item Este cambio se realiza al {\bf azar}, lo que genera una solución que será aceptada bajo las condiciones:
		\begin{itemize}
			\item Si es la función objetivo es menor que la actual, será aceptada.
			\item Si $P(0,1) \leq e^{-|delta_x|}$, será aceptada.
		\end{itemize}
	\end{itemize}
\end{itemize}
}

\begin{frame}[fragile]
\frametitle{Representaciones}
\framesubtitle{Simulated Annealing}
\tiny{
\begin{verbatim}
Inicio
t<-t_max // temperatura inicial
cont <- 0 // número de recalentamientos hechos
Leer Datos
solución<-Generar solución inicial
rest_h<-Contar restricciones violadas de solución
salir<-falso
rest<-rest_h
Mientras cont<numero
    rechazado<-falso
    Mientras rachazado=falso
        solución_gen<- Realizar movimiento a solución
        rest_gen<-Contar restricciones violadas de solución_gen
       	Verifica la aceptación
		Cambiar solución y rest por solución_gen y rest_gen si es mejor o es peor pero se acepta
        Cambiar solución_h y rest_h por solución_gen y rest_gen si es mejor
    Fin Mientras
    Decrecimiento de la temperatura   
    Si t<0 Entonces
        cont<-cont+1
        t<-t*alpha //alpha es una valor entre 0 y 1
    Fin Si
Fin Mientras
Imprimir Resultados
Fin
\end{verbatim}
}
\normalsize
\end{frame}

\frame
{
\frametitle{Representaciones}
\framesubtitle{Algoritmo Evolutivo + Simulated Annealing}
\begin{itemize}
	\item Aproximación a al representación matemática (fórmula).
	\item Estructuras de datos:
	\begin{itemize}
		\item Individuos (\texttt{{\bf struct} genotype}).
		\begin{itemize}
			\item {\tt{\bf int} gene[VARS]}, {\tt {\bf int} fitness}, {\tt {\bf int} fail}.
			\item {\tt {\bf double} rfitness}, {\tt {\bf double} cfitness}.
		\end{itemize}
	\end{itemize}
\end{itemize}
}

\frame
{
\frametitle{Representaciones}
\framesubtitle{Algoritmo Evolutivo + Simulated Annealing}
\begin{itemize}
	\item Estructuras de datos:
	\begin{itemize}
		\item Mejor Solución SA. ({\tt {\bf int} saBestSol[VARS]}).
		\item Número máximo de autos por subsecuencia.\\({\tt {\bf int} numMaxCarOptSeq[N]}).
		\item Tamaño de la subsecuencia. ({\tt {\bf int} sizeMaxCarOptSeq[N]}).
		\item Demanda y descripción de tipos de autos. ({\tt {\bf int} types[N][M]}).
	\end{itemize}
	\item Movimiento SA:
	\begin{itemize}
		\item {\bf swap}
		\item {\tt 10\% VARS}
	\end{itemize}
\end{itemize}
}

\begin{frame}[fragile]
\frametitle{Representaciones}
\framesubtitle{Algoritmo Evolutivo + Simulated Annealing}
\tiny{
\begin{verbatim}
Inicio
g <- 0 & p <- 0 // numero de generaciones y poblaciones
Leer datos de entrada
Población <- Generar población inicial
Evaluar población
    Mientras g < GENS
        Mientras n < POP - 1
            Selección individuo
            Mutación // Simulated Annealing + AM
        Fin Mientras
        Elitismo
        Cambio de población // Población actual = Población nueva
        Evaluar población
        n <- n + 1  &  p <- 0
    Fin Mientras
Imprimir resultados
Fin
\end{verbatim}
}
\normalsize
\end{frame}



\section{Descripción del algoritmo}
\label{sec:descripcionAlgoritmo}
% Como fue implementando, interesa la implementaci\'on m\'as que el algoritmo gen\'erico,
%  es decir, si se tiene que implementar SA, lo que se espera es que se explique en pseudo
%  codigo la estructura general y en parrafo explicativo cada parte como fue implementada
%  para su caso particular, si se utilizan oparadores se debe explicar por que se utilizo

%  ese operador, si fuera el caso de una t\'ecnica completa, si se utiliza recursi\'on o no,
%  etc. En este punto no se espera que se incluya codigo, eso va aparte.
El algoritmo utilizado para resolver el \emph{Car Sequencing Problem (CSP)} fue un
\emph{Algoritmo Evolutivo} con mutación \emph{Simulated Annealing}.

Las siguientes sub-secciones describen con más detalle como se adaptó la teoría de
los presentes algoritmos para resolver el \emph{CSP}.

\subsection{Población Inicial}
Una de las cosas mas importantes a la hora de evaluar si una solución es viable o no,
es su factibilidad, es decir, si cumple con todas las \emph{restricciones duras}.

Para solucionar lo anterior es que al momento de crear \emph{la población inicial},
se realizó de una forma aleatoria, pero cumpliendo con todas las \emph{restricciones
duras}, es decir, generando sólo \emph{restricciones factibles}.

Primero que todo se generó una secuencia ordenada con la cantidad de tipos de autos
requerida, es decir, si tuviéramos las siguientes demandas:
\begin{itemize}
	\item Tipo 1: 2 autos.
	\item Tipo 2: 1 auto.
	\item Tipo 3: 3 autos.
	\item Tipo 4: 6 autos.
\end{itemize}
La secuencia generada sería:

$$1\ 1\ 2\ 3\ 3\ 3\ 4\ 4\ 4\ 4\ 4\ 4$$

Una vez tengamos la secuencia, procedemos a generar todos los individuos de nuestra población
inicial, lo cual se realizó, desordenando la secuencia anterior, de una forma aleatoria.

% conclusión
Si nos damos cuenta, la forma de generar nuestra población, quizás no es la más adecuada
debido a que se generó de manera aleatoria, pero al generar solo soluciones factibles,
estamos mejorando mucho más éste aspecto, sin embargo, la utilización de  una técnica para
la construcción de soluciones iniciales, podría mejorar el performance de nuestro algoritmo,
por ejemplo si utilizáramos una técnica \emph{Greedy} o \emph{GRASP}.

\subsection{Evaluación de la población}

Para realizar la evaluación de una determinada población, se tiene en cuenta lo siguiente:

Es necesario recorrer toda la población, luego, una vez seleccionado un individuo, se procede
a recorrer mediante subsecuencias (de tamaño establecido en el problema, $sizeMaxOptCarSeq$)
todo el individuo (serie de autos), y por cada iteración vamos verificando si se cumplen las
restricciones blandas, es decir, sumamos la cantidad de autos con una opción determinada y
luego verificamos si esa suma es menor o igual al máximo permitido por las restricciones ($numMaxOptCarSeq$).

Si por cada suma en cada subsecuencia, no se cumple la restricción, se le asigna un \emph{fitness}
equivalente a las unidades por las cuales se excede el máximo establecido, al individuo en cuestión.

Finalmente, si encontramos alguna restricción factible que posea un \emph{fitness} igual a $0$,
inmediatamente, nos quedamos con dicha solución y el algoritmo termina.

\subsection{Selección de Individuos}

Al momento de seleccionar un individuo para realizar una transformación, se utilizó la conocida técnica llamada
\emph{roulette wheel}, para una Función Objetivo que minimiza.

El procedimiento es bien simple, sólo tenemos que considerar el \emph{fitness} de cada individuo y calcular un
\emph{fitness relativo} de la siguiente forma:

$$relativeFitness_{i}\ = \frac{f_{min} + f_{max} - f_{i}}{\sum\limits_{i=0}^{sizePop} (f_{min} + f_{max} - f_{i}}$$

Donde $f_{max}$ equivale al \emph{fitness} del mejor individuo,
$f_{min}$ equivale al \emph{fitness} del peor individuo y 
$f_{i}$ equivale al \emph{fitness} del i-ésimo individuo de nuestra población.

La suma de todos los \emph{fitness relativo} equivale a $1$.

Luego de que cada individuo posee su \emph{fitness relativo}, se procede a calcular un \emph{fitness acumulativo},
es decir, ir sumando las probabilidades para generar un rango entre $0$ y $1$ con todas nuestras probabilidades.

Una vez se tiene el \emph{fitness acumulativo} listo, se procede a obtener un número aleatorio entre $0$ y $1$,
para que luego sea ubicado en nuestro rango, y así el individuo que salga escogido con éste número aleatorio, será
elegido para pasar ahora a la transformación.

El siguiente ejemplo, ayuda a la comprensión del presente procedimiento:\\

\begin{minipage}{0.2\textwidth}
	\begin{tabular}{|r|r|}
	\hline
	\textbf{individuo} & \textbf{Fitness} \\ \hline
	1 & 50 \\\hline
	2 & 30 \\\hline
	3 & 20 \\\hline
	\end{tabular}
\end{minipage}
\  \ 
\hfill \begin{minipage}{0.2\linewidth}
	Tenemos ahora:
	\begin{itemize}
		\item $f_{min}$: 20
		\item $f_{max}$: 50
	\end{itemize}
\end{minipage}
\  \
\hfill \begin{minipage}{0.35\textwidth}
	\vspace{0.4cm}
	\begin{itemize}
		\item $f_{min}\ +\ f_{max}$: 70
		\item $\sum\limits_{i=1}^{3} (f_{min} + f_{max} - f_{i})$: 110
	\end{itemize}
\end{minipage}\\


Ahora procedemos a calcular el fitness relativo y acumulativo:\\

\begin{center}
\begin{tabular}{|r|r|r|r|}
\hline
\textbf{individuo} & \textbf{Fitness} & \textbf{Fitness relativo} & \textbf{Fitness acumulativo}\\ \hline
1 & 50 & 0.18 & 0.18 \\\hline
2 & 30 & 0.36 & 0.54 \\\hline
3 & 20 & 0.46 & 1.00 \\\hline
\end{tabular}
\end{center}

\begin{figure}[htb!]
	\begin{center}
	\includegraphics[scale=0.4]{img/fig1}
	\end{center}
	\label{fig:fig1}
	\caption{Gr\'afico final mediante el método de la ruleta}
\end{figure}

Si ahora escogiéramos un número aleatorio entre $0$ y $1$, por ejemplo $0.6$, tenemos que mirar la \textbf{Figura 1} de la sección~\ref{fig:fig1} y caeríamos
en el lugar del individuo 3, por lo que él sería escogido.

De esta forma, se les da una probabilidad a cada individuo para poder pasar a la nueva población, dependiendo
netamente de que tan bueno sea.

Este proceso se repite hasta que tenemos a todos los individuos de nuestra población nueva.


\subsection{Transformación de Individuos}

Por lo general, en las implementaciones de transformación, existe tanto la \emph{mutación} como el \emph{cruzamiento},
de los cuales, ambos poseen una cierta probabilidad asociada, para que cuando se escoja un individuo, se obtiene
un número aleatorio entre $0$ y $1$ y si ese número es menor o igual que la probabilidad de mutación, se muta,
lo mismo ocurre con el cruzamiento.

\subsubsection{Mutación}
Según los requerimientos del problema, la mutación debe ser la técnica \emph{Simulated Annealing} y se escogió la variación
de \emph{Alguna Mejora}, para obtener un menor tiempo de ejecución.

Pasos de la implementación:

\begin{itemize}
	\item Paso 0: Inicialización.
	\begin{itemize}
		\item X := solución inicial factible
		\item tmax := máximo número de iteraciones (IMAX)
		\item q := temperatura alta inicial (TMAX)
		\item Mejor solución := X
		\item Número de iteraciones = t := 0
	\end{itemize}
	\item Paso 1: Parada.
	\begin{itemize}
		\item Si la cantidad de iteraciones mayor a tmax o si la temperatura es menor que $0$,
			entonces paramos.
		\item Entregar mejor solución.
	\end{itemize}
	\item Paso 2: Movimiento.
	\begin{itemize}
		\item Realizamos el movimiento, en éste caso se realiza un \emph{swap},
			pero como cada individuo es del orden de $200$, $300$ y $400$ autos,
			se realiza una cantidad de \emph{swap} equivalente al $10\%$ de la cantidad de autos.
			
			Para ver que elementos hacen el \emph{swap}, se eligen aleatoriamente dos elementos para intercambiar.
		\item Calculamos la disminución de la FO (Dobj).
	\end{itemize}
	\item Paso 3: Aceptación.
	\begin{itemize}
		\item Si X(t+1) mejor el objetivo ó
		\item Si $e^{\frac{-Dobj}{q}}\ \ge\ random(0,1)$ entonces
		\item X(t+1) := X(t), sino
		\item volver al Paso 2.
		
	\end{itemize}
	\item Paso 4: Reemplazar el mejor.
	\begin{itemize}
		\item Si el valor de la FO de X(t+1) es mejor a la Mejor Solución, entonces:
		\item Mejor solución := X(t+1)
	\end{itemize}
	\item Paso 5: Reducción de la temperatura.
	\begin{itemize}
		\item Si han pasado 3 iteraciones desde el último cambio de temperatura, reducimos q a un $90\%$.
		\item q := q * 0.9.
	\end{itemize}
	\item Paso 6: Incrementar.
	\begin{itemize}
		\item t := t + 1, Volver al Paso 1. 
	\end{itemize}
\end{itemize}

Una vez terminada la mutación, se pasa el individuo (Mejor solución) a la siguiente población.

\subsubsection{Cruzamiento}
Por el lado del cruzamiento, la presente implementación no lo posee, debido a la \emph{representación} del problema.
La técnica escogida fue \emph{cruzamiento en un punto}, pero no se realizó, debido a que rompería las restricciones duras
del problema, y nos generaría individuos que serían soluciones infactibles.\\

A continuación se señala un diagrama explicativo:
Consideremos las siguientes demandas por 3 tipos de autos.\\

\begin{minipage}{0.2\textwidth}
	\begin{itemize}
		\item Tipo 1: 3
		\item Tipo 2: 1
		\item Tipo 3: 2
	\end{itemize}
\end{minipage}
\  \  
\hfill\begin{minipage}{0.4\textwidth}
	\begin{itemize}
		\item Padre 1: 3 2 1 1 3 1
		\item Padre 2: 1 3 3 2 1 1
	\end{itemize}
\end{minipage}

\begin{figure}[htb!]
	\begin{center}
	\includegraphics[scale=1]{img/fig2}
	\end{center}
	\label{fig:fig2}
	\caption{Explicación de violación de restricciones duras por cruzamiento}
\end{figure}

En la \textbf{Figura 2} de la sección~\ref{fig:fig2} podemos darnos cuenta que no estamos respetando las restricciones
duras de la cantidad de autos de Tipo 2 y 3, entonces si nos damos cuenta de que éste es un ejemplo
pequeño, la cantidad de restricciones violadas con una mayor cantidad de autos, será mucho mayor.

Como en el caso de los algoritmos evolutivos, el cruzamiento se encarga de explotar,
la explotación es suplida por la mutación \emph{Simulated Annealing}, ya que como bien sabemos,
a temperaturas altas, explora y a temperaturas bajas, explota.

\subsection{Elitismo}

Para la presente implementación, se consideró el elitismo como una pieza fundamental para poder
mejorar nuestra nuevas poblaciones.

Una vez que tengamos una población, el elitismo se encarga de buscar el individuo con más alto \emph{fitness},
y se reemplaza con el individuo que tenga el fitness más bajo, además siempre el mejor individuo de cada población
pasa directamente a la siguiente población, con lo cual se asegura no perder soluciones buenas con el pasar
de las poblaciones.

\subsection{Estructura del Algoritmo}

Tomando en cuenta toda la descripción anterior,
el algoritmo central sería:

\begin{verbatim}
	Inicio
	g <- 0 // numero de generaciones
	p <- 0 // numero de poblaciones
	Leer datos de entrada
    Población <- Generar población inicial
    Evaluar población

	    Mientras g < GENS
			Mientras n < POP - 1
				Selección individuo
				Mutación // Simulated Annealing + AM 
			Fin Mientras
			Elitismo
			Cambio de población // Población actual = Población nueva
			Evaluar población
	        n <- n + 1
	        p <- 0
		Fin Mientras
	Imprimir resultados
	Fin
\end{verbatim}

\subsection{Parámetros}

Para la presente implementación, se trabajó con una seria de constantes definidas antes de la ejecución del algoritmo,
las cuales fueron variadas para observar el desempeño del algoritmo, dejando las más apropiadas:
\begin{description}
	\item[VARS]: Número de variables (autos) de la secuencia, el cual es cambiado para cada experimento ($200$, $300$ y $400$).
	\item[POP]:  Tamaño de cada población.
	\item[GENS]: Número de generaciones.
	\item[PMUT]: Probabilidad de mutación. Se considero de que como no se realiza cruzamiento, la presente probabilidad se aumentó.
	\item[TMAX]: Temperatura Máxima para el \emph{Simulated Annealing}.
	\item[IMAX]: Número máximo de iteraciones para el \emph{Simulated Annealing}.
\end{description}


\section{Experimentos}
\label{sec:experimentos}
% Se necesita saber como experimentaron, como definieron par\'ametros,
%  como los fueron modificando, cuales problemas se trataron, instancias,
%  por que ocuparon esos problemas.

Los valores iniciales, eran menores a los señalados al final de la presente sección,
y para poder definirlo se fueron variando por separado.

Primero se fueron variando por separado los parámetros del \emph{algoritmo evolutivo},
donde se llegó a la conclusión, de que si la cantidad de las generaciones va aumentando,
y el tamaño de la población es muy grande, aparte de tardar demasiado en converger,
el desempeño no es bueno.

Luego se estableció la idea de tener poblaciones pequeñas, pero muchas generaciones,
así obtendremos mejores resultados, pero dependerá netamente, si las soluciones iniciales
son buenas, aquí el elitismo juega un papel sumamente importante.

En segundo lugar, las pruebas y modificaciones se realizaron con los parámetros de la
mutación \emph{simulated annealing}, donde se probaron distintos valores, que hicieron
llegar a la conclusión, de que es más apropiado, poseer un número no tan grande de iteraciones,
con respecto a la temperatura, pero la temperatura debe ser a lo más 1 orden de magnitud mayor
que el número de iteraciones.

De la misma forma, la temperatura, no debe tener valores muy elevados, porque no ayuda mucho
en encontrar mejores individuos, sobre todo en éste caso que era un \emph{simulated annealing}
con \emph{alguna mejora}.

Las instancias en las que se ejecutó el algoritmo, fueron los \emph{30 problemas difíciles} de
Caroline Gagne que se encuentran en el sitio de CSPLib~\cite{gagne}, por cada instancia se ejecutó
el presente algoritmo $100$ veces, para obtener una amplia gama de soluciones a distintas situaciones.

Los experimentos fueron realizados en un computador con las siguientes características:
\begin{itemize}
	\item Intel(R) Core(TM)2 Duo CPU 2.66\,[GHz]
	\item 4 Gigabyte RAM
	\item Sistema Operativo Fedora 12 para i686
\end{itemize}

Los parámetros utilizados en los experimentos, no fueron cambiados por cada población,
lo cual puede significar un desempeño inferior al sintonizarlos por en cada caso.

Los valores son:
\begin{itemize}
	\item VARS: 200, 300, 400 (dependiendo de cada caso)
	\item POP: 12
	\item GENS: 5000
	\item PMUT: 0.3
	\item TMAX: 100
	\item IMAX: 10
\end{itemize}


\section{Resultados}
\label{sec:resultados}
\frame
{
\frametitle{Resultado}
\framesubtitle{}
\begin{itemize}
	\item tabla y grafico
\end{itemize}
}


\section{Conclusiones}
\label{sec:conclusiones}
%Conclusiones revelantes del estudio realizado.

En el presente informe se ha dado un estado del arte de un problema muy popular
en el área de la inteligencia artificial, el \emph{Car Sequencing Problem}, siendo éste
una variación de otro problema connotado llamado \emph{Job Shop Scheduling}.
Es tanto la importancia del presente problema, que la \emph{French Society of Operations
Research and Decision-Making Aid} ha decidido ya hace varios años, comenzar lo que se denomina
\emph{The ROADEF challenge} cada dos años, teniendo como objetivo central,  permitir a las personas
que se desarrollan en el área de la industria el presenciar todos los avances y evoluciones
en el ámbito de la Investigación de Operaciones y Análisis de Decisiones, pero no sólo eso
sino el poder enfrentar directamente problemas decisionales complejos, que ocurren en la industria.
Siguiendo la idea anterior, lo importante de éste \emph{Challenge} es que en el 2005, se consideró
como tema principal el \emph{Car Sequencing Problem} debido a la propuesta que realizó RENAULT,
por lo cual uno podrá imaginar la cantidad de avances que se produjeron, pues cada participante
abordaba el problema desde una metodología distinta.

Por otra parte, pareciera que un problema relacionado a \emph{ordenar} un conjunto de vehículos
para ser ensamblados y así obtener el orden más óptimo, no es una tarea difícil, pero claramente
debido a la complejidad que otorgan las restricciones y de que es un problema de la vida real,
presenta un grado de dificultad mayor, lo cual queda reflejado por la cantidad de publicaciones 
e investigaciones que hay al respecto.

Se dieron a conocer también, tres áreas para atacar el presente problema.
Por un lado tenemos los métodos heurísticos que como bien sabemos, es prácticamente jugar a la ruleta
rusa con nuestra investigación, pues la heurística solamente selecciona un objetivo de los dos provenientes
de la definición, una buena solución o un buen tiempo de ejecución. Pero también se presenta que la heurística
es un mecanismo confiable para decidir \emph{utilizarlo} como un apoyo, mas que utilizarlo solo.

Siguiendo con los mecanismos planteados, se vieron también los  métodos exactos,
es decir, técnicas de optimización, donde podemos encontrar la \emph{programación lineal entera},
\emph{branch and bound} y \emph{local search}, los cuales se dedicaban netamente a construir una
solución óptima a partir de los datos que el mismo problema nos entrega. El único problema que tienen
éstas técnicas es que la complejidad temporal va a crecer demasiado con respecto al tamaño de nuestro
\emph{input} del algoritmo.

Dentro de toda la lectura realizada para las distintas técnicas, pude percatarme que las mejores soluciones
siempre son variaciones de métodos o tomar dos técnicas como complementarias, por ejemplo uno de los
mejores resultados fue la combinación de un \emph{Ant Colony Optimization} con una heurística dinámica,
pues claramente se nos señala que el buen uso de una heurística es crucial, es decir, hay que preocuparse
de leer los estudios que se han publicado, par ver cual es la combinación más óptima.

Finalmente, es impresionante la cantidad de estudios con respecto a éste problema en particular,
por lo que podemos darnos cuenta que muchos centros de investigación han dedicado tiempo valioso
para la resolución óptima del \emph{Car Sequencing Problem}, pero no tanto la versión que se estudió,
que es la propuesta por Parello~\cite{parello}, sino mas bien al desafío de la ROADEF.


%%%

\section{Bibliografía}
\bibliography{informe2}

\end{document} 
