El presente trabajo tiene como principal objetivo poner a prueba los conocimientos
adquiridos en el ramo de ``Bioinformática'', para con respecto  a las secuencias
de aminoácidos o más bien dicho, las proteínas en sí.\\

Para realizar las preguntas se tuvo en consideración como primera instancia el 
``GenBank'' el cual es la base de datos de secuencias genéticas del
\emph{National Institutes of Health de Estados Unidos} (NIH),
que nos entrega de manera fácil y gratuita una colección de secuencias
de ADN determinadas para poder realizar cualquier investigación,
de la misma forma consta también con secuencias de aminoácidos, entre otros.\\

El ``GenBank'' es parte de \emph{International Nucleotide Sequence Database Collaboration},
que está integrada por la base de datos de ADN de Japón (DNA DataBank of Japan (DDBJ)),
El Laboratorio Europeo de Biología Molecular (European Molecular Biology Laboratory (EMBL)),
y el GenBank en el \emph{National Center for Biotechnology Information}.
Las presentes organizaciones intercambian datos diariamentes producidas en laboratorios de
todo el mundo, procedentes de más de 100.000 organismos distintos.\\

Finalmente, luego de ser necesaria una interpretación de la información recopilada,
es necesario realizar otras actividades como asociación de organismos que poseen proteínas
determinadas, buscar el parentesco entre distintos \emph{matchs}, etc.
Todo a partir de una secuencia de aminoácidos formada por nuestro nombre.
