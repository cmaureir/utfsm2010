Cuando hablamos de Informática, nos estamos refiriendo a la automatización de la información,
lo cual nos entrega un amplio campo de investigación en torno a su significado, lo cual nos
hace pensar en las distintas oportunidades que tendremos como futuros profesionales,
pero es raro que en el ámbito de la Biología, en la cual actualmente se basa netamente
en estudios de datos ya existentes, salvo por nuevas investigaciones, no tenga más
instancias a lo largo de nuestra formación como ingenieros, donde podamos realizar
actividades y adquirir conocimiento.\\

A modo personal, siempre ha existido algún interés por la Biología,
pero no es grato que sólo en mi 5to año de carrera pueda comenzar a ver un ápice
de sus aplicaciones informáticas, y además es demasiado notorio que la cantidad
de temáticas posibles son realmente significativas.\\

Debido al poco entrenamiento biológico que poseo, me costó en demasía poder
entender rápidamente toda la información con la que me iba encontrando, por lo que tenía
que recurrir a cada momento a hacer un repaso de conceptos básicos para poder comprender
a lo que me estaba enfrentando. Distinto fue con la lecturas de algunas publicaciones
en las que buscaba análisis filogenético del organismos que poseía la proteína en cuestión,
donde el nivel técnico biológico con el que se referían me restringía casi en su totalidad
la comprensión de los contenidos de la publicación.\\

Finalmente es bueno poder darse cuenta de éstas grandes bases de datos mundiales se encuentran
al servicio público y que cualquier persona pueda utilizarlas sin ningún problema,
o la capacidad de poder hacer \emph{matchs} entregándoles una secuencia determinada en las
millones de secuencias que poseen, desplegando la información necesaria de sus orígenes,
publicaciones, referencias, imágenes, etc, incentivando así a cualquier interesado.
