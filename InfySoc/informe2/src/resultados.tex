%El analisis de los datos obtenidos,
%	arrojara resultados que uds. deberan presentar en esta seccion.
%	Deben poner enfasis a los impactos del tema tecnologico en el
%	publico objetivo, y sus proyecciones.

En la sección anterior, pudimos darnos cuentas de los resultados de la encuesta realizada a personas que trabajan
en el mismo ambiente que se presenta en el trabajo, dándonos cuenta cuales herramientas utilizaban, si les eran útiles, etc,
pero ahora nos queremos enfocar en los \emph{impactos} principalmente, de nuestro tema principal, en el las personas estudiadas.

Con respecto a las tecnologías, podemos profundizar mucho acerca del proceso de \emph{selección}, \emph{desarrollo} y su propio \emph{uso},
y sobre todo, en el análisis a los impactos producidos por cada uno de los factores anteriormente señalados.
Éstos impactos, no influyen sólo en la persona, sino que van más allá, poseen una suerte de influencia en el quehacer humano y por ende
todo lo que lo rodea, \emph{alterando} su comportamiento.

Analizando uno de los investigadores más importantes en ésta área, McLuhan~\cite{mcluhan}, podemos obtener ciertas preguntas que nos van a servir para ver el impacto
de las tecnologías en nuestro ambiente.
Las preguntas son:
\begin{itemize}
	\item ¿Qué genera, crea o posibilita?
	\item ¿Qué preserva o aumenta?
 	\item ¿Qué recupera o revaloriza?
	\item ¿Qué reemplaza o deja obsoleto?
\end{itemize}

Las cuales debemos resolver por cada tecnología que se quiera evaluar.

Si consideramos las \emph{tecnologías de la información} como un elemento tecnológico, podemos pasar a responder dichas preguntas para el trabajo en general:
\begin{itemize}
	\item \emph{¿Qué genera, crea o posibilita?}\\
		Gracias a las tecnologías de la información, estamos generando nuevos caminos de comunicación para realizar determinadas actividades en cualquier área,
		creando así las instancias necesarias, para que la comunicación sea bastante directa, en comparación a metodologías antiguas.
		
		Tomando en cuenta lo anterior estamos posibilitando, que dos o más personas en lugares totalmente distintos en el mundo, puedan trabajar en conjunto,
		sin tener la necesidad de una cercanía física, lo cual nos sirve para obtener una especie de comodidad a la hora de trabajar.
		
	\item \emph{¿Qué preserva o aumenta?}\\
		Estamos preservando y aumentando en algunos casos, la cantidad de trabajo realizado en conjunto, pues tenemos que tener en cuenta que al existir una
		comunicación más fluida, aumentamos la relación de dos distintas organizaciones, en un proyecto determinado.

		Por otra parte, aumenta la capacidad de que grupos pequeños de estudiantes con una misma motivación, puedan realizar actividades que contribuyan
		a proyectos mundiales, en los cuales por una razón económica, o quizás considerando la comodidad, no pueden estar presentes en un determinado lugar
		en el mundo.
	
		Tomando en cuenta ambos puntos, podemos darnos cuenta, que el crecimiento como profesional de un determinado estudiante, crece enormemente,
		pues ya no está sometido a un estilo de trabajo local, de su universidad y/o entorno, conociendo nuevas costumbres, estilos de trabajo, etc.

 	\item \emph{¿Qué recupera o revaloriza?}\\
		Al momento de abastecernos con todas los recursos necesarios utilizando tecnologías de la información, podemos comenzar a recuperar valores o 
		buenas costumbres al momento de realizar un proyecto, revalorizando el trabajo realizado; con ésto nos referimos a que cuando uno se encuentra
		en un establecimiento educacional y realiza cierto proyecto, no tiene el mismo valor, obtener una buena calificación en una asignatura determinada,
		que se reconozca el mismo trabajo, por una organización mundialmente conocida.

		Comenzamos a darnos cuenta, que junto a la perseverancia podemos llegar a cumplir objetivos bastante ambiciosos.		

	\item \emph{¿Qué reemplaza o deja obsoleto?}\\
		Si el uso de la tecnología no es utilizado de una buena forma, es decir, no abusando y llevando al máximo las comodidades que nos ofrece,
		se puede transformar en algo negativo para una persona o un entorno social, reemplazando todo lo que son las relaciones interpersonales,
		el poder entablar una conversación ``en persona'' con algún otro individuo.

		Por otro lado, la tecnología actual ha dejado de lado (no del todo) a medios de comunicación antiguos, como lo son las cartas tradicionales,
		el Fax, el telégrafo, etc , que en su tiempo, fueron lo último de la tecnología, con lo que podemos decir, que la misma tecnología,
		deja obsoleta a la tecnología.

\end{itemize}

Considerando que éste análisis puedo no ser del todo completo con respecto a la apreciación de los impactos de las \emph{tecnologías de la información},
existen trabajos complementarios para comprender mucho mejor los fenómenos observados.

Para identificar mejor los impactos, ya sean negativos o positivos de las distintas actividades relacionadas con tecnologías sobre las personas,
Solivérez~\cite{soliverez} propone un nuevo conjunto de preguntas:
\begin{itemize}
	\item \emph{Impacto práctico:}
	\begin{itemize}
		\item ¿Para qué sirve?
		\item ¿Qué permite hacer que sin ella sería imposible?
		\item ¿Qué facilita?
	\end{itemize}
	\item \emph{Impacto simbólico:}
	\begin{itemize}
		\item ¿Qué simboliza o representa?
		\item ¿Qué connota?
	\end{itemize}
	\item \emph{Impacto tecnológico:}
	\begin{itemize}
		\item ¿Qué objetos o saberes técnicos preexistentes lo hacen posible?
		\item ¿Qué reemplaza o deja obsoleto?
		\item ¿Qué disminuye o hace menos probable?
		\item ¿Qué recupera o revaloriza?
		\item ¿Qué obstáculos al desarrollo de otras tecnologías elimina?
	\end{itemize}
	\item \emph{Impacto ambiental:}
	\begin{itemize}
		\item ¿El uso de qué recursos aumenta, disminuye o reemplaza?
		\item ¿Qué residuos o emanaciones produce?
		\item ¿Qué efectos tiene sobre la vida animal y vegetal?
	\end{itemize}
	\item \emph{Impacto ético:}
	\begin{itemize}
		\item ¿Qué necesidad humana básica permite satisfacer mejor?
		\item ¿Qué deseos genera o potencia?
		\item ¿Qué daños reversibles o irreversibles causa?
		\item ¿Qué alternativas más beneficiosas existen?
	\end{itemize}
	\item \emph{Impacto epistemológico:}
	\begin{itemize}
		\item ¿Qué conocimientos previos cuestiona?
		\item ¿Qué nuevos campos de conocimiento abre o potencia?
	\end{itemize}
\end{itemize}

Las cuales nos van a permitir obtener un argumento mas completo a la hora de analizar los impactos que nos interesan.
Para éste caso sólo nos limitaremos a analizar el \emph{Impacto Ético}.

\begin{itemize}
	\item \emph{¿Qué necesidad humana básica permite satisfacer mejor?}\\
		La necesidad de la comunicación con personas que no se encuentres cerca, geográficamente hablando.
		Estamos mucho más conectados con todo el mundo, por lo que no importa el lugar, siempre podremos
		estar en contacto de las personas que necesitemos.

	\item \emph{¿Qué deseos genera o potencia?}\\
		Los deseos de continuar con distintos proyectos, en nuevas organizaciones, generar grupos de trabajo,
		potenciando el trabajo en equipo internacional, lo cual se ve beneficiado al momento de querer
		generar lazos aún más fuertes, donde se recurren a eventos como \emph{Workshops} en algún lugar del mundo,
		donde aparte de conocer a las personas e interactuar con ellas, se crean nuevos lazos que podrán estar
		vigentes con el pasar del tiempo, gracias a la tecnología de la información.

	\item \emph{¿Qué daños reversibles o irreversibles causa?}\\
		Con respecto a los daños reversibles, tenemos la adicción a la información que se ha dado en varias personas,
		tomando las palabras del último trabajo del ramo, puede producirse una esclavitud tecnológica.
		Por otro lado tenemos el problema anteriormente señalado, que es la pérdida de las relaciones sociales,
		que puede ser controlada, moderando el uso de los medios de comunicación tecnológicos actuales.

		Ahora considerando los daños irreversibles, tenemos que la actitud de algunas personas puede transformarse
		negativamente a asumir la facilidad de obtener o realizar tareas, sin ir más lejos los daños al lenguaje,
		que hoy en día la juventud posee, que tienen como fin sólo agilizar la comunicación, alterando el mensaje.
		
		Pero concentrándonos en nuestro problema, el único problema que se puede temer,
		son la pérdida de las relaciones interpersonales, que si bien es cierto considerando las colaboraciones
		internacionales es positivo que la gente trabaje, no hay que producir cierta esclavitud, no podemos vivir
		sin relacionarnos con otras personas, ya que eso es lo que sustenta la sociedad.

	\item \emph{¿Qué alternativas más beneficiosas existen?}\\
		Más que buscar una alternativa a las tecnologías de la información, tenemos el fiel pensamiento de que
		no hay que buscar alternativas sino que hay que aprender a vivir con la tecnología, que hoy por hoy,
		nadie sabe utilizar de la mejor forma la tecnología.

		¿Tenemos clara la noción de lo bueno y lo malo? a veces la información puede alterar nuestra percepción
		de ciertas actividades, pero en el caso de las relaciones internacionales, no se ve tanta coherencia
		a la presente idea.
\end{itemize}
