\subsection{Público Objetivo}

\subsubsection{Descripción}
El público objetivo para el presente estudio, se compone principalmente
por personas que se encuentren realizando proyectos de carácter internacional, 
los cuales debido a su distribución geográfica, no les permite una comunicación
relativamente fluida. Además de verse forzosamente necesitados de medios de
comunicación para suplir la distancia. Por otro lado, el no presentar
importancia el área de dichos proyectos e idioma en que se lleve a cabo.

\subsubsection{Información existente}
% Estube buscando alguna publicación/página que hable de la cultura
% americana/alemana, pero no he logrado encontrar nada util =/.

%sobre las herramientas para el trabajo colaboratio
Existen organizaciones que ya han experimentado con herramientas para el
trabajo colaborativo en linea para mejorar la fluidez del intercambio de la
información entre todos sus miembros~\cite{herramientas_trabajo_colaborativo}.
Uno de estas es el uso de wikis. Las wikis~\cite{wikis} son sitios web cuyas
páginas pueden ser editadas por múltiples personas desde un navegador web,
permitiendo redactar documentos colectivamente, siendo ampliamente usadas hoy
en día. El caso más emblemático es el de la wikipedia\cite{wikipedia},
enciclopedia virtual para compartir el conocimiento, editada por voluntarios
de todas las partes del mundo.
Algunos de los softwares más utilizados para ello son
MediaWiki~\cite{mediawiki}, MoinMoin~\cite{moinmoin} y Twiki~\cite{twiki}.
Otros tipos de herramientas utilizadas para la coordinación de diferentes equipos
de trabajo son las herramientas de Software Configuration Management
(SCM~\cite{scm}). Normalmente son utilizadas en actividades de desarrollo de
software, y permiten tratar y controlar la elaboración de código fuente por
varios desarrolladores simultáneamente, pudiendo realizar un seguimiento del
estado de las versiones y sus cambios. Algunas de las herramientas más utilizadas para
esto son Git~\cite{git}, un sistema de control de versiones distribuido,
Subversion (SVN~\cite{svn}), un sistema de control de versiones centralizado,
y  Trac~\cite{trac}, sistema que integra una wiki y sistemas de control de
versiones como Git y SVN, y posee múltiples plugins~\cite{plugin} para
extenderle nuevas funcionalidades.



\subsection{Dominio de la Investigación}

\subsubsection{Descripción}
Con respecto al dominio de la presente investigación, son dos distintas
instancias donde personas de Estados Unidos, Alemania y Chile se reunen
a discutir temáticas en relación al proyecto ALMA:

\begin{itemize}
	\item ACS Weekly Meeting: Reunión donde asisten los colaboradores
(desarrolladores) más importantes del proyecto ALMA, la cual es coordinada
 desde Garching, Alemania.
	\item OSF Coordination Meeting: Reunión en la cual acuden personas de Nuevo
México (EEUU), Garching (Alemania) y Valparaíso (Chile), la cual es 
coordinada por los representantes del grupo ALMA-UTFSM. En ella se plantean temas
de interés general.
\end{itemize}

El dominio de investigación trata sobre que será medido en el corto y largo plazo, 
con respecto a la influencia que tienen las tecnologías de la información en la forma
de trabajo mencionado anteriormente. Se tratará de enfocar los resultados en relación
a objetivos a corto plazo como lo son estados de avances en tareas específicas, para 
medir en que manera influyen las tics en una comunicación de avances de manera fluida e 
intercultural. A largo plazo se medirá a nivel de colaboración el grado de cohesión que 
estas organizaciones logran gracias a la influencia de las tics, refiriéndonos a cantidad
de proyectos logrados con éxito y creación de instancias colaborativas.

\subsubsection{Información existente}

A nuestra investigación le interesan los temas de la comunicación y las
relaciones interpersonales. El dominio es bastante amplio, desde el estudio de
la comunicación oral y corporal la construcción de las relaciones
interpersonales que forman a los grupos de trabajo tanto como las tecnologías
existentes para suplir o complementar a los medios de comunicación.
Dentro de estos ámbitos, existen investigaciones\cite{obs_collaborative} que
han concluido dentro de la información transmitida en las comunicaciones
interpersonales, el lenguaje corporal, el uso de gestos, manos y la forma en
cada persona gesticula transmite significativa información adicional, y cómo
el espacio en el cual se están comunicando influye en las relaciones de
trabajo.

En el trabajo colaborativo desarrollado entre diferentes grupos separados
geográficamente, el tipo de visibilidad que halla de los demás, el moverse
dentro del mismo ambiente compartido, el estar presentes al mismo tiempo, la
audibilidad, tangibilidad y simultaneidad son claramente diferentes al estar
limitados a la comunicación utilizando los medios que las tecnologías nos
ofrecen para ello. Como afecta cada uno de estos factores a la forma de
organización y la forma de trabajo a desarrollar ha sido
estudiada~\cite{proximity_collaboration} concluyendo que, aunque las personas
presentan una mayor facilidad para relacionarse en ambientes de proximidad
física, es posible para las personas colaborar a través de la distancia,
usando cualquier tecnología que tengan disponible. Las personas son capaces de
adaptarse a los medios que disponen, con cierto grado de aceptación, para poder
lograr la comunicar lo que desean. Además, se concluye que el medio de
comunicación utilizado cambia la naturaleza de la comunicación y la naturaleza
de la comunicación, pudiendo pasar de una comunicación menos social a una
enfocada a los tópicos a tratar.

Hoy en día los ambientes de trabajo en el mundo laboral no son lo mismo que
hace años atrás. Tradicionalmente los empresarios han organizado el trabajo de
forma centralizada, sin importar del tipo de trabajo llevado a cabo por cada
uno de los miembros de la empresa. Ahora, la tendencia es la descentralización
e individualización de las diferentes funciones desarrolladas, refiriéndose
con ello a la división de las funciones en los individuos de un grupo de
trabajo, a la delegación y la subdivisión de las diferentes tareas a
realizar por una empresa.
Por ello, mantener una comunicación fluida que evite la pérdida de
identificación de cada trabajador con los objetivos de la empresa o grupo de
trabajo se vuelve de suma importancia.
Para poder evitar estas situaciones, se espera encontrar similitudes que
permitan mejorar las falencias de los modelos de trabajo y la distanciación de
sus identidades con la del grupo de trabajo\cite{trabajo_flexible}.

También se cuenta con la experiencia vivida por el grupo ALMA-UTFSM durante
su formación y desarrollo\cite{utfsm_alma}, en donde se analiza los cambios
que experimentó durante la maduración de su modelo organizacional, donde las
tecnologías de la información determinaron en gran medida las formas de
trabajo colaborativas llevadas a cabo. Se destacan las variables del ambiente
que lograron alcanzar el éxito del grupo, las cuales se deben analizar en
conjunto para entender la real importancia de cada una de ellas. El tener estudiantes
entusiastas y motivados, seleccionados tanto por sus habilidades sociales como
técnicas, y un grupo docente involucrado, que les guía y da soporte, permite
que se fomente la proactividad y que los proyectos estén en
continuo desarrollo.

