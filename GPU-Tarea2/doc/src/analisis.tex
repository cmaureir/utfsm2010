\subsection{CPU}
	Analizando la paralelización en la implementación en la CPU,
	es decir, utilizando y no utilizando el \emph{pragma}
	podemos darnos cuenta que el tiempo cambia notoriamente,
	como se puede ver en \ref{tab:1}, esto es debido a que pasa
	de utilizar los 8 procesadores a sólo utilizar 1.
	Además podemos darnos cuentas que al paralelizar,
	el tiempo \emph{Real} disminuye en casi $\frac{1}{8}$,
	pues $\frac{6.1}{8} = 0.765$ y nosotros tenemos $0.443$.

	Con respecto a los resultados de la integral, señalados también
	en \ref{tab:1}, podemos darnos cuentas que son distintos,
	esto se debe que al paralelizar no tenemos control sobre los
	8 threads (en éste caso), los cuales van calculando segmentos
	de la integral de forma independiente, por lo que al momento del cálculo,
	algunos pueden diferir un poco, ocasionando la leve diferencia.


\subsection{GPU}

%Medir cómo influye el tamaño del bloque en el tiempo de ejecución del programa.
Se realizaron distintas pruebas para ver la diferencia de valores entregados
por los distintos tamaños de los bloques y el tiempo de ejecución.

Mediante el desarrollo del programa, se encontraron varios comportamientos extraños,
los valores de \emph{Result} en la tabla~\ref{tab:2} son los obtenidos la noche del Domingo 17 de Octubre,
los cuales se acercaban al valor entregado por el cálculo en la CPU.

En cambio cuando el programa se comenzó a ejecutar hoy Lunes 18 de Octubre,
los valores comenzaron a acumularse y llegaron a números de la forma $165.2e26$
lo cual no está correcto.

Probé utilizando mejor la memoria compartida, asignándole mediante el kernel
un espacio a la memoria compartida con distintos valores $(512, N, nr\_blocks, etc)$
sin obtener buenos resultados.

A medida que vamos aumentando el tamaño del bloque podemos hacer distintas apreciaciones
con respecto al tiempo:
\begin{itemize}
	\item Real time: se mantiene cerca de valores como $1.506s$ pero no sigue un patrón.
	\item User time: pareciera que fuera aumentando pero hay algunos valores que se salen de lo común.
	\item Sys  time: se ve un parcial aumento, pero en los últimos valores, el tiempo vuelve a ser como el principio.
\end{itemize}

Finalmente se puede decir que no hay una tendencia a aumentar o disminuir el tiempo bien notorio,
pero se podría inferir que con algunos valores de bloques el programa se comporta de una manera extraña
que desordenan un supuesto patrón de aumento de tiempo.
