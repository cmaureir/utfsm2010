\documentclass[letter, 10pt]{article}
\usepackage[utf8]{inputenc}
\usepackage[spanish]{babel}
\usepackage{amsfonts}
\usepackage{amsmath}
\usepackage[dvips]{graphicx}
\usepackage{url}
\usepackage[top=3cm,bottom=3cm,left=3.5cm,right=3.5cm,footskip=1.5cm,headheight=1.5cm,headsep=.5cm,textheight=3cm]{geometry}


\begin{document}
\title{Taller de Modelos y Métodos Cuantitativos \\ \begin{Large}Implementación: Algoritmo \emph{Clonal Selection} para el problema \emph{Car Sequencing Problem}\end{Large}}
\author{Cristián D. Maureira Fredes.}
\date{\today}
\maketitle

\section{Introducción}
Una explicación breve de lo que consiste el informe.

\section{Descripción del problema}
Descripción breve del problema que se va resolver (revisar entrega anterior): Objetivo, restricciones blandas y duras, observaciones especiales. Modelo matem\'atico. Instancias a usar.  

\section{Algoritmo Propuesto}

\subsection{Representación}
Descripción detallada de la representación elegida.

\subsection{Técnica}
\textbf{Breve} descripción de la técnica que utilizará para resolver el problema, descripción del algoritmo base a utilizar.
.
\subsection{Algoritmo propuesto}
\subsubsection{Componentes}
Descripción los componentes que se probaron. Ej. PSO: inercia, velocidad m\'axima, movimientos, funciones objetivos, etc. AIS: Operadores de seleccion, Operadores de clonaci\'on, Operadores de reemplazo, movimientos, funci\'ones objetivos, etc. Resultados de las pruebas realizadas (tablas o gr\'aficos). Ejemplo:

\begin{table}
\centering
\begin{tabular}{| l | c | c | c | c |}
\hline
 fobj & primeros & ruleta & torneo & random \\
\hline
 instancia 1 & 33000 &  31000 & 30000 & 40000 \\
\hline
 instancia 2 & 56000 &  45000 & 47000 & 60000 \\
\hline
\end{tabular}
\caption{Resultados pruebas Operadores de selecci\'on}
\end{table}


\subsubsection{Algoritmo Final}
Pseudo c\'odigo del algoritmo final (con los componentes elegidos a trav\'es de las pruebas y su criterio). En lo posible adjuntar una tabla de los resultados encontrados por su algoritmo y otros algoritmos de la literatura (no es importante que sean buenos, es solo para comparar en que punto se encuentra). 

\section{Bibliograf\'ia}
Se debe referenciar todo paper citado. En caso de referenciar p\'aginas web, debe incluir fecha.

\end{document} 
