% Descripción detallada de la representación elegida.

A continuación se explica la representación del algoritmo implementado,
que consiste en un \emph{Algoritmo Inmune} utilizando \emph{Selección Clonal}.

\subsection{Representación Matemática}

La presente representación, tiene un carácter simplista, y toma la esencia del modelo
matemático señalado en la sección anterior, buscando en éste caso, dar prioridad a la satisfacción
de la restricciones del problema, de una forma más apropiada.

\begin{itemize}
	\item Parámetros
	\begin{itemize}
		\item $cN$: Número total de autos.
		\item $oN$: Número total de opciones disponibles.
		\item $tN$: Número total de tipos/clases de autos.
	\end{itemize}
	\item Variables
	\begin{itemize}
		\item $nMax_{ij}$: Número máximo de autos con la opción $i$ en una subsecuencia $j$.
		\item $n_{ij}$: Número de autos con la opción $i$ en una subsecuencia $j$.
		\item $sMax_{i}$: Tamaño de la subsecuencia $j$ donde deben haber $nMax_{ij}$ autos.
		\item $q_{k}$: Cantidad de autos del tipo/clase $k$.
		\item $types_{il}$: Booleano que indica si la opción $i$ está presente en el auto $l$.
	\end{itemize}
	\item Función Objetivo
	$$FO\ :\ Min\ \sum\limits_{i=1}^{oN} \sum\limits_{l=1}^{cN} \sum\limits_{j=0}^{sMax_{i}} types_{il}\cdot (n_{ij} - nMax_{ij}), \forall n_{ij} > nM_{ij}$$
	\item Restricciones Duras
	$$\sum\limits_{k=1}^{cN} q_{k} = cN$$
	\item Restricciones Blandas
	$$\sum\limits_{i=1}^{oN} \sum\limits_{l=1}^{cN} \sum\limits_{j=0}^{sMax_{i}} types_{il}\cdot n_{ij} \leq nMax_{ij}$$
\end{itemize}
\subsection{Representación en Estructuras de datos}
\vspace{2cm}
Actualizar una vez terminado
\vspace{2cm}
La presente representación, posee una alta similitud con la representación matemática, buscando así presentar un código simple
con respecto a la satisfacción de restricciones y función objetivo.

\begin{itemize}
	\item Individuos ($\textbf{struct}\ genotype$):
				
		Cada individuo de nuestra población, corresponde a una secuencia de autos, más algunos atributos del mismo, por lo tanto,
		para representarlo es necesaria una estructura que posee los siguientes atributos:
		\begin{itemize}
			\item $\textbf{int}\ gene[VARS]$: Arreglo de enteros que corresponden  a la secuencia de autos, en su respectivo orden.
			\item $\textbf{int}\ fitness$: Corresponde a la aptitud de cada individuo, la cual tiene un valor correspondiente a las unidades
				extra por cada opción que sobrepase al máximo de autos permitidos con una cierta opción en una determinada subsecuencia.
			\item $\textbf{int}\ fail$: Corresponde a la cantidad de restricciones blandas violadas totales.
			\item $\textbf{double}\ rfitness$: Corresponde a la aptitud relativa de cada individuo, la cual tiene un valor correspondiente
				a una probabilidad relacionada con su \emph{fitness}, la cual será explicada más adelante cuando se hable de la \emph{Selección de
				individuos}.
			\item $\textbf{double}\ cfitness$: Corresponde a la aptitud acumulativa de cada individuo, la cual tiene un valor correspondiente
				a una secuencia de probabilidades relacionadas con su \emph{fitness}, la cual será explicada más adelante cuando se hable de la \emph{Selección
				de individuos}.
		\end{itemize}

	\item Número máximo de autos por subsecuencia ($\textbf{int}\ numMaxCarOptSeq[N]$):
	
		Corresponde a un arreglo de enteros, que representa el número máximo de autos con la opción determinada, en una subsecuencia se nuestro individuo.


	\item Tamaño subsecuencia ($\textbf{int}\ sizeMaxCarOptSeq[N]$):

		Corresponde a un arreglo de enteros, que representa el tamaño de la subsecuencia donde debe haber un número máximo de opciones en un auto (numMaxCarOptSeq).

	\item Demandas y descripción de los tipos de autos ($\textbf{int}\ types[N][M]$):
	
	Corresponde a un arreglo bidimensional, que posee la siguiente información:
	\begin{itemize}
		\item $[N]$ posee sólo el valor del índice de la cantidad de tipos/clases de autos.
		\item $[M]$ posee el valor del índice anterior para cada tipo/clase , demanda del tipo de auto, y para cada opción si la posee o no (1 o 0).
	\end{itemize}
	

\end{itemize}
