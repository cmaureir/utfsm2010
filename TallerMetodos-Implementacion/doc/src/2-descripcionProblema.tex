%Descripción breve del problema que se va resolver (revisar entrega anterior):
% Objetivo, restricciones blandas y duras, observaciones especiales.
% Modelo matem\'atico. Instancias a usar. 
\subsection{Descripción}
El \emph{Car Sequencing Problem} es un problema de satisfacción de restricciones (CSP), posee la
característica de ser NP-duro~\footnote{
NP-duro es el conjunto de los problemas de decisión que contiene los problemas H
tales que todo problema L en NP puede ser transformado polinomialmente en H.
Esta clase puede ser descrita como conteniendo los problemas de decisión que son
al menos tan difíciles como un problema de NP.}
, además corresponde a un tipo de variación del problema NP-completo \emph{Job-Shop Scheduling},
pero con un uso de razonamiento automatizado, es decir, con un enfoque dedicado a estudiar y comprender
diferentes características del razonamiento, permitiendo así construir programas que le den la posibilidad
a los computadores para razonar en forma autónoma.

Siguiendo la misma idea, es válido señalar que el \emph{Car Sequencing Problem} es un tipo de problema de planificación
de tareas en una línea de ensamblaje de autos, donde cada uno es perteneciente a un clase de automóvil, debido al conjunto
de opciones y accesorios que posee (airo acondicionado, centralizado eléctrico, etc), y cada una de las opciones o
accesorios se instala en una planta distinta, por lo que el objetivos principal es el poder encontrar el orden en la
secuencia de los vehículos, preocupándonos de no exceder la capacidad de cada planta de ensamblaje y también cumplir con la demanda.

Por lo tanto, si realizamos una definición más formal de nuestro problema, podríamos decir lo siguiente:
Teniendo una lista de vehículos dada, cada uno con sus respectivas opciones requeridas,
necesitamos establecer un orden en la línea de ensamblaje, con el fin de que cada subsecuencia de $q$ vehículos
tengamos a lo más $p$ que requieren de una determinada opción. Es importante tener en consideración que los
valores de $p$ y $q$ están asociados a cada opción de los vehículos.

Con respecto a la información que el problema otorga, podemos decir que contamos con:
\begin{itemize}
    \item Cantidad de vehículos de cada tipo o clase a producir (demanda)
    \item Lista de las opciones con la cual se constituye cada tipo o clase de vehículo, la cual puede utilizar una representación
        booleana para saber si cierto tipo de automóvil posee o no una determinada opción.
    \item Capacidad de las plantas que se preocupan de instalar la determinada opción.
\end{itemize}

Nuestro objetivo principal es:
\begin{itemize}
    \item Encontrar un orden en nuestra secuencia, que sirva para minimizar el costo por cada restricción insatisfecha.
\end{itemize}

Con respecto a las restricciones, tenemos que:
\begin{itemize}
    \item En cada subsecuencia de los $q$ vehículos, a lo más pueden haber $p$ que requieran de la opción determinada.
        Donde $p$ y $q$ son valores asociados a cada opción.
    \item La capacidad de cada planta de ensamblaje no puede ser excedida, es decir, cumplir con la demanda de cada automóvil
        sin abusar de una planta determinada.
    \item Por cada tipo de auto, el numero de autos de ese tipo debe ser secuenciado, es decir, todos los automóviles de cada clase
        deben estar presente en una secuencia determinada.
\end{itemize}

\vspace{4cm}

Modelo matematica

Instancias
