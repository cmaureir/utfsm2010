% Descripción los componentes que se probaron.
% Ej. AIS: Operadores de seleccion,
%		   Operadores de clonacion,
%		   Operadores de reemplazo,
%		   Movimientos,
%		   Funciones objetivos, etc.
% Resultados de las pruebas realizadas (tablas o gráficos). Ejemplo:
%	\begin{table}
%	\centering
%	\begin{tabular}{| l | c | c | c | c |}
%	\hline
%	 fobj & primeros & ruleta & torneo & random \\
%	\hline
%	 instancia 1 & 33000 &  31000 & 30000 & 40000 \\
%	\hline
%	 instancia 2 & 56000 &  45000 & 47000 & 60000 \\
%	\hline
%	\end{tabular}
%	\caption{Resultados pruebas Operadores de selecci\'on}
%	\end{table}
\begin{enumerate}
	\item \textbf{Operador de Selección}
		% 1. Seleccion por ruleta

		Al momento de seleccionar individuos para realizar la clonación, se utilizó la conocida técnica llamada
		\emph{roulette wheel}, para una Función Objetivo que minimiza.
		
		El procedimiento es bien simple, sólo tenemos que considerar el \emph{fitness} de cada linfocito y calcular un
		\emph{fitness relativo} de la siguiente forma:
		
		$$relativeFitness_{i}\ = \frac{f_{min} + f_{max} - f_{i}}{\sum\limits_{i=0}^{sizePop} (f_{min} + f_{max} - f_{i}}$$
		
		Donde $f_{max}$ equivale al \emph{fitness} del mejor linfocito,
		$f_{min}$ equivale al \emph{fitness} del peor linfocito y
		$f_{i}$ equivale al \emph{fitness} del i-ésimo linfocito de nuestra población.
		
		La suma de todos los \emph{fitness relativo} equivale a $1$.
			
		Luego de que cada linfocito posee su \emph{fitness relativo}, se procede a calcular un \emph{fitness acumulativo},
		es decir, ir sumando las probabilidades para generar un rango entre $0$ y $1$ con todas nuestras probabilidades.
			
		Una vez se tiene el \emph{fitness acumulativo} listo, se procede a obtener un número aleatorio entre $0$ y $1$,
		para que luego sea ubicado en nuestro rango, y así el linfocito que salga escogido con éste número aleatorio, será
		elegido para pasar ahora a la transformación.

		% 2. Seleccionar siempre los mejores
		% Pendiente

		% Tabla comparativa

	\item \textbf{Operador de Clonación}
		% 1. Clonación una cantidad aleatoria de veces
		% Pendiente


		% 2. Clonacion por fórmula
		% Pendiente
			

		% Tabla comparativa
	\item \textbf{Operador de Reemplazo}
		% 1. Reemplazo aleatorio
		% Pendiente


		% 2. Reemplazo de los más malos
		% Pendiente


		% Tabla comparativa
	\item \textbf{Movimiento}
		% 1. Swap con 10%
		% 2. Swap con 20%

		En éste caso se realiza un \emph{swap}, pero como cada individuo es del orden de $200$, $300$ y $400$ autos,
		se realiza una cantidad de \emph{swap} equivalente al $10\%$ y $20\%$ de la cantidad de autos.
	
		Para ver que elementos hacen el \emph{swap}, se eligen aleatoriamente dos elementos para intercambiar.~\footnote{
		Éste proceso se podría mejorar, estudiando a fondo que swaps nos convienen más, para obtener siempre mejores
		resultados}


		% Tabla comparativa
\end{enumerate}
