% Descripción los componentes que se probaron.
% Ej. AIS: Operadores de seleccion,
%		   Operadores de clonacion,
%		   Operadores de reemplazo,
%		   Movimientos,
%		   Funciones objetivos, etc.
% Resultados de las pruebas realizadas (tablas o gráficos). Ejemplo:
%	\begin{table}
%	\centering
%	\begin{tabular}{| l | c | c | c | c |}
%	\hline
%	 fobj & primeros & ruleta & torneo & random \\
%	\hline
%	 instancia 1 & 33000 &  31000 & 30000 & 40000 \\
%	\hline
%	 instancia 2 & 56000 &  45000 & 47000 & 60000 \\
%	\hline
%	\end{tabular}
%	\caption{Resultados pruebas Operadores de selecci\'on}
%	\end{table}

Debido a la cantidad de elementos que varían para probar el desempeño
del algoritmo, se ha establecido un conjunto de parámetros que hacen
referencia a la ejecución "base" del algoritmo.
\begin{itemize}
	\item Repertorio Inicial: Arreglado (Cumpliendo restricciones duras).
	\item Selección de Clones: Mejores (Considerando la población actual y la de clones).
	\item Movimiento: Swap al 20\%.
	\item Reemplazo: Peores.
	\item Clonación: Mediante fórmula (explicada más adelante).
\end{itemize}

\begin{enumerate}
	\item \textbf{Repertorio Inicial}

		El repertorio inicial, se ha generado de dos formas para poder establecer una mejor opción.
		\begin{itemize}
			\item \emph{Aleatorio:}
	
				Para éste caso, la población se ha generado completamente aleatoria, es decir,
				se ha creado una secuencia de vehículos considerando los tipos que el problema
				requiere, pero no se ha verificado que cada tipo de vehículo sea la cantidad
				necesitada por el problema.
			\item \emph{Arreglado:} 
				
				Se ha generado la población inicial con una lista de vehículos que contiene
				la cantidad exacta de los tipos necesitados por el problema, los cuales
				han sido desordenados, para dejar satisfechas en todo momento las restricciones
				blandas.
		\end{itemize}

	\item \textbf{Operador de Selección}
		% 1. Seleccion por ruleta
		La cantidad de elementos seleccionados corresponde al $selRate$ que consiste en una cantidad que corresponde
		al $40\%$ de la población.

		\begin{itemize}
			\item \emph{Roulette wheel:}
 
				Al momento de seleccionar individuos para realizar la clonación, se utilizó la conocida técnica llamada
				\emph{roulette wheel}, para una Función Objetivo que minimiza.
				
				El procedimiento es bien simple, sólo tenemos que considerar el \emph{fitness} de cada linfocito y calcular un
				\emph{fitness relativo} de la siguiente forma:
				
				$$relativeFitness_{i}\ = \frac{f_{min} + f_{max} - f_{i}}{\sum\limits_{i=0}^{sizePop} (f_{min} + f_{max} - f_{i}}$$
				
				Donde $f_{max}$ equivale al \emph{fitness} del mejor linfocito,
				$f_{min}$ equivale al \emph{fitness} del peor linfocito y
				$f_{i}$ equivale al \emph{fitness} del i-ésimo linfocito de nuestra población.
				
				La suma de todos los \emph{fitness relativo} equivale a $1$.
					
				Luego de que cada linfocito posee su \emph{fitness relativo}, se procede a calcular un \emph{fitness acumulativo},
				es decir, ir sumando las probabilidades para generar un rango entre $0$ y $1$ con todas nuestras probabilidades.
					
				Una vez se tiene el \emph{fitness acumulativo} listo, se procede a obtener un número aleatorio entre $0$ y $1$,
				para que luego sea ubicado en nuestro rango, y así el linfocito que salga escogido con éste número aleatorio, será
				elegido para pasar ahora a la transformación.
		
				% 2. Seleccionar siempre los mejores
				% Pendiente
			\item \emph{Mejores:}
				La selección de clones se realiza considerando todos los individuos de la población inicial y también
				la lista de clones que ya han sido hipermutados, por lo cual detrás del procedimiento está inmerso una
				escencia de elitismo, pues no estamos perdiendo los buenos elementos que hayan quedado en la población
				antes de la clonacion e hipermutación.

		% Tabla comparativa
		\end{itemize}
	\item \textbf{Operador de Clonación}
		\begin{itemize}
			% 1. Clonación una cantidad aleatoria de veces
			% Pendiente
			\item \emph{Fórmula:}

				Se ha tomado la escencia de la fórmula propuesta por (REFERENCIA CLONALG),
				en el cual consideramos tres elementos: ``factor de clonación ($\beta$)~\footnote{$clonationRate$
				corresponde al tamaño del $50\%$ de la población}'', ``Total de anticuerpos($N$)'' y
				``Afinidad (ranking) ($a$)''.
				La relación está dada por un número $m$ que equivale a la cantidad de clones que se generarán
				para cada anticuerpo ordenados por afinidad, partiendo del mejor al peor.
				$$m\ =\ \lceil\frac{\beta \cdot N}{a}\rceil$$

				Con ésto estamos siguiendo la idea central del algoritmo, pues estamos favoreciendo a que se clonen más
				los mejores elementos de nuestra población.
				
			% 2. Clonacion por fórmula
			% Pendiente
			\item \emph{Secuencia fija:}
			
				En éste caso, se considera una cantidad $n$ que consiste al $50\%$ de la diferencia entre el total de la población
				y los elementos que han sido seleccionados para la clonación (elementos nulos), luego utilizan por cada loop
				dicha cantidad, pero descontando el $10\%$ del total de la población, por lo que vamos obteniendo una cantidad
				de clones que va en disminución, dando prioridad a los mejores clones.
			
		\end{itemize}
		% Tabla comparativa
	\item \textbf{Operador de Reemplazo}
		Para mantener la diversidad en nuestra población e intentar escapar de los óptimos locales,
		se utiliza un $replaceRate$ que corresponde a la cantidad del $60\%$ del tamaño de la población.

		Los nuevos elementos son generados cumpliendo las restricciones duras, para obtener nuevos
		elementos un poco mejores en comparación a la creación aleatoria.
		\begin{itemize}
			% 1. Reemplazo aleatorio
			% Pendiente
			\item \emph{Aleatorio:}
 
				Se realiza el reemplazo en posiciones aleatorias de la poblacion de los nuevos elementos
				generados.

			% 2. Reemplazo de los más malos
			% Pendiente
			\item \emph{Peores:}
				Se realiza el reemplazo en los peores elementos de la nueva población a utilizar.

		\end{itemize}
		% Tabla comparativa
	\item \textbf{Movimiento}
		% 1. Swap con 20%
		% 2. Swap con 40%

		En éste caso se realiza un \emph{swap}, pero como cada individuo es del orden de $200$, $300$ y $400$ autos,
		se realiza una cantidad de \emph{swap} equivalente al $20\%$ y $40\%$ de la cantidad de autos.
	
		Para ver que elementos hacen el \emph{swap}, se eligen aleatoriamente dos elementos para intercambiar.~\footnote{
		Éste proceso se podría mejorar, estudiando a fondo que swaps nos convienen más, para obtener siempre mejores
		resultados}


		% Tabla comparativa
\end{enumerate}
