\frame
{
\frametitle{Descripción del Problema}
\framesubtitle{Información}

Con respecto a la información que el problema otorga, podemos decir que contamos con:
\begin{itemize}
    \item \green{Cantidad} de vehículos de cada \blue{tipo} o clase a producir (demanda)
    \item Lista de las \green{opciones} con la cual se \blue{constituye cada tipo} o clase
		de vehículo.
    \item Capacidad de las plantas que se preocupan de instalar la determinada opción.
\end{itemize}
}

\frame
{
\frametitle{Descripción del Problema}
\framesubtitle{Objetivo Principal}
\begin{itemize}
    \item Encontrar un \red{orden} en nuestra secuencia,
		que sirva para \blue{minimizar} el costo por cada \green{restricción} insatisfecha.
\end{itemize}
}

\frame
{
\frametitle{Descripción del Problema}
\framesubtitle{Objetivo Principal}
Con respecto a las restricciones, tenemos que:
\begin{itemize}
    \item En cada subsecuencia de los $q$ vehículos,
		\blue{a lo más} pueden haber $p$ que \red{requieran de la opción} determinada.
		(por cada opción).
    \item La \red{capacidad} de cada \blue{planta de ensamblaje} no puede ser excedida.
    \item Por cada \red{tipo} de auto, el numero de autos de ese tipo debe ser
		\blue{secuenciado} (restricción dura).
\end{itemize}
}

\frame
{
\frametitle{Descripción del Problema}
\framesubtitle{Objetivo Principal}
Las instancias utilizadas son las de entregadas por la \red{CSPlib}.
\begin{itemize}
    \item 10 instancias para \blue{200} vehículos.
    \item 10 instancias para \blue{300} vehículos.
    \item 10 instancias para \blue{400} vehículos.
\end{itemize}
}
