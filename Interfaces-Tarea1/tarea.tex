\documentclass[letter, 10pt]{article}
\usepackage[utf8]{inputenc}
\usepackage[spanish]{babel}
\usepackage{amsfonts}
\usepackage{amsmath}
\usepackage[dvips]{graphicx}
\usepackage{url}
\usepackage[top=1cm,bottom=1.5cm,left=2cm,right=2cm,footskip=1.5cm,headheight=1.5cm,headsep=.5cm,textheight=3cm]{geometry}
\usepackage{fancyhdr}

%\pagestyle{fancyplain}
%\lhead{Departamento de Informática}
%\rhead{Universidad Técnica Federico Santa María}
%\makeatletter

\begin{document}
\title{\small{Universidad Técnica Federico Santa María\\Departamento de Informática}\\ \vspace{0.5cm}\huge{Diseño de Interfaces Usuarias}\\\Large{\emph{``Escenarios''}}}

\author{
\begin{tabular}{cc}
	\begin{tabular}{c}
		Cristián Maureira Fredes\\
		\url{cmaureir@inf.utfsm.cl}\\
		(Alumno)
	\end{tabular}
	&
	\begin{tabular}{c}
		Lioubov Dombrovskaia\\
		\url{lioubov.dombrovskaia@usm.cl}\\
		(Profesora)
	\end{tabular}
\end{tabular}}
\date{\today}
\maketitle

\section{Tarea}
Tomando en cuenta la tarea pasada, que constaba en \emph{reservar una cita} en la página~\url{http://www.agendy.com},
es necesario ahora realizar un escenario de cómo \textbf{deber\'ia} ser el procedimiento para reservar una cita con un consultor.

\subsection{Escenario}
\begin{enumerate}
   \item El usuario entra al sitio del consultor \url{http://www.agendy.com/consultor}.
   \item Una vez la página esté cargada completamente se presiona el botón \emph{reservar}.
   \item El sistema presenta al usuario un mensaje que dice \emph{``para reservar una cita con el consultor,
	se debe estar autentificado''}, presentando dos opciones, \emph{Log In} y \emph{Registrar}.
   \item El usuario \textbf{no} posee una cuenta. (Registrar)
   \begin{enumerate}
	\item El sistema despliega una pantalla pidiendo los siguiente datos:
		\emph{Nombres, Apellidos, Email, Telefono}. El usuario los ingresa mediante el teclado.
	\item El usuario presiona el botón de \emph{Crear cuenta} a lo que el sistema despliega una nueva pantalla
		advirtiendo al usuario que se ha enviado un correo electrónico a su \emph{Email} para confirmar
		la nueva cuenta creada.
	\item El usuario al leer el correo electrónico, presiona el link que viene adjunto y confirma así su cuenta.
	\item El sistema autentica al usuario.
	\item El usuario vuelve a la página del consultor y presiona el botón de reservar. (Continuar con el paso \emph{5.d})
   \end{enumerate}
   \item El usuario posee una cuenta. (Log In)
   \begin{enumerate}
	\item El usuario selecciona \emph{Log In}.
	\item El sistema solicita al usuario ingresar su \emph{email} y \emph{contraseña}, el usuario ingresa
		dichos datos mediante el teclado.
	\item El sistema autentica al usuario.
	\item El sistema despliega una lista de los servicios del presente \emph{consultor}, el usuario selecciona
		el servicio que necesita. 
	\item El sistema ofrece un calendario para que el usuario seleccione la \emph{fecha} que más le acomode. El usuario
		selecciona la \emph{fecha} en la que tiene disponibilidad.
	\item El sistema despliega una lista con los \emph{horarios} disponibles para el día previamente seleccionado. El
		usuario selecciona la \emph{hora} que le acomoda o puede \emph{volver atrás} si ninguna hora le sirve.
	\item Al seleccionar la \emph{hora}, el sistema despliega una pantalla advirtiendo al usuario que está reservando
		una cita con la \emph{fecha} y \emph{hora} seleccionadas anteriormente, para que el usuario pueda \emph{confirmar} la información.
		El usuario presiona el botón de \emph{confirmar}.
	\item El sistema despliega una nueva pantalla informando al usuario que se ha realizado la reservación para la cita
		exitosamente.
   \end{enumerate}
\end{enumerate}

\end{document} 
