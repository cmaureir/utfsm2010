Retome las 6 proteínas encontradas en la tarea 1 (los primeros 6 matches que les dio BLAST)

\begin{enumerate}
	\item Alinee sus secuencias usando Clustal (www.clustal.org). Muestre el alineamiento. ¿Quedan
		alineados entre sí los segmentos que se alineaban con su nombre?
	
		\blue{Respuesta:}

		Se seleccionó 3 proteínas por cada integrante, como se señaló en la clase.\\

		\begin{center}
		\begin{tabular}{|l|l|}
			\hline
			GARIELAMRA & CRISTIANMAREIRA \\
			\hline
			ZP\_01751021.1 & EFO18319.1 \\ 
			YP\_001613385.1 & AAK08622.1 \\
			ZP\_05971288.1 & ZP\_05842161.1 \\
			\hline
		\end{tabular}
		\end{center}

	El alineamiento puede verse en el Anexo III~\ref{sec:anexo3}.No quedan alineados los nombres.
			

	\item Corte, de cada secuencia, el segmento que se alinea con su nombre. Con las 7 secuencias,
		haga un nuevo alineamiento en Clustal.

		\blue{Respuesta:}

		Debido a que somos dos integrantes,
		tomaremos las 3 primeras secuencias encontraras por cada uno y el
		segmento GARIELAMRA.

		\red{Importante:} El alineamiento puede verse en el Anexo IV~\ref{sec:anexo4}.

	\item Usando ese alineamiento, construya y dibuje (“a mano”, es decir, sin usar software de HMM)
		un HMM.

		\blue{Respuesta:}

		\red{Importante:} La HMM puede verse en el Anexo V~\ref{sec:anexo5}.

	\item Determine la secuencia de estados internos más probable en ese HMM para emitir su
		nombre.

		\blue{Respuesta:}

		\begin{center}
		C G A R I E L A M R A A
		\end{center}

\end{enumerate}
