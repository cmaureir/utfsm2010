\begin{enumerate}

	\item Siga programando en su lenguaje favorito. Esta vez, haga un programa que reciba una
		secuencia de DNA y encuentre en ella los 10 palíndromes más largos (un palíndrome, en
		sentido clásico, es una palabra que se lee igual en orden inverso; en el caso de DNA, además se
		complementan las bases, de modo que ACCTGATCAGGT es un palíndrome.).

		\blue{Respuesta:}
		
		Ver anexo 2~\ref{sec:anexo2}

	\item Vaya a~\footnote{\url{http://www.ncbi.nlm.nih.gov/genomes/genlist.cgi?taxid=2g\&type=0\&name=Complete\%20Bacteria}}
		y elija un cromosoma de bacteria (en general corresponderán a genomas completos de las bacterias),
		de no menos de 1.000.000 nt. Aplíquele su programa de la parte (a).

		\textbf{Ojo:} aparte de los resultados, especifique bien (con link y código de acceso) la secuencia que usó.

		\blue{Respuesta:}
		\begin{itemize}
			\item Información General
				\begin{itemize}
					\item Organism: \emph{Anaplasma centrale str. Israel}~\footnote{\url{http://www.ncbi.nlm.nih.gov/Taxonomy/Browser/wwwtax.cgi?id=574556}}
					\item Name: \emph{chromosome}
					\item Accession: \emph{NC\_013532}~\footnote{\url{http://www.ncbi.nlm.nih.gov/sites/entrez?db=genome&cmd=Retrieve&dopt=Overview&list_uids=25444}}
					\item Lenght: \emph{1206806 nt}
					\item Proteins:	\emph{923}~\footnote{\url{http://www.ncbi.nlm.nih.gov/sites/entrez?db=genome&cmd=Retrieve&dopt=Protein+Table&list_uids=25444}}
					\item RNAs: \emph{40}~\footnote{\url{http://www.ncbi.nlm.nih.gov/sites/entrez?db=genome&cmd=Retrieve&dopt=Structural+RNA+Table&list_uids=25444}}
					\item Genes: \emph{982}~\footnote{\url{http://www.ncbi.nlm.nih.gov/sites/gene?term=NC_013532[accn]}}
					\item Create date: \emph{Nov 26 2009}
					\item Update date: \emph{Mar 23 2010}
				\end{itemize}
			\item Secuencia
				\begin{itemize}
					\item Anaplasma centrale str. Israel, complete genome (NCBI)~\footnote{\url{http://www.ncbi.nlm.nih.gov/nuccore/CP001759}},
						notar que es necesario habilitar en el panel derecho ``Customize view'' y habilitar ``Show sequence'',
						la que aparecerá al final de la página.\\

					\texttt{
					ORIGIN\\ 
		        1 gtgggggggt ttatgccttt agaacagcag actactgata actccaatcc tgggttgaaa\\
		      	61 aaaaactcac atgccaaggg cgccagagag ccaaacgatg agcgttggac cacaaacgat\\
				...\\
				1206721 cagacctgta taaatccgcc ggggttgatg atgcactgct catcaatcgg ggggatttgt\\
  				1206781 tcttttttgt cgtatgtatt caaaac\\
					//
					}
				\end{itemize}

			\item Aplicación:\\

				Aplicando nuestra implementación, los resultados obtenidos son los siguientes:

				\begin{itemize}
					\item \texttt{gggaccaaggggtggttataaccaccccttggtccc}
					\item \texttt{acgggacactatagtgtcccgt}
					\item \texttt{gctataattgcaattatagc}
					\item \texttt{ggcatcatgcatgatgcc}
					\item \texttt{gggatgcagctgcatccc}
					\item \texttt{agacgttggccaacgtct}
					\item \texttt{tgtttgtatatacaaaca}
					\item \texttt{tcagtaagcttactga}
					\item \texttt{caagattgcaatcttg}
					\item \texttt{tctttgccggcaaaga}
				\end{itemize}
		
				El problema fue que el encontrar los palíndromos tomó \texttt{19414.7 [sec]},
				lo que es bastante tiempo, pero considerando que la primera aplicación que se
				programó estuvo corriendo todo un día, sin obtener resultados, es mucho más
				óptima a nivel de tiempo,

				por otro lado, el ordenar todos los palíndromos encontrados, desde el más
				largo al más corto, se demoró \texttt{24 [sec]}, lo cual es un muy buen
				resultado, considerando que el archivo que tenía todos los palíndromos
				pesaba \texttt{4.3 M}.
	
		\end{itemize}

	\item Sugiera una estrategia, basada en Smith-Waterman, para realizar esta tarea si nos interesaran
		palíndromes aproximados, donde se permitan reemplazos o inserciones.

			\blue{Respuesta:}

			\red{TO DO:} Completar

			Se propone el siguiente procedimiento para llevar a cabo la búsqueda
			de palíndromos aproximada.

			\begin{enumerate}
				\item Contamos con una secuencia inicial de DNA, la cual utilizaremos
					para realizar la búsqueda.
				\item Dividimos la secuencia en $n$, donde $n$ es un número par entero
					creciente, que va desde $2$ hasta $N$. (Con $N$ dependiente,
					del tamaño de la $n$-ésima parte de la secuencia, donde ésta
					no puede ser menor a $2$.
				\item Vamos tomando de a $2$ sub-secuencias contiguas, para prepararnos
					a realizar un alineamiento. Las sub-secuencias son generadas por el
					paso anterior.
				\item Por cada $2$ sub-secuencias, realizamos un alineamiento utilizando
					\emph{Smith-Waterman} modificado, para que alinie con respecto al
					complemento de una base determinada, por ejemplo un resultado sería
					de la forma:
					\begin{center}	
					\texttt{ATTTGCCGGGA}
					\texttt{TAA-CGGC--T}
					\end{center}	
				\item Como buscamos palíndromos aproximados, determinaremos un porcentaje
					$P$, para aceptar o rechazar el alineamiento, por ejemplo, un $80\%$. 
				\item Si el alineamiento pasa la prueba, entonces estamos en preciencia
					de un palíndromo aproximado, de lo contrario, se rechaza.
				\item Aumentamos el valor de $n$ y seguimos iterando en nuestro algoritmo.
			\end{enumerate}
			


\end{enumerate}
