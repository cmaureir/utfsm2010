\begin{enumerate}
\item Investigue y conteste.
	¿Cuántos megabytes requieren las secuencias
	de los genomas de un ser humano, un choclo y una mosca
	de la fruta (Drosophila melanogaster)?
	(Admitiendo agrupar varias bases en un mismo byte de
	8 bits, pero sin comprimir más allá de eso.)
	Si sólo nos interesan las secuencias que codifican
	proteínas (“CDS”), ¿cuántos megas necesitaríamos,
	en el caso humano?
	
	\blue{Respuesta:}
		Para responder, solo es necesario conocer el
		numero de pares de base.
		Debido a que el código genético consta de 4 letras,
		podemos expresar cada base en 2 bits. Por lo que: \\

		$$\left\lceil \frac{\text{Numero de pares de base} \cdot 2}{8 \cdot 1024^2} \right\rceil = megabytes$$

		\begin{center}
		\begin{tabular}{|l|l|l|}
		\hline
			\textbf{Especie}  & \textbf{Número de pares de base} & \textbf{Bytes} \\\hline
			Ser humano        & 2858015675 & 682 MB\\
			Choclo            & 2500000000 & 597 MB\\
			Mosca de la fruta & 165000000  & 40 MB\\
			Solo CDS ser humano~\footnote{Secuencias
				codificadoras de proteínas comprenden menos del $1.5\%$ del genoma humano}
							  & 42870235   & 11 MB\\\hline
		\end{tabular}
		\end{center}


\item Tenemos un cierto gen (humano) y queremos escoger un
	trocito de él (un segmento de N bases contiguas) que sea
	único dentro del genoma completo.
	Suponiendo que las bases del genoma fuesen equiprobables e
	independientes,
	¿cuál es el menor valor de N que nos garantizaría una probabilidad
	menor a $0.01$ de encontrar la misma secuencia en otro punto del
	genoma? Explique su cálculo y detalle cualquier supuesto adicional
	que haga.

	\blue{Respuesta:}

		Si el largo del genoma humano es de $L = 2858015675$,
		además si el tamaño del trozo es de $N$, habría que buscar en
		$L - N$ pares de bases, si nuestro trozo se encuentra.\\

		\begin{itemize}
			\item ¿Cuántas veces encontraremos nuestro trozo en el
				resto del genoma (Casos favorables)?\\

				$(L - N) - N + 1$
				(posiciones que podemos deslizar un string de largo $N$,
				entre $L - N$ espacios, con $L > N$)

			\item ¿Cuántos strings de 4 letras podemos formar con $L - N$
				casillas (Casos totales)?\\

				$4^{(L-N)}$

			\item ¿Ecuación?\\

				\begin{eqnarray}
					\frac{\text{encontrar al menos una vez el trozo}}{\text{todos los trozos generables}} &<& 0.01\\
					\frac{\sum_{i=1}^{L- 2N + 1} \frac{1}{L - 2N + 1}}{4^{(L-N)}} &<& 0.01\\
					\frac{1}{0.01} &<& 4^{(L-N)} \ \ \ / log_{4}\\
					N &<& L - log_{4}(100)\\
					N &<& L - 3.3219\\
				\end{eqnarray}

\end{itemize}

\item Pequeñas secuencias como la descrita en la parte anterior
	se usan en al menos dos tecnologías importantes, la PCR
	(reacción en cadena de polimerasa) y los microarrays.
	Averigue sobre esas tecnologías y explique qué tienen que
	ver con la parte (b); además comente sobre la necesidad (o no)
	de unicidad de la secuencia, en cada una.

	\begin{itemize}
		\item PCR \\

			La reacción en cadena de la polimerasa, conocida como
			PCR por sus siglas en inglés (Polymerase Chain Reaction),
			es una técnica de biología molecular desarrollada en 
			1986 por Kary Mullis, cuyo objetivo es obtener un gran
			número de copias de un fragmento de ADN particular,
			partiendo de un mínimo; en teoría basta partir de 
			una única copia de ese fragmento original, o molde.
			
			Esta técnica sirve para amplificar un fragmento de ADN;
			su utilidad es que tras la amplificación resulta mucho más
			fácil identificar con una muy alta probabilidad virus o
			bacterias causantes de una enfermedad, identificar personas
			(cadáveres) o hacer investigación científica sobre el ADN
			amplificado. Estos usos derivados de la amplificación han
			hecho que se convierta en una técnica muy extendida, con el 
			consiguiente abaratamiento del equipo necesario para
			llevarla a cabo.

		\item Microarrays \\

			Un chip de ADN (del inglés DNA microarrays) es una superficie
			sólida a la cual se unen una serie de fragmentos de ADN.
			Las superficies empleadas para fijar el ADN son muy variables
			y pueden ser vidrio, plástico e incluso chips de silicio. 
			Los arreglos de ADN son utilizadas para averiguar la expresión
			de genes, monitorizándose los niveles de miles de ellos de
			forma simultanea.
			
			La tecnología del chip de ADN es un desarrollo de una técnica
			muy usada en biología molecular, el Southern blot.
			Con esta tecnología podemos observar de forma casi instantánea
			la presencia de todos los genes del genoma de un organismo.
			De tal forma que suelen ser utilizados para identificar genes
			que producen ciertas enfermedades mediante la comparación de
			los niveles de expresión entre células sanas y células que
			están desarrollando ciertos tipos de enfermedades

		\item Relación con parte (b) \\

			En tanto PCR como microarrays, se hace necesario partir de un
			fragmento único de ADN, ya que las técnicas tratan de identificar
			el funcionamiento de tanto de virus, bacterias o el origen de
			ciertas enfermedades, por lo que un trozo que sea único,
			garantiza de cierta forma que el fragmento es el responsable
			de dichos organismos o enfermedades

	\end{itemize}

\end{enumerate}
