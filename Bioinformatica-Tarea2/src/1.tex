\begin{enumerate}
\item Investigue y conteste. ¿Cuántos megabytes requieren las secuencias de los genomas de un ser
humano, un choclo y una mosca de la fruta (Drosophila melanogaster)? (Admitiendo agrupar varias
bases en un mismo byte de 8 bits, pero sin comprimir más allá de eso.) Si sólo nos interesan las
secuencias que codifican proteínas (“CDS”), ¿cuántos megas necesitaríamos, en el caso humano?

Para responder, solo es necesario conocer el numero de pares de base. Debido a que el código genético
consta de 4 letras, podemos expresar cada base en 2 bits. Por lo que: \\

$ \lceil \frac{Numero \ de \ pares \ de \ base * 2}{8 * 1024^2} \rceil = megabytes$ \\

\begin{tabular}{|l|l|l|}
\hline
 & Numero de pares de base & Bytes \\
\hline
Ser humano & 2858015675 & 682 MB\\
Choclo & & \\
Mosca de la fruta & & \\
\hline
\end{tabular}


\item Tenemos un cierto gen (humano) y queremos escoger un trocito de él (un segmento de N bases
contiguas) que sea único dentro del genoma completo. Suponiendo que las bases del genoma fuesen
equiprobables e independientes, ¿cuál es el menor valor de N que nos garantizaría una probabilidad
menor a 0.01 de encontrar la misma secuencia en otro punto del genoma? Explique su cálculo y detalle
cualquier supuesto adicional que haga.

Si el largo del genoma humano es de L = 2858015675, además si el tamaño del trozo es de N, habría que buscar en
L - N pares de bases, si nuestro trozo se encuentra.\\

\begin{itemize}

\item ¿ Cuántas veces encontraremos nuestro trozo en el resto del genoma (Casos favorables)?

$(L - N) - N + 1$ (posiciones que podemos deslizar un string de largo N, entre L - N espacios, con $L > N$)

\item ¿ Cuántos strings de 4 letras podemos formar con L - N casillas (Casos totales)?

$4^{(L-N)}$

\item ¿ Ecuación?

$\frac{casos \ favorables}{casos \ totales} < 0.01$\\
$\leftrightarrow \frac{(L - 2N + 1}{4^{(L-N)}} = 0.01$\\
$\leftrightarrow \frac{(L - 2N + 1}{0.01} = 4^{(L-N)} \ \ \ / log_4$\\
$\leftrightarrow log_4(L - 2N + 1) - log_4(0.01) = (L-N)$\\

NO CONSIDEREEEE QUE EL STRING FUERA CORTOO Y HUBIERA ESPACIO PARA MAS DE 1 N EN L!!! (CONSULTAR AL PROFE!!)


\end{itemize}


\item Pequeñas secuencias como la descrita en la parte anterior se usan en al menos dos tecnologías
importantes, la PCR (reacción en cadena de polimerasa) y los microarrays. Averigue sobre esas
tecnologías y explique qué tienen que ver con la parte (b); además comente sobre la necesidad (o no) de
unicidad de la secuencia, en cada una.

%PCR http://es.wikipedia.org/wiki/Reacción_en_cadena_de_la_polimerasa
%microarrays http://es.wikipedia.org/wiki/Chip_de_ADN

\end{enumerate}
