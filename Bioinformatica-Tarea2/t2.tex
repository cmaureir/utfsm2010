\documentclass[letter, 10pt]{article}
\usepackage[utf8]{inputenc}
\usepackage[spanish]{babel}
\usepackage{amsfonts}
\usepackage{amsmath}
\usepackage[dvips]{graphicx}
\usepackage{url}
\usepackage[top=3cm,bottom=3cm,left=3.5cm,right=3.5cm,footskip=1.5cm,headheight=1.5cm,headsep=.5cm,textheight=3cm]{geometry}
\usepackage{listings}
\usepackage{color}
\usepackage{fancyvrb}
\usepackage{fancyhdr}

%%%%%%%%%%%%%%%%%%%%%%
%Estilo del documento%
%%%%%%%%%%%%%%%%%%%%%%
\pagestyle{fancyplain}

%%%%%%%%%%%%%%%%%%%%%%%%%%%%%%%%%%%%%%%%%%%
%Fancyheadings. Top y Bottom del documento%
%%%%%%%%%%%%%%%%%%%%%%%%%%%%%%%%%%%%%%%%%%%
% Recuerde que en este documento la portada del documento no posee
% numeracion, pero de igual manera llamaremos a esa primera pagina la numero
% 1, y la que viene la dos. Esto es para tener una idea de las que
% llamaremos pares e impares
\lhead{Comportamiento Organizacional} %Parte superior izquierda
\rhead{\bf \it Entrevista Apreciativa} %Parte superior derecha
\lfoot{} %Parte inferior izquierda.
\cfoot{} %Parte inferior central
\rfoot{\bf \thepage} %Parte inferior derecha
\renewcommand{\footrulewidth}{0.4pt} %Linea de separacion inferior


\begin{document}
\bibliographystyle{plain}

%%%%%%%%%%%%%%%%%%%%%%%%%%
%Definicion de la portada%
%%%%%%%%%%%%%%%%%%%%%%%%%%
\begin{titlepage}
    \begin{center}
	\begin{tabular}{ccc}
	    \includegraphics[width=3cm]{img/utfsm}
	    & 
	    \hspace{-0.2cm}
	    \begin{tabular}{c}
		Universidad Técnica Federico Santa María \\ \hline
		\vspace{0.2cm}
		Departamento de Informática\\
		\vspace{1.2cm}
	    \end{tabular}
	    \hspace{0.2cm}
	    &
            \includegraphics[width=2cm]{img/di}
	\end{tabular}

	\vspace{1cm}
	%Titulo del Documento
	\begin{tabular}{c}
		\huge{\sc{Entrevista Apreciativa - Parte 2}}\\\\
		\huge{\sc{Empresa Elian y Cía., Ltda.}}\\\\
		%\includegraphics[scale=0.7]{img/portada} \\\\
	\end{tabular}

    \vspace{3cm}
	\begin{tabular}{c}
		\large{\sc{Integrantes}}
	\end{tabular}
	\\
	\begin{tabular}{cr}
         	\normalsize{Rodrigo Fernández Gaete}  & \url{rfernand@csrg.inf.utfsm.cl}\\
         	\normalsize{Cristián Maureira Fredes} & \url{cmaureir@csrg.inf.utfsm.cl}\\
         	\normalsize{Gabriel Zamora Nelson} & \url{gzamora@csrg.inf.utfsm.cl}\\
	\end{tabular}

\vspace{5cm}

	\begin{tabular}{ll}
         	\normalsize \textbf{Asignatura:} & {Comportamiento Organizacional}\\
            \normalsize \textbf{Profesor:}   & {Lautaro Guerra}\\
         	\normalsize \textbf{Fecha de Entrega:} & {14 de Septiembre del 2010}\\
	\end{tabular}
	%Fecha
    %\normalsize{\sc{\today}}\\
    \end{center}
\end{titlepage}


\section{Problema 1}	
\begin{enumerate}
\item Investigue y conteste. ¿Cuántos megabytes requieren las secuencias de los genomas de un ser
humano, un choclo y una mosca de la fruta (Drosophila melanogaster)? (Admitiendo agrupar varias
bases en un mismo byte de 8 bits, pero sin comprimir más allá de eso.) Si sólo nos interesan las
secuencias que codifican proteínas (“CDS”), ¿cuántos megas necesitaríamos, en el caso humano?

Para responder, solo es necesario conocer el numero de pares de base. Debido a que el código genético
consta de 4 letras, podemos expresar cada base en 2 bits. Por lo que: \\

$ \lceil \frac{Numero \ de \ pares \ de \ base * 2}{8 * 1024^2} \rceil = megabytes$ \\

\begin{tabular}{|l|l|l|}
\hline
 & Numero de pares de base & Bytes \\
\hline
Ser humano & 2858015675 & 682 MB\\
Choclo & & \\
Mosca de la fruta & & \\
\hline
\end{tabular}


\item Tenemos un cierto gen (humano) y queremos escoger un trocito de él (un segmento de N bases
contiguas) que sea único dentro del genoma completo. Suponiendo que las bases del genoma fuesen
equiprobables e independientes, ¿cuál es el menor valor de N que nos garantizaría una probabilidad
menor a 0.01 de encontrar la misma secuencia en otro punto del genoma? Explique su cálculo y detalle
cualquier supuesto adicional que haga.

Si el largo del genoma humano es de L = 2858015675, además si el tamaño del trozo es de N, habría que buscar en
L - N pares de bases, si nuestro trozo se encuentra.\\

\begin{itemize}

\item ¿ Cuántas veces encontraremos nuestro trozo en el resto del genoma (Casos favorables)?

$(L - N) - N + 1$ (posiciones que podemos deslizar un string de largo N, entre L - N espacios, con $L > N$)

\item ¿ Cuántos strings de 4 letras podemos formar con L - N casillas (Casos totales)?

$4^{(L-N)}$

\item ¿ Ecuación?

$\frac{casos \ favorables}{casos \ totales} < 0.01$\\
$\leftrightarrow \frac{(L - 2N + 1}{4^{(L-N)}} = 0.01$\\
$\leftrightarrow \frac{(L - 2N + 1}{0.01} = 4^{(L-N)} \ \ \ / log_4$\\
$\leftrightarrow log_4(L - 2N + 1) - log_4(0.01) = (L-N)$\\

NO CONSIDEREEEE QUE EL STRING FUERA CORTOO Y HUBIERA ESPACIO PARA MAS DE 1 N EN L!!! (CONSULTAR AL PROFE!!)


\end{itemize}


\item Pequeñas secuencias como la descrita en la parte anterior se usan en al menos dos tecnologías
importantes, la PCR (reacción en cadena de polimerasa) y los microarrays. Averigue sobre esas
tecnologías y explique qué tienen que ver con la parte (b); además comente sobre la necesidad (o no) de
unicidad de la secuencia, en cada una.

%PCR http://es.wikipedia.org/wiki/Reacción_en_cadena_de_la_polimerasa
%microarrays http://es.wikipedia.org/wiki/Chip_de_ADN

\end{enumerate}

\section{Problema 2}
Programe en su lenguaje favorito. Necesitará (al menos) funciones que hagan lo siguiente:
Generar una secuencia aleatoria de 200 bases equiprobables e independientes. Nos interesa la
evolución de una especie alienígena en que las bases del DNA son 6, no 4: {A,C,G,T,B,D}.
Una función que aplique una mutación a una secuencia; la mutación se escoge entre inserción,
borrado y reemplazo de manera equiprobable, y su lugar de aplicación se elige al azar a lo largo
de la secuencia. El borrado borra una letra, la inserción inserta una letra (equiprobable), y el
reemplazo reemplaza una letra por cualquiera de las otras 5 (de manera equiprobable).
Una función que calcule la distancia de Levenshtein entre dos secuencias (implementando
Needleman-Wunsch).
Con esas funciones, hará lo siguiente:

\begin{enumerate}

\item Generar una secuencia, y aplicar M mutaciones; para M entre 0 y 300, grafique la relación
entre M y D, donde D es la distancia de Levenshtein entre la secuencia final y la secuencia
inicial.


\item Genere una secuencia, clónela, y a cada copia aplíquele M mutaciones (de modo que tendrá
dos secuencias crecientemente distintas). Grafique la relación entre M y D’, donde D’ es la
distancia entre las dos secuencias que están mutando.


\item Genere 10.000 pares de secuencias (largo 200 c/u) y evalúe su distancia de Levenshtein; haga
un histograma de la distribución de estos valores, y calcule media y $\sigma$


\item Considerando (b) y (c), ¿por sobre qué valor de M diría usted que el parentesco entre las
secuencias es indetectable?


\end{enumerate}

\section{Problema 3}
\begin{enumerate}

	\item Siga programando en su lenguaje favorito. Esta vez, haga un programa que reciba una
		secuencia de DNA y encuentre en ella los 10 palíndromos más largos (un palíndromo, en
		sentido clásico, es una palabra que se lee igual en orden inverso; en el caso de DNA, además se
		complementan las bases, de modo que ACCTGATCAGGT es un palíndromo.).

		\blue{Respuesta:}
		
		Ver anexo 2~\ref{sec:anexo2}

	\item Vaya a~\footnote{\url{http://www.ncbi.nlm.nih.gov/genomes/genlist.cgi?taxid=2g\&type=0\&name=Complete\%20Bacteria}}
		y elija un cromosoma de bacteria (en general corresponderán a genomas completos de las bacterias),
		de no menos de 1.000.000 nt. Aplíquele su programa de la parte (a).

		\textbf{Ojo:} aparte de los resultados, especifique bien (con link y código de acceso) la secuencia que usó.

		\blue{Respuesta:}
		\begin{itemize}
			\item Información General
				\begin{itemize}
					\item Organismo: \emph{Anaplasma centrale str. Israel}~\footnote{\url{http://www.ncbi.nlm.nih.gov/Taxonomy/Browser/wwwtax.cgi?id=574556}}
					\item Nombre: \emph{chromosome}
					\item Accession: \emph{NC\_013532}~\footnote{\url{http://www.ncbi.nlm.nih.gov/sites/entrez?db=genome&cmd=Retrieve&dopt=Overview&list_uids=25444}}
					\item Largo: \emph{1206806 nt}
					\item Proteínas:	\emph{923}~\footnote{\url{http://www.ncbi.nlm.nih.gov/sites/entrez?db=genome&cmd=Retrieve&dopt=Protein+Table&list_uids=25444}}
					\item RNAs: \emph{40}~\footnote{\url{http://www.ncbi.nlm.nih.gov/sites/entrez?db=genome&cmd=Retrieve&dopt=Structural+RNA+Table&list_uids=25444}}
					\item Genes: \emph{982}~\footnote{\url{http://www.ncbi.nlm.nih.gov/sites/gene?term=NC_013532[accn]}}
					\item Create date: \emph{Nov 26 2009}
					\item Update date: \emph{Mar 23 2010}
				\end{itemize}
			\item Secuencia
				\begin{itemize}
					\item Anaplasma centrale str. Israel, complete genome (NCBI)~\footnote{\url{http://www.ncbi.nlm.nih.gov/nuccore/CP001759}},
						notar que es necesario habilitar en el panel derecho ``Customize view'' y habilitar ``Show sequence'',
						la que aparecerá al final de la página.\\

					\texttt{
					ORIGIN\\ 
		        1 gtgggggggt ttatgccttt agaacagcag actactgata actccaatcc tgggttgaaa\\
		      	61 aaaaactcac atgccaaggg cgccagagag ccaaacgatg agcgttggac cacaaacgat\\
				...\\
				1206721 cagacctgta taaatccgcc ggggttgatg atgcactgct catcaatcgg ggggatttgt\\
  				1206781 tcttttttgt cgtatgtatt caaaac\\
					//
					}
				\end{itemize}

			\item Aplicación:\\

				Aplicando nuestra implementación, los resultados obtenidos son los siguientes:

				\begin{itemize}
					\item \texttt{gggaccaaggggtggttataaccaccccttggtccc}
					\item \texttt{acgggacactatagtgtcccgt}
					\item \texttt{gctataattgcaattatagc}
					\item \texttt{ggcatcatgcatgatgcc}
					\item \texttt{gggatgcagctgcatccc}
					\item \texttt{agacgttggccaacgtct}
					\item \texttt{tgtttgtatatacaaaca}
					\item \texttt{tcagtaagcttactga}
					\item \texttt{caagattgcaatcttg}
					\item \texttt{tctttgccggcaaaga}
				\end{itemize}
		
				El problema fue que el encontrar los palíndromos tomó \texttt{19414.7 [sec]},
				lo que es bastante tiempo, pero considerando que la primera aplicación que se
				programó estuvo corriendo todo un día, sin obtener resultados, es mucho más
				óptima a nivel de tiempo,

				por otro lado, el ordenar todos los palíndromos encontrados, desde el más
				largo al más corto, se demoró \texttt{24 [sec]}, lo cual es un muy buen
				resultado, considerando que el archivo que tenía todos los palíndromos
				pesaba \texttt{4.3 M}.
	
		\end{itemize}

	\item Sugiera una estrategia, basada en Smith-Waterman, para realizar esta tarea si nos interesaran
		palíndromos aproximados, donde se permitan reemplazos o inserciones.

			\blue{Respuesta:}

			\red{TO DO:} Completar

			Se propone el siguiente procedimiento para llevar a cabo la búsqueda
			de palíndromos aproximada.

			\begin{enumerate}
				\item Contamos con una secuencia inicial de DNA, la cual utilizaremos
					para realizar la búsqueda.
				\item Dividimos la secuencia en $n$, donde $n$ es un número par entero
					creciente, que va desde $2$ hasta $N$. (Con $N$ dependiente,
					del tamaño de la $n$-ésima parte de la secuencia, donde ésta
					no puede ser menor a $2$.
				\item Vamos tomando de a $2$ sub-secuencias contiguas, para prepararnos
					a realizar un alineamiento. Las sub-secuencias son generadas por el
					paso anterior.
				\item Por cada $2$ sub-secuencias, realizamos un alineamiento utilizando
					\emph{Smith-Waterman} modificado, para que alinie con respecto al
					complemento de una base determinada, por ejemplo un resultado sería
					de la forma:
					\begin{center}	
					\texttt{ATTTGCCGGGA}\\
					\texttt{TAA-CGGC--T}
					\end{center}	
				\item Como buscamos palíndromos aproximados, determinaremos un porcentaje
					$P$, para aceptar o rechazar el alineamiento, por ejemplo, un $80\%$. 
				\item Si el alineamiento pasa la prueba, entonces estamos en presencia
					de un palíndromo aproximado, de lo contrario, se rechaza.
				\item Aumentamos el valor de $n$ y seguimos iterando en nuestro algoritmo.
			\end{enumerate}
			


\end{enumerate}

\section{Problema 4}
Retome las 6 proteínas encontradas en la tarea 1 (los primeros 6 matches que les dio BLAST)

\begin{enumerate}
\item Alinee sus secuencias usando Clustal (www.clustal.org). Muestre el alineamiento. ¿Quedan
alineados entre sí los segmentos que se alineaban con su nombre?


\item Corte, de cada secuencia, el segmento que se alinea con su nombre. Con las 7 secuencias,
haga un nuevo alineamiento en Clustal.


\item Usando ese alineamiento, construya y dibuje (“a mano”, es decir, sin usar software de HMM)
un HMM.


\item Determine la secuencia de estados internos más probable en ese HMM para emitir su
nombre


\end{enumerate}


\end{document}
