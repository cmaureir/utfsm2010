En el siguiente informe, se tratará de ahondar lo suficiente en el concepto de
la Cadena de Valor sin dejar de lado todas las relaciones que pudiera tener
con respecto a otro puntos que Porter nos enseñe, mediante sus documentos.

El concepto de cadena de valor, fue introducido por Michael E. Porter, Ph.D.,
académico estadounidense, que se centra en temas de economía y gerencia.

Cadena de Valor, es una forma alternativa de ver la empresa y todo lo que ella
conlleva, lo cual redefine completamente estas por partes o diferentes actividades
basicas, para que mediante ellas se puedan lograr objetivos deseados.. Obviamente
dentr de las actividades las tecnologias de la informacion pueden ser un pilar
fundamental para poder obtener una suerte de ventaja competitiva.

El examen sistemático de la modalidad que tiene un negocio para lograr una ventaja
competitiva duradera, no puede realizarse a nivel de una firma como un todo.
Es necesario reconocer las actividades de la unidad de negocios, separándolas
en etapas estratégicamente relevantes, si es que se pretende tomar plenamente
en cuenta todas las tareas llevadas a cabo para agregar valor. Estas tareas
incluyen desarrollo del producto y diseño, producción, distribución, marketing,
ventas, servicios y las muchas formas de apoyo que se necesitan para lograr la
fluidez de operación de un negocio. Un marco valioso para conseguir este objetivo
es la cadena de valor, cuyas implicaciones para lograr una ventaja competitiva han
sido exploradas a fondo por Porter, y hoy en dia forman parte de el verdadero camino
que una empresa debe seguir para poder ser exitosa.

\newpage
