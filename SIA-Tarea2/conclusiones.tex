Todas las empresas buscan lograr ventajas competitivas en su negocio, y hay muchos modos
distintos de hacerlo. La finalidad de la estrategia competitiva de una unidad de negocios de una
industria determinada es encontrar una posición en la industria desde la cual la compañía pueda
defenderse a sí misma de estas fuerzas competitivas o bien pueda influir en ellas a su favor.

La estrategia competitiva consiste en tomar medidas ofensivas o defensivas para encontrar una
posición defendible en una industria, para poder afrontar con éxito las cinco fuerzas competitivas
y de este modo conseguir un mayor rendimiento de las inversiones.

Las ventajas competitivas no se pueden comprender analizando a la empresa como un todo,
ya que en general se encuentran en el modo en que una empresa realiza sus actividades: comprar
mejor, tener mejor logíistica o un marketing de mayor calidad, etcétera.

Finalmente, el separar las actividades que realiza la empresa para analizarlas individualmente,
brinda oportunidades de mejorar la calidad con que se las realiza, hallar modos de hacerla mejor
o de agregar más valor para el cliente... o de dejar de hacerla si encontramos actividades que son
parte de la rutina pero ya no agregan valor.
\newpage
