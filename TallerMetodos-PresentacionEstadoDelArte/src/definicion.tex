%origen, objetivo, restricciones, etc.
%Incluir las cosas que a ud. le parescan relevantes del
%estado del arte del problema para su propuesta de proyecto.

\frame
{
\frametitle{Definición del problema}
\framesubtitle{Origen}
\begin{itemize}
	\item Nombrado por primera vez por B. D. Parello (1986).
	\item Enfoque en las lineas de ensablaje de vehículos.
	\item Nivel de importancia lo lleva al \emph{ROADEF challenge} (Renault).
	\begin{itemize}
		\item Pequeña variación. (Pinturas).
	\end{itemize}
\end{itemize}
}

\frame
{
\frametitle{Definición del Problema}
\framesubtitle{Características}
\begin{itemize}
	\item Problema de satisfacción de restricciones (CSP).
 	\item Posee la característica de ser NP-duro.
	\item Corresponde a un tipo de variación del problema NP-completo Job-Shop Scheduling. 
\end{itemize}
}

\frame
{
\frametitle{Definición del Problema}
\framesubtitle{Descripción}
\begin{itemize}
	\item Es un tipo de problema de {\bf planificación} de tareas en una línea de {\bf ensamblaje de autos}.
 	\item Cada uno es perteneciente a un {\bf clase de automóvil}.
	\item Debido al {\bf conjunto de opciones} y accesorios que posee (aire acondicionado, centralizado eléctrico, etc).
 	\item Cada una de las opciones o accesorios se instala en una {\bf planta} distinta.
	\begin{itemize}
		\item El {\bf objetivo} principal es el poder encontrar el {\bf orden} en la secuencia de los vehículos.
		\item Preocupándonos de no exceder la {\bf capacidad} de cada planta de ensamblaje y también cumplir con la {\bf demanda}.
	\end{itemize}
\end{itemize}
}

\frame
{
\frametitle{Definición del Problema}
\framesubtitle{Aspectos relevantes}
\begin{itemize}
	\item Buena elección de \emph{Técnicas Incompletas}, sobre las \emph{Técnicas Completas}. 
	\item Los mejores acercamientos han sido técnicas híbridas.
	\item De los distintos movimientos utilizados, depende el rendimiento del algoritmo.
	\begin{itemize}
		\item Dificultad extra el preocuparse de no romper restricciones duras.
	\end{itemize}
	\item \emph{Roullete Wheel} es la técnica de selección más usada.
	\item Previa experiencia en \emph{Algoritmos Genéticos}.
\end{itemize}
}
