\subsection{El Punto ``.''}

El punto es interpretado por el motor de búsqueda como cualquier otro carácter
excepto los caracteres que representan un salto de línea,
a menos que se le especifique esto al motor de Expresiones Regulares.

\emph{El punto se utiliza de la siguiente forma:}\\
Si se le dice al motor de RegEx que busque ``g.t'' en la cadena ``el gato de piedra en la gótica puerta de getisboro goot''
el motor de búsqueda encontrará ``gat'', ``gót'' y por último ``get''.

Nótese que el motor de búsqueda no encuentra ``goot''; esto es porque el punto representa un solo carácter y únicamente uno.

Aunque el punto es muy útil para encontrar caracteres que no conocemos,
es necesario recordar que corresponde a cualquier carácter y que muchas veces esto no es lo que se requiere.

Es muy diferente buscar cualquier carácter que buscar cualquier carácter alfanumérico
o cualquier dígito o cualquier no-dígito o cualquier no-alfanumérico.

\subsection{Los corchetes ``[ ]''}

La función de los corchetes en el lenguaje de las expresiones regulares es representar
``clases de caracteres'',
o sea,
agrupar caracteres en grupos o clases.

Son útiles cuando es necesario buscar uno de un grupo de caracteres.

Dentro de los corchetes es posible utilizar el guión ``-'' para especificar rangos de caracteres.
Adicionalmente,
los meta-caracteres pierden su significado y se convierten en literales cuando se encuentran dentro de los corchetes.

Por ejemplo para buscar cualquier carácter que representa un dígito
o un punto podemos utilizar la expresión regular ``$[\backslash d.]$''.

Los corchetes nos permiten también encontrar palabras aún si están escritas de forma errónea,
por ejemplo,
la expresión regular ``expresi$[$oó$]$n'' permite encontrar en un texto la palabra ``expresión'' aunque se haya escrito con o sin tilde.

\subsection{La barra ``$\mid$''}

Sirve para indicar una de varias opciones.
Por ejemplo,
la expresión regular ``$a\mid e$'' encontrará cualquier ``$a$'' o ``$e$'' dentro del texto.
La expresión regular ``este$\mid$oeste$\mid$norte$\mid$sur'' permitirá encontrar cualquiera de los nombres de los puntos cardinales.
La barra se utiliza comúnmente en conjunto con otros caracteres especiales.

\subsection{El signo peso ``\$''}

Representa el final de la cadena de caracteres o el final de la línea,
si se utiliza el modo multi-línea.
No representa un carácter en especial sino una posición.
Si se utiliza la expresión regular ``$\backslash .$\$'' el motor encontrará todos los lugares donde un punto finalice la línea,
lo que es útil para avanzar entre párrafos.

\subsection{El acento circunflejo ``$\wedge$''}

Este carácter tiene una doble funcionalidad,
que difiere cuando se utiliza individualmente y cuando se utiliza en conjunto con otros caracteres especiales.

En primer lugar su funcionalidad como carácter individual:
el carácter ``$\wedge$'' representa el inicio de la cadena
(de la misma forma que el signo peso ``\$'' representa el final de la cadena).
Por tanto,
si se utiliza la expresión regular ``$\wedge$[a-z]''
el motor encontrará todos los párrafos que den inicio con una letra minúscula.
Cuando se utiliza en conjunto con los corchetes de la siguiente forma ``[$\wedge\backslash w$ ]'' permite encontrar cualquier carácter
que NO se encuentre dentro del grupo indicado.
La expresión indicada permite encontrar,
por ejemplo,
cualquier carácter que no sea alfanumérico o un espacio,
es decir,
busca todos los símbolos de puntuación y demás caracteres especiales.

La utilización en conjunto de los caracteres especiales ``$\wedge$'' y ``\$'' permite realizar validaciones en forma sencilla.

Por ejemplo ``$\wedge\backslash d$'' permite asegurar que la cadena a verificar representa un único dígito inicial.

\subsection{Los paréntesis ``()''}

De forma similar que los corchetes,
los paréntesis sirven para agrupar caracteres,
sin embargo existen varias diferencias fundamentales entre los grupos
establecidos por medio de corchetes y los grupos establecidos por paréntesis:
\begin{itemize}
	\item Los caracteres especiales conservan su significado dentro de los paréntesis.
	\item Los grupos establecidos con paréntesis establecen una ``etiqueta''
	      o ``punto de referencia'' para el motor de búsqueda que puede ser utilizada
	      posteriormente como se denota más adelante.
	\item Utilizados en conjunto con la barra ``$\mid$'' permite hacer búsquedas opcionales.
	      Por ejemplo la expresión regular ``al (este$\mid$oeste$\mid$norte$\mid$sur) de''
	      permite buscar textos que den indicaciones por medio de puntos cardinales,
	      mientras que la expresión regular ``este$\mid$oeste$\mid$norte$\mid$sur'' encontraría
	      ``este'' en la palabra ``esteban'', no pudiendo cumplir con este propósito.
	\item Utilizado en conjunto con otros caracteres especiales que se detallan posteriormente,
	      ofrece funcionalidad adicional.
\end{itemize}

\subsection{El signo de interrogación ``?''}

El signo de pregunta tiene varias funciones dentro del lenguaje de las expresiones regulares.
La primera de ellas es especificar que una parte de la búsqueda es opcional.
Por ejemplo,
la expresión regular ``ob?scuridad'' permite encontrar tanto ``oscuridad'' como ``obscuridad''.
En conjunto con los parentesis redondos permite especificar que un conjunto mayor de caracteres es opcional;
por ejemplo ``Nov($\setminus$.$\mid$iembre$\mid$ember)?'' permite encontrar tanto ``Nov'' como ``Nov.'', ``Noviembre'' y ``November''.

\subsection{Las llaves ``$\{\}$''}

Comúnmente las llaves son caracteres literales cuando se utilizan por separado en una expresión regular.
Para que adquieran su función de metacaracteres
es necesario que encierren uno o varios números separados por coma
y que estén colocados a la derecha de otra expresión regular de la siguiente forma:
``$\backslash d\{2\}$'' Esta expresión le dice al motor de búsqueda que encuentre dos dígitos contiguos.
Utilizando esta fórmula podríamos convertir el ejemplo ``$\wedge\backslash d\backslash d/\backslash d\backslash d/\backslash d\backslash d\backslash d\backslash d\$$'' que servía para validar un formato de fecha
en ``$\wedge\backslash d\{2\}/\backslash d\{2\}/\backslash d\{4\}\$$'' para una mayor claridad en la lectura de la expresión.

\subsection{El asterisco ``*''}

El asterisco sirve para encontrar algo que se encuentra repetido 0 o más veces.

Por ejemplo,
utilizando la expresión ``$[a-zA-Z]\backslash d*$'' será posible encontrar tanto ``H'' como ``H1'', ``H01'', ``H100'' y ``H1000'',
es decir, una letra seguida de un número indefinido de dígitos.

Es necesario tener cuidado con el comportamiento del asterisco,
ya que este por defecto
trata de encontrar la mayor cantidad posible de caracteres que correspondan
con el patrón que se busca.

\subsection{El signo de suma ``+''}

Se utiliza para encontrar una cadena que se encuentre repetida 1 o más veces.

A diferencia del asterisco,
la expresión ``$[a-zA-Z]\backslash d+$'' encontrará ``H1'' pero no encontrará ``H''.
También es posible utilizar este metacaracter en conjunto con el signo de pregunta para limitar hasta donde se efectúa la repetición.
