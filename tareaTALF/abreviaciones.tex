\subsection{Abreviaciones}
\begin{description}
	\item \textbf{$\backslash t$\ :\ } Representa un tabulador.
	\item \textbf{$\backslash r$\ :\ } Representa el ``retorno de carro'' o ``regreso al inicio'' o sea el lugar en que la línea vuelve a iniciar.
	\item \textbf{$\backslash n$\ :\ } Representa la ``nueva línea'' el carácter por medio del cual una línea da inicio.
    	\item \textbf{$\backslash a$\ :\ } Representa una ``campana'' o ``beep'' que se produce al imprimir este carácter.
    	\item \textbf{$\backslash e$\ :\ } Representa la tecla ``Esc'' o ``Escape''
    	\item \textbf{$\backslash f$\ :\ } Representa un salto de página
    	\item \textbf{$\backslash v$\ :\ } Representa un tabulador vertical
    	\item \textbf{$\backslash x$\ :\ } Se utiliza para representar caracteres ASCII o ANSI si conoce su código.
    	\item \textbf{$\backslash u$\ :\ } Se utiliza para representar caracteres Unicode si se conoce su código.
    	\item \textbf{$\backslash d$\ :\ } Representa un dígito del 0 al 9.
    	\item \textbf{$\backslash w$\ :\ } Representa cualquier carácter alfanumérico.
    	\item \textbf{$\backslash s$\ :\ } Representa un espacio en blanco.
    	\item \textbf{$\backslash D$\ :\ } Representa cualquier carácter que no sea un dígito del 0 al 9.
    	\item \textbf{$\backslash W$\ :\ } Representa cualquier carácter no alfanumérico.
    	\item \textbf{$\backslash S$\ :\ } Representa cualquier carácter que no sea un espacio en blanco.
    	\item \textbf{$\backslash A$\ :\ } Representa el inicio de la cadena. No un carácter sino una posición.
    	\item \textbf{$\backslash Z$\ :\ } Representa el final de la cadena. No un carácter sino una posición.
    	\item \textbf{$\backslash b$\ :\ } Marca el inicio y el final de una palabra.
    	\item \textbf{$\backslash B$\ :\ } Marca la posición entre dos caracteres alfanuméricos o dos no-alfanuméricos.
\end{description}

\subsection{Conjuntos (grupos)}
\begin{description}
	\item[[a-z]] Letras minúsculas
	\item[[A-Z]] Letras mayúsculas
	\item[[0-9]] Números
	\item[[',?!¡;:$\backslash$.?]] Caracteres de puntuación
	\begin{itemize} 
		\item La barra invertida hace que no se consideren como comando ni en punto ni el interrogante
	\end{itemize}
	\item[[A-Za-z]] Letras del alfabeto
	\item[[A-Za-z0-9]] Todos los caracteres alfanuméricos habituales (sin los de puntuación)
	\item[[$\wedge$ a-z]] El símbolo $\wedge$ es el de negación. Esto es decir \emph{todo menos} las letras minúsculas.
	\item[[$\wedge$ 0-9]] Todo menos los números.
	\item[etc]
\end{description}
