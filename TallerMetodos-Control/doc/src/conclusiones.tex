Desde el comienzo del presente proyecto el objetivo principal ha sido poder
mejorar cada día la implementación del algoritmo, por lo que independiente de las pruebas
señaladas, siempre existió la preocupación y constante mejora del código correspondiente.

La primera parte del proyecto consistió netamente en una investigación tanto del problema
como de la técnica a utilizar, la que sirvió enormemente para poder comprender en la realidad
el verdadero funcionamiento de la ``Selección Clonal'', aplicado a distintos problemas,
desde uno tan simple con un \emph{Crew Scheduling} hasta la resolución de problemas del área
de la robótica para determinar los caminos de ciertos robots. De la misma forma, se comprendió
el impacto e importancia del ``Car Sequengin Problem'' en el mundo, lo que sirvió de motivación
central para el trabajo.

La segunda parte del proyecto consistió en la primera implementación del algoritmo en sí,
la cual se realizó buscando la mejor combinación de componentes para obtener buenos resultados.
Cada procedimiento del algoritmo de ``Selección Clonal'' se probó realizándolo de al menos
dos formas posibles, ya sea desde la creación de la población inicial, hasta la misma hipermutación.
Esta parte fue esencial para el resto del proyecto, debido a que una elección incorrecta de algún
componente, provocaría malos resultados más adelante.
Los resultados obtenidos en ésta parte, fueron bastante malos, lo que se pudo mejorar con creces
en las siguientes partes del proyecto.

La tercera parte consistió en la sintonización de parámetros, ésta parte fue una de las más
enriquecedoras para el algoritmo, pues los valores obtenidos mejoraron en gran cantidad,
lo cual se puede deber a que se realizó una sintonización tanto manual como automática (REVAC~\cite{REVAC}),
de ésta forma, no sólo se pudo obtener una batería de valores que poseían buenos resultados,
sino que se pudo establecer una relación entre los parámetros utilizados por el algoritmo.
La sintonización en sí, se hizo en dos partes, primero se hicieron pruebas por separado,
para luego analizar las relaciones y poder establecer una configuración determinada, utilizada
después para el control. Lo importante de la sintonización, es que gracias a ella los valores
obtenidos pudieron disminuir en más de un 50\%.

La curta parte y final, consistió en el Control de parámetros, donde se utilizaron dos
aproximaciones, una propuesta en éste informa, y otra adaptando la idea principal que expone
Riff et al~\cite{riff}, para controlar parámetros en problemas de selección clonal.
Si bien es cierto los valores obtenidos acá fueron un poco mejor por casi nada,
se obtuvo una mejora, lo cual sirve como motivación para poder trabajar más a fondo el control
de los parámetros del algoritmo. Las dos implementaciones de control no fueron técnicas muy
sofisticadas, por lo que no implicaron un gran aporte al tiempo en ejecución.


Finalmente, como trabajo futuro sería muy aconsejable poder estudiar la estructura básica
del algoritmo, pensando en poder mejorar el movimiento de la hipermutación, ya que actualmente
consiste en realizar una cantidad determinada de swaps al individuo en cuestión, proceso que podría ser
mejorado. Por otro lado, la forma de selección de individuos se encuentra actualmente desarrollada
por la ruleta, y la de los clones considerando sólo los mejores, quizás buscar otros mecanismo
de selección podrían ayudar al algoritmo, de la misma forma se podría utilizar otra técnica para poder
realizar una sintonización automática, como por ejemplo \emph{ParamILS}.
Por último y no menos importante, poder repasar el control de parámetros sería beneficioso para
la implementación, pues con un estudio más profundo se podrían llegar a mejores resultados que los obtenidos
en el presente informe.
