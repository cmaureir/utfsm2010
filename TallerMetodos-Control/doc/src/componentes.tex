Debido a la cantidad de elementos que varían para probar el desempeño
del algoritmo, se ha establecido un conjunto de parámetros que hacen
referencia a la ejecución "base" del algoritmo.
\begin{itemize}
	\item Repertorio Inicial: Arreglado (Cumpliendo restricciones duras).
	\item Selección de Clones: Mejores (Considerando la población actual y la de clones).
	\item Movimiento: Swap al 40\%.
	\item Reemplazo: Peores.
	\item Clonación: Mediante fórmula (explicada más adelante).
\end{itemize}

\begin{enumerate}
	\item \textbf{Repertorio Inicial}

		El repertorio inicial, se ha generado de dos formas para poder establecer una mejor opción.
		\begin{itemize}
			\item \emph{Arreglado:} 
				
				Se ha generado la población inicial con una lista de vehículos que contiene
				la cantidad exacta de los tipos necesitados por el problema, los cuales
				han sido desordenados, para dejar satisfechas en todo momento las restricciones
				blandas.
		\end{itemize}

	\item \textbf{Operador de Selección}
		\begin{itemize}
			\item \emph{Roulette wheel:}
 
				Al momento de seleccionar individuos para realizar la clonación, se utilizó la conocida técnica llamada
				\emph{roulette wheel}, para una Función Objetivo que minimiza.
				
				El procedimiento es bien simple, sólo tenemos que considerar el \emph{fitness} de cada linfocito y calcular un
				\emph{fitness relativo} de la siguiente forma:
				
				$$relativeFitness_{i}\ = \frac{f_{min} + f_{max} - f_{i}}{\sum\limits_{i=0}^{sizePop} (f_{min} + f_{max} - f_{i}}$$
				
				Donde $f_{max}$ equivale al \emph{fitness} del mejor linfocito,
				$f_{min}$ equivale al \emph{fitness} del peor linfocito y
				$f_{i}$ equivale al \emph{fitness} del i-ésimo linfocito de nuestra población.
				
				La suma de todos los \emph{fitness relativo} equivale a $1$.
					
				Luego de que cada linfocito posee su \emph{fitness relativo}, se procede a calcular un \emph{fitness acumulativo},
				es decir, ir sumando las probabilidades para generar un rango entre $0$ y $1$ con todas nuestras probabilidades.
					
				Una vez se tiene el \emph{fitness acumulativo} listo, se procede a obtener un número aleatorio entre $0$ y $1$,
				para que luego sea ubicado en nuestro rango, y así el linfocito que salga escogido con éste número aleatorio, será
				elegido para pasar ahora a la transformación.
		
		\end{itemize}
	\item \textbf{Operador de Clonación}
		\begin{itemize}
			\item \emph{Fórmula:}

				Se ha tomado la esencia de la fórmula propuesta por De Castro~\cite{decastro},
				en el cual consideramos tres elementos: ``factor de clonación ($\beta$)~\footnote{$clonationRate$
				corresponde al tamaño del $40\%$ de la población}'', ``Total de anticuerpos($N$)'' y
				``Afinidad (ranking) ($a$)''.
				La relación está dada por un número $m$ que equivale a la cantidad de clones que se generarán
				para cada anticuerpo ordenados por afinidad, partiendo del mejor al peor.
				$$m\ =\ \lceil\frac{\beta \cdot N}{a}\rceil$$

				Con ésto estamos siguiendo la idea central del algoritmo, pues estamos favoreciendo a que se clonen más
				los mejores elementos de nuestra población.
				
		\end{itemize}
	\item \textbf{Operador de Reemplazo}
		Para mantener la diversidad en nuestra población e intentar escapar de los óptimos locales,
		se utiliza un $replaceRate$ que corresponde a la cantidad del $60\%$ del tamaño de la población.

		Los nuevos elementos son generados cumpliendo las restricciones duras, para obtener nuevos
		elementos un poco mejores en comparación a la creación aleatoria.
		\begin{itemize}
			\item \emph{Peores:}
				Se realiza el reemplazo en los peores elementos de la nueva población a utilizar.

		\end{itemize}
		% Tabla comparativa
	\item \textbf{Movimiento}
		En éste caso se realiza un \emph{swap}, pero como cada individuo es del orden de $200$, $300$ y $400$ autos,
		se realiza una cantidad de \emph{swap} equivalente al $40\%$ de la cantidad de autos.
	
		Para ver que elementos hacen el \emph{swap}, se eligen aleatoriamente dos elementos para intercambiar.
\
\end{enumerate}


Respecto a los gráficos comparativos de cada situación,
se ha realizado 5 configuraciones distintas, para ver la diferencia en el rendimiento:
\begin{enumerate}
    \item Normal:

        \begin{itemize}
            \item Generación de población: Cumpliendo restricciones duras.
            \item Selección de clones: Selección de los mejores.
            \item Reemplazo al término de iteración:  Se reemplazan los peores.
            \item Tipo de clonación: Mediante fórmula.
        \end{itemize}
    \item Población aleatoria:

        \begin{itemize}
            \item Generación de población: \textbf{Rompiendo restricciones duras}.
            \item Selección de clones: Selección de los mejores.
            \item Reemplazo al término de iteración:  Se reemplazan los peores.
            \item Tipo de clonación: Mediante fórmula.
        \end{itemize}

    \item Selección ruleta:

        \begin{itemize}
            \item Generación de población: Cumpliendo restricciones duras.
            \item Selección de clones: \textbf{Selección mediante ruleta}.
            \item Reemplazo al término de iteración:  Se reemplazan los peores.
            \item Tipo de clonación: Mediante fórmula.
        \end{itemize}

    \item Reemplazo aleatorio:

        \begin{itemize}
            \item Generación de población: Cumpliendo restricciones duras.
            \item Selección de clones: Selección de los mejores.
            \item Reemplazo al término de iteración:  \textbf{Se reemplazan aleatoriamente}.
            \item Tipo de clonación: Mediante fórmula.
        \end{itemize}

    \item Secuencia Fija:

        \begin{itemize}
            \item Generación de población: Cumpliendo restricciones duras.
            \item Selección de clones: Selección de los mejores.
            \item Reemplazo al término de iteración:  Se reemplazan los peores.
            \item Tipo de clonación: \textbf{Secuencia fija}.
        \end{itemize}

\end{enumerate}

Los resultados obtenidos y graficados corresponden al \emph{promedio}, \emph{desviación estándar},
\emph{valores mínimos}, \emph{valores máximos} y \emph{tiempo de ejecución}.


Ver anexo~\ref{sec:anexo}.

