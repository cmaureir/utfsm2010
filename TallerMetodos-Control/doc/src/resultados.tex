%Incluir los mejores resultados obtenidos en terminos de fitness, utilizando una tabla y si es posible comparándolos con los mejores de la literatura. Análisis de tiempo requerido para lograr esos resultados.

La siguiente tabla, posee los resultados de los siguientes procedimientos:
\begin{itemize}
	\item \textbf{Mejor resultado conocido}

		Mejor resultado conocido según la CSPlib~\cite{CSP}.
		Se entiende que éstas soluciones pueden contener los resultados de implementaciones de algoritmos completos,
		por lo que al recorrer todo el espacio de búsqueda es más probable que lleguen al mejor óptimo,
		pero quizás el tiempo les juega una mala pasada.

	\item \textbf{Implementación Inicial}

		Corresponden a los mejores resultados obtenidos cuando se realizó la primera implementación del algoritmo,
		es decir, sin considerar ningún tipo de sintonización o control de parámetros, sólo se probaron
		los desempeños de distintas técnicas para ciertos componentes del algoritmo.

	\item \textbf{Sintonización de Parámetros por separado}

		Los resultados de este procedimiento consisten en los mejores obtenidos con respecto a las distintas pruebas
		que se realizaron en la sintonización. Corresponden a la elección del mejor valor de las pruebas determinadas,
		es decir, por separado, ya que las mejoras de alguna prueba no se aplicaban a las demás.

	\item \textbf{Sintonización de Parámetros en conjunto}

		Corresponden a los resultados considerados en conjunto luego de la sintonización, considerando las mejores
		combinaciones de las pruebas realizadas. También cabe señalar que acá se cambió el mecanismo de la generación
		de números aleatorios, lo cual favoreció en enorme manera el desempeño del algoritmo.

	\item \textbf{Control de Parámetros}

		Los valores expuestos en este procedimiento, corresponden a los mejores valores obtenidos luego de realizar
		un control de parámetros de dos formas distintas, controlando distintos parámetros. Realizado después de la sintonización.
\end{itemize}

\begin{center}

\footnotesize{
\begin{tabular}{|l|c|c|c|c|c|c|}
\hline
						    & \multicolumn{2}{|c|}{\texttt{pb\_200\_01.txt}} & \multicolumn{2}{|c|}{\texttt{pb\_200\_09.txt}} & \multicolumn{2}{|c|}{\texttt{pb\_200\_10.txt}} \\ \hline
\textbf{Procedimiento} 		& \textbf{Fitness} & \textbf{Tiempo [s]} & \textbf{Fitness} & \textbf{Tiempo [s]} & \textbf{Fitness} & \textbf{Tiempo [s]} \\\hline 
Mejor resultado conocido    & 0				   & -					 &  10				& 		-			  &		19			 & 		-			   \\\hline
Implementación Inicial      & 199			   & 1.146				 &  185				& 		1.393		  &		160			 & 	1.139			   \\\hline
Sint. de Parám. por separado &  167& 15.110				 &  152				& 		11.243		  &		126			 & 	11.210			   \\\hline
Sint. de Parám. en conjunto &  	77 & 104.734			 &  71				& 		104.156		  &		55			 & 	103.314			   \\\hline
Control de Parámetros       & 75			   & 101.152			 &  64				& 		104.641		  &		63			 & 	105.684			   \\\hline
\end{tabular}
}
\end{center}

Claramente se puede observar la evolución del algoritmo en términos de los distintos procedimientos que se han realizado,
a lo largo del proyecto.

Si bien es cierto, los resultados se fueron mejorando cada vez más, estos siguen estando muy alejados
de los mejores resultados obtenidos a nivel mundial, pero se justifico anteriormente la diferencia
que podría provocar esto, con respecto a la resolución del presente problema con algoritmos completos.

De la misma forma que han mejorado los resultados, podemos ver que le tiempo de ejecución por cada procedimiento,
ha ido en crecimiento notablemente, aumentando en casi un 10000\% los tiempos iniciales,
pero a pesar de que el tiempo aumentó, no son grandes cantidades, como días o semanas, si no más bien unos pocos
segundos, lo que insta a poder aumentar los valores de algunos parámetros para que el algoritmo se tome
mucho más tiempo y pueda obtener soluciones un poco mejor.

A pesar de utilizar instancias distintas, los valores de los parámetros fueron evolucionando casi al unísono,
lo que habla bien del algoritmo, ya que no posee un comportamiento totalmente independiente de las
instancias a utilizar, si no más bien, está bien estructurado para poder funcionar de una buena forma,
independiente de la complejidad del problema.

Un análisis de los procedimientos serán abarcados en la siguiente sección.
