%Debe identificar el algoritmo sintonizador que elegirá
% y los parámetros que serán sintonizados por éste.
Luego del proceso manual anterior, es tiempo de automatizar el proceso,
que para algoritmos mucho más complejos y elaborados puede ser realmente un alivio,
a la hora de buscar los valores adecuados para los parámetros utilizados.

En la presente sección, utilizaremos un algoritmo sintonizador basado en búsqueda,
muy similar a un algoritmo genético, nos referimos a REVAC~\cite{REVAC},
el cual fue propuesto por Eiben y Nannen hace 3 años.

La idea principal de REVAC es que nos entrega una herramienta de estimación
de distribución, la cual va a requerir un rango determinado de valores para
cada parámetro que deseemos sintonizar

El ser un algoritmo genético, lo guía a utilizar un
conjunto de configuraciones de parámetros como población, la cual es inicializada
utilizando una distribución uniforme para los valores de los parámetros a sintonizar.

Algunas otras características son que en cada iteración del algoritmo, genera sólo
un hijo, a diferencia de los algoritmos evolutivos comunes, por otra parte posee
un cruzamiento uniforme considerando a los mejores elementos.
%Entropía → Relevancia de un parámetro
%Mutación con intervalo 2*H

Finalmente es necesario precisar que REVAC sólo necesitará un intervalo y precisión
de los parámetros a sintonizar, con los cuales después de cumplir todas las iteraciones,
o converger, nos entregará el valor e intervalo de los valores para cada parámetro.\\ 

\textbf{Prueba:} \blue{prueba5}\\

\textbf{Sintonizador:} \texttt{REVAC}\\

\textbf{Parámetros a Sintonizar:}\\

\begin{itemize}
	\item \texttt{clonationRate} TASA\_CL \blue{[0,1]}
	\item \texttt{clonationFactor} FACT\_CL \blue{[0,1]}
	\item \texttt{replaceRate} TASA\_RP \blue{[0,1]}
\end{itemize}

\textbf{Detalles del sintonizador:}\\

\begin{itemize}
	\item Iteraciones: \blue{200}
	\item Tamaño de población (M): \blue{18}
	\item Cantidad de padres (N): \blue{6}
	\item Intervalo de mutación (H) : \blue{3}
	\item Tipo de búsqueda: Algoritmo genético.
\end{itemize}

\textbf{Resultados:}

Los resultados entregados por REVAC para cada instancia fueron los siguientes:

\begin{itemize}
	\item \texttt{pb\_200\_01.txt}\\

		\begin{verbatim}
		TASA_CL         56      [55 .. 56]
        FACT_CL         65      [64 .. 65]
        TASA_RP         57      [56 .. 57]
		\end{verbatim}
	\item \texttt{pb\_200\_09.txt}\\

		\begin{verbatim}
        TASA_CL         95      [95 .. 95]
        FACT_CL         6       [6 .. 6]
        TASA_RP         13      [13 .. 14]
		\end{verbatim}

	\item \texttt{pb\_200\_10.txt}\\

		\begin{verbatim}
        TASA_CL         80      [80 .. 80]
        FACT_CL         66      [66 .. 66]
        TASA_RP         60      [59 .. 60]
		\end{verbatim}

\end{itemize}

Para cada instancia, se escogió el valor determinado por REVAC,
y como podemos apreciar los rangos son bastante precisos, lo que nos señala
que el algoritmo convergió a un valor determinado.

A continuación, se describe la tabla con los valores entregados
por REVAC y el \emph{fitness} asociado a cada resultado, los cuales
si bien poseen una mejora considerable, en comparación a la primera implementación,
no son tan buenos como los conseguidos en la sintonización manual,
pero esto puede deberse a que la configuración de REVAC es una simplificación de
lo recomendado por los autores.

\begin{center}
\begin{tabular}{|l|c|c|c|c|c|}
	\hline
	\textbf{Instancia} & \textbf{clonationRate} & \textbf{clonationFactor} & \textbf{replaceRate} & \textbf{Fitness} & \textbf{Tiempo [s]} \\\hline
	\texttt{pb\_200\_01.txt} & 0.56 & 0.65 & 0.57 & 203 & 161 \\\hline
	\texttt{pb\_200\_09.txt} & 0.95 & 0.60 & 0.13 & 189 & 195 \\\hline
	\texttt{pb\_200\_10.txt} & 0.80 & 0.66 & 0.60 & 159 & 169\\\hline
\end{tabular}
\end{center}

Si nos damos cuenta, los valores del ``Factor de Clonación'' son bastante parecidos,
por lo que podríamos decir que independiente de la instancia nuestro algoritmo siempre requerirá el mismo
parametro para poder definir cuantos clones obtendremos de cada anticuerpo seleccionado.

Por otra parte, los parametros utilizados por el sintonizador, no son
los recomendados por los autores, pues, si utilizamos los parámetros
que corresponden, es decir:

\textbf{Detalles del sintonizador:}\\

\begin{itemize}
	\item Iteraciones: \blue{900}
	\item Tamaño de población (M): \blue{100}
	\item Cantidad de padres (N): \blue{50}
	\item Intervalo de mutación (H) : \blue{10}
	\item Tipo de búsqueda: Algoritmo genético.
\end{itemize}

\textbf{Resultados:}

Los resultados entregados por REVAC para cada instancia fueron los siguientes:

\begin{itemize}
	\item \texttt{pb\_200\_01.txt}\\

		\begin{verbatim}
        TASA_CL         80      [77 .. 81]
        FACT_CL         79      [65 .. 83]
        TASA_RP         49      [49 .. 50]
		\end{verbatim}
	\item \texttt{pb\_200\_09.txt}\\

		\begin{verbatim}
        TASA_CL         99      [98 .. 99]
        FACT_CL         69      [59 .. 77]
        TASA_RP         72      [70 .. 73]
		\end{verbatim}

	\item \texttt{pb\_200\_10.txt}\\

		\begin{verbatim}
        TASA_CL         92      [91 .. 93]
        FACT_CL         75      [62 .. 81]
        TASA_RP         23      [20 .. 24]
		\end{verbatim}

\end{itemize}

Si nos damos cuenta, los rangos obtenidos para poder determinar el valor de los parámetros que queremos sintonizar
no son tan precisos como el caso anterior, es decir, acá no hay una convergencia notoria, lo que a primera vista
puede ser provocado por el notable aumento de la cantidad de iteraciones, y a la vez los factores $M$, $N$ y $H$.

\begin{center}
\begin{tabular}{|l|c|c|c|c|c|}
	\hline
	\textbf{Instancia} & \textbf{clonationRate} & \textbf{clonationFactor} & \textbf{replaceRate} & \textbf{Fitness} & \textbf{Tiempo [s]} \\\hline
	\texttt{pb\_200\_01.txt} & 0.80 & 0.79 & 0.49 & 123 & 676 \\\hline
	\texttt{pb\_200\_09.txt} & 0.99 & 0.69 & 0.72 & 115 & 819 \\\hline
	\texttt{pb\_200\_10.txt} & 0.92 & 0.75 & 0.23 & 92 & 710\\\hline
\end{tabular}
\end{center}

En comparación a los factores anteriores, podemos darnos cuenta que no son iguales, y que la diferencia entre algunos factores
es diminuta, pero en  otro grande, por ejemplo, para el \texttt{clonationRate} de la primera prueba de REVAC era $0.56$ para la instancia \texttt{pb\_200\_01.txt}
y en la segunda prueba, con los parámetros recomendados el valor es $0.80$, en el otro extremo tenemos el mismo parámetro,
pero para la instancia \texttt{pb\_200\_10.txt}.

Si nos ponemos a pensar en cual prueba escogeremos, claramente si miramos los \emph{fitness}, los resultados obtenidos en la segunda prueba son más beneficiosos,
y en gran magnitud.
