Michael Portes es profesor de Administración de Negocios, en la Escuela de Negocios de Harvard,
y además es una destacada autoridad en Estrategia Competitiva y Competitividad Internacional.

Porter ha escrito más de 16 libros, y también ha publicado más de 60 artículos.

Unas de las ideas principales planteadas por Porter es la de “Cadena de Valor”, ya que puede
orientar a las organizaciones para que redistribuyan sus recursos, con el fin de que mejoren el
rendimiento, y así lograr mejorar la ventaja competitiva. A continuación se hablara algo más de
este tema.

La cadena de valor describe las organizaciones como cadenas causales de actividades, que agregan
valor para los clientes mediante la transformación de insumos en entrega de productos o servicios.

Para el cliente el valor es la suma de los beneficios menos los costos.

Según Porter, al abordarse el diseño estratégico de la organización se configurará en ella una
cadena de valores en forma efectiva, para lo que se debe eliminar las actividades que no agregan
valor a los productos o servicios, y además se tiene que mejorar aquellas que lo agregan. De esta
forma, una organización en la cual se optimizó la cadena de valor está en condiciones de incrementar
sus ventajas competitivas, en materia de costos y calidad, en la medida que pueda satisfacer las
expectativas de los clientes con el mejor precio.

Porter divide las tareas existentes en la empresa en 2 grupos, que son los de actividades primarias
y actividades de apoyo. Las actividades primarias son propias de la empresa, en cambio las de apoyo
generalmente se realizan por empresas externas.

Las actividades primarias son 5, y son las siguientes:
\begin{itemize}
	\item Logística Interna.
	\item Operaciones.
	\item Logística Externa.
	\item Marketing y Ventas.
	\item Servicios.
\end{itemize}
Las actividades de apoyo son 4, y son:
\begin{itemize}
	\item Adquisiciones.
	\item Desarrollo de tecnología.
	\item Manejo de Recursos Humanos.
	\item Infraestructura de la firma.
\end{itemize}
Las actividades primarias debieran ser internas, y se convendría maximizar su rendimiento. En las
actividades de apoyo se debe definir un beneficio mínimo y variar el costo, buscar mejores ofertas en
coste, que cumplan con el beneficio mínimo, o el mejor beneficio a costo mínimo.

Con esto Porter nos dice que toda empresa se puede dividir en 9 actividades, maximizables por distintas
técnicas, para así lograr la mejor oferta para el cliente.

\newpage
