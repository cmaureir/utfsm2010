\begin{itemize}
	\item \emph{Actividades generadoras de valor:} Suma de los beneficios percibidos
		 por un cliente menos los costos en los que incurre por adquirir y usar el
		 producto o servicio (definición según Porter).
	\item \emph{Actividades de valor:}  Son las distintas actividades que puede desarrollar
		 la empresa, por margen entendemos la diferencia entre el valor total y el costo
		 colectivo de desempeñar las actividades de valor.
	\item \emph{Cadenas de valor:} Son una forma de análisis de la actividad empresarial, que
		 se basa en descomponer las funciones constitutivas, en actividades generadoras
		 de valor, buscando identificar fuentes de ventaja competitiva, ya sea reduciendo
		 costos u optando por la diferenciación.
	\item \emph{Empresa:} Es la actividad donde involucro la planeación, organización, dirección
		 y control de recursos de mano de obra, de producción, finanzas, ventas, etc, enfocando
		 los esfuerzos, en la mayoría de los casos a la obtención de lucro.
	\item \emph{Estrategía:} Es la teoría que la alta dirección tiene sobre la base para sus éxitos
		 pasados y futuros. Es una herramienta de dirección que facilita procedimientos y
		 técnicas con un basamento científico, que empleadas de manera iterativa y transfuncional,
		 contribuyen a lograr una interacción proactiva de la organización con su entorno,
		 coadyuvando a lograr efectividad en la satisfacción de las necesidades del público
		 objetivo a quien está dirigida la actividad de la misma.
	\item \emph{Gestión de la información:} Su valorización permite entender las organizaciones como
		 sistemas de interpretación y acción que incrementan el potencial competitivo, siempre
		 bajo un doble prisma individual y colectivo. es quien gestiona la compra y la carga de
		 los documentos y acaba convirtiéndose en un controlador de acceso a las informaciones
	\item \emph{Margen:} Diferencia entre el valor total y el costo colectivo de desempeñar las
		 actividades de valor.
	\item \emph{Mercadotecnia:} Actualmente representa una herramienta fundamental en la recopilación
		 y análisis de la información, para transformarla en elemento clave en la toma de decisiones,
		 que llevará sus productos o servicios con éxito hasta sus consumidores.
	\item \emph{SI (Sistema de Información):} Conjunto de componentes interrelacionados que permiten
		 capturar, procesar, almacenar y distribuir la información para apoyar la toma de decisiones
		 y el control en una institución.
	\item \emph{TI (Tecnologías de la Información):} Herramienta esencial para gestionar la productividad

	\item \emph{Ventajas competitivas:} Se logra cuando la empresa desarrolla e integra las actividades
		 de su cadena de valor de forma menos costosa y mejor diferenciada que sus rivales. 
	\item \emph{RP:} Comprendiendo qué es un proceso y cómo este forma parte integral de las empresas e
		 instituciones, cuales quiera sea su naturaleza, es posible entonces llegar a una definición.
		 Hammer y Champy definen a la RP como “la reconcepción fundamental y el rediseño radical de los
		 procesos de negocios para lograr mejoras dramáticas en medidas de desempeño tales como en costos,
		 calidad, servicio y rapidez” (Fuente: Institute of Industrial Engineers, "Más allá de la
		 Reingeniería", CECSA, México, 1995, p.4). Por lo tanto se trata de una reconcepción fundamental
		 y una visión holística de una organización. Preguntas como: ¿por qué hacemos lo que hacemos? y
		 ¿por qué lo hacemos como lo hacemos?, llevan a interiorizarse en los fundamentos de los procesos
		 de trabajo. La RP es radical hasta cierto punto, ya que busca llegar a la raíz de las cosas,
		 no se trata solamente de mejorar los procesos, sino y principalmente, busca reinventarlos.
	\item \emph{Costo:} Es la valorización monetaria de la suma de recursos y esfuerzos que han de invertirse
		 para la producción de un bien o de un servicio. El precio y gastos que tienen una cosa, sin
		 considerar ninguna ganancia.
	\item \emph{Valor:} Es la cantidad de dinero que los clientes están dispuestos a pagar por los productos o
		 servicios de la empresa.
	\item \emph{Empresa:} Es la unidad productora de bienes y servicios homogéneos para lo cual organiza y
		 combina el uso de factores de la producción.
	\item \emph{Capital:} Es el total de recursos fíisicos y financieros que posee un ente económico, obtenidos
		 mediante aportaciones de los socios o accionistas destinados a producir beneficios, utilidades
		 o ganancias.
	\item \emph{Producto:} Es el punto central de la oferta que realiza toda empresa u organización (ya sea
		 lucrativa o no) a su mercado meta para satisfacer sus necesidades y deseos, con la finalidad
		 de lograr los objetivos que persigue.
	\item \emph{Outsourcing:} Es el proceso en el cual una firma identifica una porción de su proceso de ne-
		 gocio que podrá ser desempeñada más eficientemente y/o más efectivamente por otra
		 corporación, la cual es contratada para desarrollar esa porción de negocio.

\end{itemize}
\newpage
